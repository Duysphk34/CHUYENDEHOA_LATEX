\begin{tcolorbox}
	\begin{center}
		{\GSND[\fontsize{30pt}{24pt}\selectfont\bfseries][][\maudam]{Phương pháp \lq\lq tam suất \rq \rq\\trong các bài toán hiệu suất}}
	\end{center}
\end{tcolorbox}
\subsubsection{Các bước giải toán}
	\begin{myenum}[label=\protect\khungvuong{Bước \arabic*:}]
		\item {\bfseries Viết phương trình của phản ứng ( nếu đề bài có một phương trình)\\}
		\[\begin{tikzpicture}
			\tikzset{%
				mynode/.style={%
					ultra thick,
					minimum height=0.65cm,
				},
				mymatrix/.style={%
					matrix of nodes,
					nodes={mynode},
					column 1/.style={%
						align=right,
					}
				}
			}
			\matrix (pt)[mymatrix]
			{%
				& aA &+ &bB &[1cm]&cC &+ &dD&\\
			\node[anchor=base east] {Phương trình:};&$\ldots$& &$\ldots$ &[1cm]&$\ldots$ & &$\ldots$&(1)\\
			\node[anchor=base east] {Đề cho:};	&$\ldots$ & & $\ldots$&[1cm]&$\ldots$ & &$\ldots$&(2)\\
			};
			\draw[->,>=stealth,thick] (pt-1-4) --(pt-1-6);
		\end{tikzpicture}\]
		\item {\bf Đặt các số liệu đề bài cho và đề bài hỏi vào hàng đề bài (hàng 2).\\Hạ các số liệu của phương trình xuống hàng phương trình (hàng số 1) tương ứng với số liệu đề bài .}\\
		\[\begin{tikzpicture}
			\tikzset{%
				mynode/.style={%
					ultra thick,
					minimum height=0.65cm,
				},
				mymatrix/.style={%
					matrix of nodes,
					nodes={mynode},
					column 1/.style={%
						align=right
					}
				}
			}
			\matrix (pt)[mymatrix]
			{%
				& aA &+ &bB &[1cm]&cC &+ &dD&\\
				\node[anchor=base east] {Phương trình:};&$axM_{A}$& & &[1cm]&$cxM_{C}$ & & &(1)\\
				\node[anchor=base east] {Đề cho:};	&t (gam) & & &[1cm]&m (gam)& &&(2)\\
			};
			\draw[->,>=stealth,thick] (pt-1-4) --(pt-1-6);
		\end{tikzpicture}\]
		\item {\bf Nếu bài toán cho hiệu suất thì nhân thêm hiệu suất vào chất sản phẩm ở hàng phương trình (hàng 1)}\\
			\begin{tikzpicture}
				\tikzset{%
					mynode/.style={%
						ultra thick,
						minimum height=0.65cm,
					},
					mymatrix/.style={%
						matrix of nodes,
						nodes={mynode},
						column 1/.style={%
							align=right
						}
					}
				}
				\matrix (pt)[mymatrix]
				{%
					& aA &+ &bB &[1cm]&cC &+ &dD&\\
					\node[anchor=base east] {Phương trình:};&$axM_{A}$& & &[1cm]& cx$M_{C}$xH & & &(1)\\
					\node[anchor=base east] {Đề cho:};	&t (gam) & & &[1cm]&m (gam)& &&(2)\\
				};
				\draw[->,>=stealth,thick] (pt-1-4) --(pt-1-6);
			\end{tikzpicture}
		\item {\bf Áp dụng quy tắc  \lq\lq nhân chéo chia ngang\rq\rq ~ hoặc ~ \lq\lq Nhân chéo bằng nhau\rq\rq để tìm được kết quả bài toán}
		\[\hopcttoan{t=\dfrac{a\cdot M_{A}\cdot m}{c\cdot M_{C}\cdot H}}\hspace{.5cm}\text{hoặc}\hspace{.5cm}   \hopcttoan{m=\dfrac{t\cdot c\cdot M_{C}\cdot H}{a\cdot M_{A}}}\]
		
	\end{myenum}
	\newpage
	\begin{note}
		{\GSND[\bfseries\sffamily][\faBell][\maunhan]{Chú ý:}}
		Số liệu ở hàng một và hàng hai ta vẫn áp dụng tương tự cho các đại lượng chỉ lượng chất khác như mol, thể tích
		\[\begin{tikzpicture}
			\tikzset{%
				mynode/.style={%
					ultra thick,
					minimum height=0.65cm,
				},
				mymatrix/.style={%
					matrix of nodes,
					nodes={mynode},
					column 1/.style={%
						align=right
					}
				}
			}
			\matrix (pt)[mymatrix]
			{%
				& aA &+ &bB &[1cm]&cC &+ &dD&\\
				\node[anchor=base east] {Phương trình:};&$a\cdot 22,4$& & &[1cm]& $c \cdot 22,4\cdot H$ & & &(1)\\
				\node[anchor=base east] {Đề cho:};	& $V_{1}$ (lít) & & &[1cm]&$V_{2}$ (lít)& & &(2)\\
			};
			\draw[->,>=stealth,thick] (pt-1-4) --(pt-1-6);
		\end{tikzpicture}\]
	\end{note}
\subsubsection{Một số công thức cần nhớ}
	\begin{center}
		\begin{tikzpicture}[declare function={d=3cm;},node distance=1.75*d and d]
		\tikzstyle{mycircle} = [circle,fill= \mycolor!20,inner sep =2pt, minimum width =3cm, font = \large\color{\maunhan}\bfseries\sffamily,text width =3cm,align=center]
		\tikzstyle{myrec} = [fill= \mauphu!20,inner sep =2pt, minimum width =3cm,minimum height =1.5cm,text width =3cm,align=center,rounded corners=6pt]
		\tikzstyle{myshape} = [fill= \maunhan!10,inner sep =2pt, minimum width =3cm,minimum height =1.5cm,text width =4cm,align=center]
		\node(H) [mycircle]{Hiệu suất \small\color{black}{0<H<1}};
		\node (LBH) [below left of= H,myrec] {Tính theo chất tham gia};
		\node (H1) [below of= LBH,node distance=.55*d,myshape]{$\textbf{H}=\dfrac{\textbf{lượng}_\textbf{\scriptsize phản ứng}}{\textbf{lượng}_\textbf{\scriptsize ban đầu}}$};
		
		\node(DKH) [mycircle]{Hiệu suất \small\color{black}{0<H<1}};
		\node (RBH) [below right of= H,myrec] {Tính theo chất sản phẩm};
		\node (H2) [below of= RBH,node distance=.55*d,myshape]{$\textbf{H}=\dfrac{\textbf{lượng}_\textbf{\scriptsize thực tế}}{\textbf{lượng}_\textbf{\scriptsize lý thuyết}}$};
		\node [below  of= H,node distance=1.78*d] {$\dfrac{\text{\textbf{số nhỏ}}}{\text{\textbf{số lớ}n}}$};
		\draw[->,>=stealth,line width=2pt,line cap=round] (H.south) .. controls ++(-90:1.2cm) and ++(90:1.2cm) .. (LBH.north);
		\draw[->,>=stealth,line width=2pt,line cap=round] (H.south) .. controls ++(-90:1.2cm) and ++(90:1.2cm) .. (RBH.north);
		\node[below of=H,node distance=2.5*d]{\hopcttoan{\textbf{m}_{\textsf{tinh chất}}=\textbf{m}_{\textsf{quặng}} \cdot \textbf{độ tinh khiết}}};
	\end{tikzpicture}\\
	\end{center}
\subsubsection{Một số ví dụ minh họa}
\begin{bt}[2][Hiệu suất phản ứng]
	Người ta dùng quặng boxit để sản xuất $Al$. Hàm lượng $Al_2O_3$ trong quặng là $50\%$. Để có được $2$ tấn nhôm nguyên chất cần bao nhiêu tấn quặng? Biết rằng hiệu suất của quá trình sản xuất là $90 \%$.
	\huongdan{
		\begin{tikzpicture}[declare function={d=1.75cm;},node distance=.75*d and 0.25*d]
			\node (a) {$2Al_2O_3$};
			\node (b)[right of=a,node distance=1.75*d]{4Al};
			\node (plus)[right of=b]{$+$};
			\node (c)[right of=plus]{$3O_2$};
			\draw[->,>=stealth,line width=1.2pt] (a)--(b)node [pos =.5,above,font=\scriptsize]{$t^\circ$} node [pos =.5,below,font=\scriptsize]{crioit};
			\node (aa)[below of=a,node distance=0.5*d]{$2 \cdot 102$};
			\node (bb)[below of=b,node distance=0.5*d]{$4 \cdot 27 \cdot H$};
			\node (bbb)[below of=bb,node distance=0.5*d]{$2$};
			\node (cc)[below of=c,node distance=0.5*d]{\vphantom{x}};
			\node (rcc)[right of=cc]{(tấn)};
			\node (brcc)[below of=rcc,node distance=0.5*d]{(tấn)};
			\node (aaa)[below of=aa,node distance=0.5*d]{$?$};
			\node (laa)[left of=aa,anchor=east]{Phương trình:};
			\node (laaa)[left of=aaa,anchor=east]{Đề bài:};
		\end{tikzpicture}\\
		Áp dụng quy tắc tam suất ta có: $m_{Al_2O_3} = \dfrac{2 \cdot 102\cdot 2 }{4 \cdot 27 \cdot 0.9 } = \dfrac{340}{81 } \mathrm{~\text{tấn}}  $\\
		$\Rightarrow m_{\text{quặng}} =m_{Al_2O_3} \cdot \dfrac{100}{50}=\dfrac{340}{81} \cdot \dfrac{100}{50} \approx 8,395\mathrm{~\text{tấn}} $
	}
\end{bt}

\begin{bt}[2][Hiệu suất phản ứng]
	Để điều chế ra 1 tấn gang với hàm lượng $Fe$ là $95 \%$ người ta cần dùng bao nhiêu tấn quặng manhetit có lẫn $40\%$ tạp chất. Biết hiệu suất phản ứng là $80\%$.
	\huongdan{
	\begin{tikzpicture}[declare function={d=1.5cm;},node distance=.75*d and 0.25*d]
		\node (a) {$Fe_3O_4$};
		\node (plusM)[right of=a]{+};
		\node (b)[right of=plusM]{4CO};
		\node (c)[right of=b,node distance=1.75*d]{3Fe};
		\node (plusH)[right of=c]{+};
		\node (d)[right of=plusH]{$4CO_2$};
		\node (aa)[below of=a,node distance=0.5*d]{232};
		\node (cc)[below of=c,node distance=0.5*d]{$3\cdot56 \cdot H$};
		\node (laa)[left of=aa,anchor=east]{Phương trình:};
		\node (e)[right of=d]{\vphantom{x}};
		\node (ee)[below of=e,node distance=0.5*d]{(tấn)};
		\node (aaa)[below of=aa,node distance=0.5*d]{?};
		\node (db)[left of=aaa,anchor=east]{Đề bài:};
		\node (ccc)[below of=cc,node distance=0.5*d]{0,95};
		\node (eee)[below of=ee,node distance=0.5*d]{(tấn)};
		\draw[->,>=stealth,line width=1.2pt] (b)--(c)node [pos =.5,above]{$t^\circ$};
	\end{tikzpicture}\\
	Ta có khối lượng sắt có trong 1 tấn gang là: $m_{Fe} = m_{\text{gang}}\cdot 0{,}95 =0,95\mathrm{~\text{tấn}}$\\
	Áp dụng quy tắc tam suất ta có: $m_{Fe_3O_4} = \dfrac{232 \cdot 0,95 }{3\cdot 56\cdot 0.8 } = \dfrac{551}{336 } \mathrm{~\text{tấn}}  $\\
	$\Rightarrow m_{\text{quặng}} =m_{Fe_3O_4} \cdot \dfrac{100}{60} \approx 2,733\mathrm{~\text{tấn}} $
	}
\end{bt}


\begin{bt}[2][Tính hiệu suất phản ứng]
	Dùng $100$ tấn quặng $Fe_3O_4$ để luyện gang ($95\% Fe$). Tính khối lượng gang thu được. Cho biết hàm lượng $Fe_3O_4$ trong quặng là $80\%$ và hiệu suất quá trình phản ứng là $93\%$.
	\huongdan{
		\begin{tikzpicture}[declare function={d=1.5cm;},node distance=.75*d and 0.25*d]
		\node (a) {$Fe_3O_4$};
		\node (plusM)[right of=a]{+};
		\node (b)[right of=plusM]{4CO};
		\node (c)[right of=b,node distance=1.75*d]{3Fe};
		\node (plusH)[right of=c]{+};
		\node (d)[right of=plusH]{$4CO_2$};
		\node (aa)[below of=a,node distance=0.5*d]{232};
		\node (cc)[below of=c,node distance=0.5*d]{$3\cdot56 \cdot H$};
		\node (laa)[left of=aa,anchor=east]{Phương trình:};
		\node (e)[right of=d]{\vphantom{x}};
		\node (ee)[below of=e,node distance=0.5*d]{(tấn)};
		\node (aaa)[below of=aa,node distance=0.5*d]{$80$};
		\node (db)[left of=aaa,anchor=east]{Đề bài:};
		\node (ccc)[below of=cc,node distance=0.5*d]{?};
		\node (eee)[below of=ee,node distance=0.5*d]{(tấn)};
		\draw[->,>=stealth,line width=1.2pt] (b)--(c)node [pos =.5,above]{$t^\circ$};
	\end{tikzpicture}\\
	Ta có khối lượng $Fe_3O_4$ có trong 100 tấn quặng là: $m_{Fe_3O_4} = m_{\text{quặng}}\cdot 0{,}8 =100\cdot 0,8 =80\mathrm{~\text{tấn}}$\\
	Áp dụng quy tắc tam suất ta có: $m_{Fe} = \dfrac{80 \cdot 3\cdot 56\cdot0,93 }{232 } \approx 53,876 \mathrm{~\text{tấn}}  $\\
	$\Rightarrow m_{\text{gang}} =m_{Fe} \cdot \dfrac{100}{95} \approx 56,71\mathrm{~\text{tấn}} $
	}
\end{bt}


