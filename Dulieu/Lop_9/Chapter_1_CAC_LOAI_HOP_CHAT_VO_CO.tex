\chapter{Các loại hợp chất vô cơ}
\section{Oxit}
\begin{mtbh}
	\begin{itemize}
		\item Kiến thức
		\begin{itemize}
			\item Nêu được các tính chất hóa học của oxit (oxit axit và oxit bazơ).
			\item Trình bày được khái quát về sự phân loại oxit (4 loại).
			\item So sánh được tính chất hóa học (giống nhau và khác nhau) của oxit axit và oxit bazơ.
			\item Trình bày được tính chất vật lí, tính chất hóa học, ứng dụng và điều chế của $\mathrm{CaO}$ và $\mathrm{SO}_2$.
		\end{itemize}
		\item Kĩ năng
		\begin{itemize}
			\item Viết được phương trình hóa học minh họa cho tính chất hóa học của oxit axit và oxit bazơ. 
			\item Phân biệt được hai oxit $\mathrm{CaO} ; \mathrm{SO}_2$ và một số oxit cụ thể khác.
			\item Vận dụng được tính chất hóa học của oxit và kiến thức về $\mathrm{CaO} ; \mathrm{SO}_2$ để giải một số bài tập định tính (viết phương trình hóa học...) và định lượng (tính phần trăm về khối lượng của oxit trong hỗn hợp hai chất,\ldots).
		\end{itemize}
	\end{itemize}
\end{mtbh}
\subsection{Lí thuyết trọng tâm}
\subsubsection{Định nghĩa}
\begin{hoplythuyet}
	\begin{itemize}
		\item \indam{Oxit} là hợp chất của nguyên tố oxi với một nguyên tố hóa học khác.
		\item Công thức tổng quát: $\mathrm{M}_{\mathrm{x}} \mathrm{O}_{\mathrm{y}}$.
		\item Một số kim loại (oxit bazơ) có màu đặc trưng, người ta có thể dùng nó là chất tạo màu trong tạo men gốm, sứ.(Xem hình \ref{fig:mau_oxit})
	\end{itemize}
\end{hoplythuyet}
\begin{center}
\begin{tikzpicture}[font=\Large,declare function={h=2.4cm;r=5cm;}]
	\tikzset{%
		ntdnode/.style={%
			align=center,
			font=\sffamily\bfseries,
			minimum height =h,
			draw=\mycolor,
		},
		ntdmatrix/.style={%
			matrix of nodes,
			inner sep=3pt,
			nodes in empty cells,
			row sep=-\pgflinewidth,
			column sep=-\pgflinewidth,
			nodes={ntdnode},
			row 1/.style={%
				nodes={%
					align=center,
					font=\bfseries\sffamily,
					inner sep=3pt,
					fill=\mauphu!20!white,
					minimum height =.75cm,
				}
			},
			column 1/.style={%
				nodes={%
					align=center,
					font=\bfseries\sffamily,
					minimum width =.2\textwidth,
				}
			},
			column 2/.style={%
				nodes={%
					align=center,
					font=\bfseries\sffamily,
					minimum width =.3\textwidth,
				}
			},
				column 3/.style={%
				nodes={%
					align=center,
					font=\bfseries\sffamily,
					minimum width =.3\textwidth,
				}
			},
		},
		line/.style={%
			\maunhan!50!red,
			arrows = {-Stealth},
			line width=3pt,
		}
	}
	
	\matrix (Oxit) [ntdmatrix] {%
		& & \\
		& & \\
		& & \\
		& & \\
		& & \\
		& & \\
		& & \\
		& & \\
	};
	\foreach \i/\p in {2/Al2O3,3/Cu2O,4/CuO,5/FeO,6/Fe2O3,7/Fe3O4,8/Cr2O3}{%
	\path(Oxit-\i-2)node[anchor=center]{\includegraphics[height=2cm]{Images/Mau_oxit/\p.png}};
	}
	\foreach \i/\c in {1/Màu sắc,2/Trắng,3/Đỏ gạch,4/đen,5/đen,6/đỏ nâu,7/đỏ,8/Xanh lục}{%
		\path(Oxit-\i-3)node[anchor=center,font=\bfseries\sffamily]{\c};
	}
	\foreach \i/\O in {1/Oxit,2/$Al_{2}O_{3}$,3/$Cu_{2}O$,4/$CuO$,5/$FeO$,6/$Fe_{2}O_{3}$,7/$Fe_{3}O_{4}$,8/$Cr_{2}O_{3}$}{%
	\path(Oxit-\i-1)node[anchor=center,font=\bfseries\sffamily]{\O};
	}
	\path (Oxit-1-2) node[anchor=center,font=\bfseries\sffamily]{Hình ảnh};
\end{tikzpicture}
\captionof{figure}{Màu sắc của một số oxit}
\label{fig:mau_oxit}
\end{center}
\subsubsection{Tính chất hóa học}
\begin{tikzpicture}[font=\Large,declare function={h=1.5cm;r=5cm;}]
	\tikzset{%
		ntdnode/.style={%
			align=center,
			font=\sffamily,
			minimum height =h,
			draw=\mycolor,
		},
		ntdmatrix/.style={%
			matrix of nodes,
			inner sep=3pt,
			nodes in empty cells,
			row sep=-\pgflinewidth,
			column sep=-\pgflinewidth,
			nodes={ntdnode},
			row 1/.style={%
				nodes={%
					align=center,
					font=\LARGE\bfseries\sffamily,
					inner sep=3pt,
					fill=\mauphu!20!white,
					minimum height =1cm,
				}
			},
				row 2/.style={%
				nodes={%
					align=center,
					font=\bfseries\sffamily,
					inner sep=3pt,
					minimum height =1cm,
				}
			},
			column 1/.style={%
				nodes={%
					align=center,
					minimum width =.5\textwidth,
				}
			},
			column 2/.style={%
				nodes={%
					align=center,
					minimum width =.5\textwidth,
				}
			},	
		},
		line/.style={%
			\maunhan!50!red,
			arrows = {-Stealth},
			line width=3pt,
		}
	}
	
	\matrix (tchh) [ntdmatrix] {%
	 Oxit bazo	& Oxit axit \\
		& \\
		Một số oxit bazo $+\mathrm{H}_2 \mathrm{O}$
		$\rightarrow$ Bazơ tương ứng & Hầu hết oxit axit $+\mathrm{H}_2 \mathrm{O}$ $\rightarrow$ Axit tương ứng \\
		\makecell[l]{\GSND[\bfseries\sffamily][\faStar][\maunhan]{Tác dụng với axit}\\ Oxit bazơ + axit $\rightarrow$ Muối + nước} & \makecell[l]{\GSND[\bfseries\sffamily][\faStar][\maunhan]{Tác dụng với bazơ} \\Oxit axit + bazơ tan $\rightarrow$ Muối + nước}\\
	\makecellbox[l]{\GSND[\bfseries\sffamily][\faStar][\maunhan]{Tác dụng với oxit axit}\\ Một số oxit bazơ + Oxit axit $\rightarrow$ Muối}	& \makecellbox[l]{ \GSND[\bfseries\sffamily][\faStar][\maunhan]{Tác dụng với oxit bazơ} \\Oxit axit + Một số oxit bazơ $\rightarrow$ Muối}\\
	};
	\path[fill=\mycolor!20,draw=\mycolor] (tchh-2-1.north west) rectangle (tchh-2-2.south east);
	\path (tchh-2-1.east) node [anchor=center]{\GSND[\bfseries\sffamily][\faStar][\maunhan]{Tác dụng với nước}};
\end{tikzpicture}
\begin{multicols}{2}
	
	\begin{vidu}
	\par	
	\schemestart 
	$\mathrm{Na}_2 \mathrm{O}$
	\+
	$\mathrm{H}_2 \mathrm{O}$
	\arrow{->[][][]}[0,0.65,-{Stealth[length=2mm]}]
	$2\mathrm{NaOH}$
	\schemestop
	\par
	\schemestart 
	$\mathrm{N}_2 \mathrm{O}_5$
	\+
	$\mathrm{H}_2 \mathrm{O}$
	\arrow{->[][][]}[0,0.65,-{Stealth[length=2mm]}]
	$2\mathrm{HNO}_3$
	\schemestop
	\end{vidu}
	\begin{vidu}
		\par
		\schemestart 
		$\mathrm{Cu}\mathrm{O}$
		\+
		$2\mathrm{H}\mathrm{Cl}$
		\arrow{->[][][]}[0,0.65,-{Stealth[length=2mm]}]
		$\mathrm{CuCl}_2$
		\+
		$\mathrm{H}_2\mathrm{O}$
		\schemestop
		\par
		\schemestart 
		$\mathrm{C} \mathrm{O}_2$
		\+
		$\mathrm{Ca} \mathrm{(OH)}_2$
		\arrow{->[][][]}[0,0.65,-{Stealth[length=2mm]}]
		$\mathrm{CaCO}_3$
		\+
		$\mathrm{H}_2\mathrm{O}$
		\schemestop
	\end{vidu}
\end{multicols}
\begin{vidu}
	\par
	\schemestart 
	$\mathrm{Na}_2 \mathrm{O}$
	\+
	$\mathrm{C} \mathrm{O}_2$
	\arrow{->[][][]}[0,0.65,-{Stealth[length=2mm]}]
	$\mathrm{Na}_2\mathrm{CO}_3$
	\schemestop
	\par
	\schemestart 
	$\mathrm{C}\mathrm{O}_{2(r)}$
	\+
	$\mathrm{Ca}\mathrm{O}$
	\arrow{->[][][1.5pt]}[0,0.65,-{Stealth[length=2mm]}]
	$\mathrm{Ca}\mathrm{CO}_3$
	\schemestop
\end{vidu}
\subsubsection{Phân loại oxit theo tính chất hóa học}
	\begin{tikzpicture}[node distance=1cm and 1cm,declare 	function={d=2.25cm;}]
	\tikzstyle{rect} = [rectangle, minimum width=3cm, minimum height=1cm, text centered, draw=\maunhan, fill=\mycolor!30]
	\tikzstyle{cir} = [circle, minimum width=2cm, minimum height=2cm, text centered, draw=\maudam, fill=\mauphu!30,inner sep=1pt,font=\Large\bfseries\sffamily]
	\tikzstyle{arrow} = [draw, -latex,\maudam,line width=2pt]
	\tikzstyle{line} = [draw,\maudam,line width=2pt]
	%%%=====================%%%
	\path (0,0)coordinate (A) node [cir] (OX) {OXIT} ;
	\path (A)--++(-90:2cm) coordinate (Neo) ;
	\path(OX)--(Neo)--([turn]90:d)--([turn]-90:{0.25*d}) node[anchor=north,rect](OXLT){Oxit lưỡng tính};
	\path(OX)--(Neo)--([turn]-90:d)--([turn]90:{0.25*d}) node[anchor=north,rect](OXBZ){Oxit Bazo};
	\path(OX)--(Neo)--([turn]90:{3*d})--([turn]-90:{0.25*d}) node[anchor=north,rect](OXTT){Oxit trung tính};
	\path(OX)--(Neo)--([turn]-90:{3*d})--([turn]90:{0.25*d}) node[anchor=north,rect](OXAX){Oxit Axit};
	\path(OXAX.south)node[anchor=north,text width =.2\textwidth,font=\small](TCOA){\begin{itemize}
			\item Thường là oxit của phi kim.
			\item Tác dụng với dung dịch bazơ tạo thành muối và nước.
	\end{itemize}
		\GSND[\bfseries][][\maunhan]{Ví dụ:} $\mathrm{CO}_2, \mathrm{SO}_3, \mathrm{P}_2 \mathrm{O}_5, \ldots$
	
	};
	\path(OXBZ.south)node[anchor=north,text width =.2\textwidth,font=\small](TCOB){\begin{itemize}
			\item Thường là oxit của kim loại.
			\item Tác dụng với dung dịch axit tạo thành muối và nước.
	\end{itemize}
	\GSND[\bfseries][][\maunhan]{Ví dụ:}
		$\mathrm{Na}_2 \mathrm{O}, \mathrm{FeO}, \mathrm{CuO}, \ldots$
	
	};
	\path(OXLT.south)node[anchor=north,text width =.2\textwidth,font=\small](TCLT){\begin{itemize}
			\item Vừa tác dụng được với dung dịch axit, vừa tác dụng được với dung dịch bazơ tạo thành muối và nước.
	\end{itemize}
	\GSND[\bfseries][][\maunhan]{Ví dụ:}
		$\mathrm{ZnO}, \mathrm{Al}_2 \mathrm{O}_3, \mathrm{Cr}_2 \mathrm{O}_3, \ldots$
	};
	\path(OXTT.south)node[anchor=north,text width =.2\textwidth,font=\small](TCTT){\begin{itemize}
			\item Là oxit không tạo muối.
			\item Không tác dụng với dung dịch axit, dung dịch bazơ, nước.
	\end{itemize}
	\GSND[\bfseries][][\maunhan]{Ví dụ:}	$\mathrm{CO}, \mathrm{NO}, \mathrm{N}_2 \mathrm{O}, \ldots$
	};
	%%%===================%%%
	\draw[arrow](OX)--(Neo)--([turn]90:d)--([turn]-90:{0.25*d});
	\draw[arrow](OX)--(Neo)--([turn]-90:d)--([turn]90:{0.25*d});
	\draw[arrow](OX)--(Neo)--([turn]90:{3*d})--([turn]-90:{0.25*d});
	\draw[arrow](OX)--(Neo)--([turn]-90:{3*d})--([turn]90:{0.25*d});
	\end{tikzpicture}
\subsubsection{Một số oxit quan trọng}

\begin{longtable}{*{2}{|p{.48\textwidth}}|}
	\hline
	\thead{\bfseries\sffamily\large{Canxi oxit (CaO)}} & \thead{\bfseries\sffamily\large{Lưu huỳnh đioxit (SO$_2$)}} \\
	\hline
	\endhead
    \multicolumn{2}{|c|}{\bfseries\sffamily\large{a) Tính chất Vật lý}}\\
    \hline
     Canxi oxit (vôi sống) là chất rắn, màu trắng, nóng chảy ở nhiệt độ rất cao $\left(2585^{\circ} \mathrm{C}\right)$.	& Lưu huỳnh đioxit (khí sunfurơ) là chất khí không màu, mùi hắc, độc (gây ho, viêm đường hô hấp\ldots), nặng hơn không khí.\\
     \hline  
     \multicolumn{2}{|c|}{\bfseries\sffamily\large{b) Tính chất hóa học}}\\
     \hline
       CaO là một oxit bazơ. & $\mathrm{SO}_2$ là một oxit axit.\\
       \hline
       \vspace*{-12pt}
      \GSND[\bfseries][\faStar]{Tác dụng với nước}\par
      \schemestart CaO \+ $\mathrm{H_2O}$ \arrow(.mid east--.mid west){->[][][]}[,.5,,] \chemname{$\mathrm{Ca(OH)_{2}}$}{Vôi tôi,ít tan} \schemestop &\vspace*{-12pt}\GSND[\bfseries][\faStar]{Tác dụng với nước}\par
      \schemestart $\mathrm{SO}_{2}$ \+ $\mathrm{H_2O}$ \arrow(.mid east--.mid west){->[][][]}[,.5,,] \chemname{$\mathrm{H_{2}SO_{3}}$}{axit sunfuro} \schemestop
       \\ 
      \hline\vspace*{-12pt}
      \GSND[\bfseries][\faStar]{Tác dụng với axit tạo thành muối và nước}\par
      Ví dụ:
      \schemestart $\mathrm{CaO}$ \+ $2\mathrm{HCl}$ \arrow{->[][][]}[,.5,,] $\mathrm{CaCl_{2}}$ \+ $H_{2}O$ \schemestop  & \vspace*{-12pt}\GSND[\bfseries][\faStar] {Tác dụng với bazơ tạo thành muối và nước}\par
      Ví dụ:
      \schemestart $\mathrm{SO}_2$\+$\mathrm{Ca}(\mathrm{OH})_2$  \arrow{->[][][]}[,.5,,] $\mathrm{CaSO}_3$ $\downarrow$\+$\mathrm{H}_2 \mathrm{O}$ \schemestop  \\
      \hline
       {\bfseries\sffamily{\faStar\:Tác dụng với oxit axit tạo thành muối}}\par
       Ví dụ: $\mathrm{CaO}+\mathrm{CO}_2 \rightarrow \mathrm{CaCO}_3$  & {\bfseries\sffamily{\faStar\;Tác dụng với oxit bazo (tan) tạo thành muối}}\par Ví dụ: $\mathrm{SO}_2+\mathrm{Na}_2 \mathrm{O} \rightarrow \mathrm{Na}_2 \mathrm{SO}_3$ \\
       \hline
       \multicolumn{2}{|c|}{\bfseries\sffamily\large{c) Ứng dụng}}\\
       \hline
       \begin{itemize}
       	\item Phần lớn $\mathrm{CaO}$ được dùng trong công nghiệp luyện kim và làm nguyên liệu cho công nghiệp hóa học.
       \item Dùng để khử chua đất trồng trọt, xử lí nước thải công nghiệp, sát trùng, diệt nấm, khử độc môi trường,...
       	\item Dùng làm khô nhiều chất.
       \end{itemize} &\begin{itemize}
       \item  Phần lớn $\mathrm{SO}_2$ dùng để sản xuất axit sunfuric $\mathrm{H}_2 \mathrm{SO}_4$. 
       \item Dùng làm chất tẩy trắng bột gỗ trong sản xuất giấy, đường,\ldots
       \item Dùng làm chất diệt nấm mốc,\ldots
       \end{itemize}\\
       \hline
       \multicolumn{2}{|c|}{\bfseries\sffamily\large{d) Điều chế}}\\
       \hline
         & {\bfseries\sffamily{\faStar\;Trong phòng thí nghiệm}}\par
         - Cho muối sunfit tác dụng với axit mạnh như $\mathrm{HCl}$, $\mathrm{H}_2 \mathrm{SO}_4, \ldots$\\
         \hline
          {\bfseries\sffamily{\faStar\;Trong công nghiệp}}\par
         {\bfseries\sffamily {\faArrowAltCircleRight\; Nguyên liệu:}}\par
         + Đá vôi (chứa $\mathrm{CaCO}_3$ ).\par
         + Chất đốt là than đá, củi, dầu, khí tự nhiên,\ldots\par
         {\bfseries\sffamily\faArrowAltCircleRight \;Các phản ứng hóa học xảy ra khi nung vôi:}
         + Than cháy sinh ra khí $\mathrm{CO}_2$ và tỏa nhiều nhiệt:
         	$$\mathrm{C}+\mathrm{O}_2 \stackrel{\mathrm{t}^{\circ}}{\longrightarrow} \mathrm{CO}_2$$
         	+ Nhiệt sinh ra phân hủy đá vôi ở khoảng trên $900^{\circ} \mathrm{C}$ :
         	$$
         	\mathrm{CaCO}_3 \stackrel{\mathrm{t}^{\circ}}{\longrightarrow} \mathrm{CaO}+\mathrm{CO}_2
         	$$ & {\bfseries\sffamily \faStar\;Trong công nghiệp:}\par
         	Đốt lưu huỳnh hoặc pirit sắt $\left(\mathrm{FeS}_2\right)$ bằng oxi trong không khí:\par
         	\schemestart $\mathrm{S}$\+$\mathrm{O}_{2}$  \arrow{->[$t^o$][][]}[,.5,,] $\mathrm{SO}_3$ $\downarrow$\+$\mathrm{H}_2 \mathrm{O}$ \schemestop\par
         	\schemestart 4$\mathrm{FeS_{2}}$\+11$\mathrm{O}_{2}$  \arrow{->[$t^o$][][]}[,.5,,] 2$\mathrm{Fe_{2}O_{3}}$ \+ 8$\mathrm{SO}_2$ \schemestop
         	\\
         \hline
\end{longtable}