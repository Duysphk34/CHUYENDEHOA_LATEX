\titlespacing*{\subsubsection}{0cm}{0pt}{0pt}
\chapter{Rút gọn căn thức}
\section{So sánh  và tìm điều kiện xác định }
\begin{MuctieuH}
	\begin{itemize}
		\item Hiểu và vận dụng quy tắc cơ bản so sánh hai căn bậc hai.
		\item Nắm vững các phương pháp so sánh hai căn bậc hai.
		\item Nắm được kiến thức cơ bản về điều kiện xác định của một số biểu thức
	\end{itemize}
\end{MuctieuH}
%%%===============Kiến thức cần nhớ================%%%%
\phan[\maunhan]{Kiến thức cần nhớ}
%%%
	\vspace{-0.25cm}\begin{mylt}
		%%%
		\Noibat[][\qagfont]{So sánh hai căn thức}
		
		Nếu $\mathrm{a}, \mathrm{b} \geq 0$ thì $a<b \Leftrightarrow \sqrt{a} <\sqrt{b} $
		%%%
		\Noibat[][\qagfont]{Điều kiện có nghĩa của một số biểu thức}
		
		Xét các đa thức $A(x)$ và $B(x)$ thì:
		\begin{enumEX}{2}
			\item $A(x)$ là đa thức thì $A(x)$ luôn có nghĩa.
			\item $\dfrac{A(x)}{B(x)}$ có nghĩa khi $B(x) \neq 0$.
			\item $\sqrt{A(x)}$ có nghĩa khi $A(x) \geq 0$.
			\item $\dfrac{\sqrt{A(x)}}{\sqrt{B(x)}}$ có nghĩa khi $B(x) > 0$.
		\end{enumEX}
	\end{mylt}
	%%%===============Các dạng toán================%%%%
	\vspace{0.25cm}\phan[\mauphu]{Các dạng toán}
	\begin{dang}{So sánh số chứa căn bậc hai }
		\subsubsection{Phương pháp giải}
		\begin{itemize}
			\item \indam{Cách 1:} Vận dụng đưa thừa số ra ngoài dấu căn để đưa hai số không âm A và B về một trong ba trường hợp sau:
			\begin{itemize}
				\item $A=\sqrt{a}$ và $B=\sqrt{b}$, khi đó: $a<b \Rightarrow \sqrt{a} <\sqrt{b} \Rightarrow A<B$.
				\item $A=m\sqrt{a}$ và $B=n\sqrt{a}$, khi đó: $m<n \Rightarrow A<B$.
				\item $A=m+k\sqrt{a}$ và $B=n+k\sqrt{a}$ $(a\geq0)$, khi đó: $m<n \Rightarrow A<B$.
			\end{itemize}
			\item \indam{Cách 2:} Nếu hai số $A$ và $B$ cùng dương và $A^2>B^2$ thì $A>B$
			\item \indam{Cách 3:} So sánh mỗi A và B với một số trung gian.
			Nếu $A<C$ và $C<B$ thì $A<B$\\
			Số trung gian có thể là số chứa căn bậc hai hoặc không chứa căn bậc hai.
		\end{itemize}
	\Noibat[][\qagfont][\faBell][0.5]{Công thức nâng cao:}\\
		   \begin{itemize} 
			\item \hopcttoan{\sqrt{n+a} +\sqrt{n-a}\leq 2 \cdot \sqrt{n}} với $0 \leq a \leq n$.\\
			\item \hopcttoan{\dfrac{1}{\sqrt{k+1}}<2\sqrt{k+1}-2\sqrt{k}} với $k \geq 0$.
			\\\\
           \end{itemize}
	\subsubsection{Ví dụ minh họa}
	\vspace{0.5cm}
	   \hienthiloigiaivd
	    %%%===========Ví dụ 1===========================%%%
		\begin{vd}[So sánh biểu thức số chứa căn bậc hai][1]
			So sánh các số sau:
			\begin{enumEX}{2}
				\item $2\sqrt{38}$ và $\sqrt{151}$.
				\item $-7\sqrt{11}$ và $-11\sqrt{7}$.\\
			\end{enumEX}
			\loigiai{%
				\begin{enumerate}
					\item Ta có $2\sqrt{38}=\sqrt{2^2\cdot38}=\sqrt{152}>\sqrt{151}$. Vậy $2\sqrt{38}>\sqrt{151}$
					\item $11\sqrt{7}=\sqrt{11^2 \cdot 7}=\sqrt{847}$; $7\sqrt{11}=\sqrt{7^2 \cdot 11}= \sqrt{539}$.\\
					Do $539<847$ $\Rightarrow$ $7\sqrt{11}<11\sqrt{7}$ $\Rightarrow$ $-7\sqrt{11}>-11\sqrt{7}$
				\end{enumerate}
				}
		\end{vd}
		%%%===========Ví dụ 2===========================%%%
		\begin{vd}[So sánh biểu thức số chứa căn bậc hai][2]
			So sánh các số sau:
			\begin{enumEX}{2}
				\item $9+\sqrt{17}$ và $13$.
				\item $25-\sqrt{63}$ và $17$.
			\end{enumEX}
			\loigiai{%
				\begin{enumerate}
					\item Ta có $13=9+4 =9+\sqrt{16}$.
					\\
					Vì $\sqrt{17}>\sqrt{16}$ nên $\Rightarrow 9+\sqrt{17} >9+\sqrt{16} $. Vậy $9+\sqrt{17}>13$.
					\item Xét $17 =25-8 =25-\sqrt{64}$.
					\\
					Vì $-\sqrt{63}>-\sqrt{64}$ nên $\Rightarrow 25-\sqrt{63}>25-\sqrt{64}$. Vậy $25-\sqrt{63}>17$ 
				\end{enumerate}
			}
		\end{vd}
		%%%=========================Ví dụ 3===========================%%%
		\begin{vd}[So sánh biểu thức số chứa căn bậc hai][3]
			So sánh các số sau:
			\begin{enumEX}{2}
				\item $\sqrt{13}+\sqrt{5}$ và $\sqrt{11}+\sqrt{7}$.
				\item $\sqrt{25}+\sqrt{3}$ và $\sqrt{5}+\sqrt{15}$.
			\end{enumEX}
			\loigiai{%
				\begin{enumerate}
					\item $\left(\sqrt{13}+\sqrt{5}\right)^2=18+2\sqrt{13} \cdot \sqrt{5}=18+2\sqrt{45}$
					\\
					$\left(\sqrt{11}+\sqrt{7}\right)^2=18+2\sqrt{11} \cdot \sqrt{7}=18+2\sqrt{77}$
					\\
					Vì $18+2\sqrt{45}<18+2\sqrt{77}$ nên $\Rightarrow$ $\left(\sqrt{13}+\sqrt{5}\right)^2 < \left(\sqrt{11}+\sqrt{7}\right)^2$
					$\Rightarrow$ $\sqrt{13}+\sqrt{5} < \sqrt{11}+\sqrt{7} $
					\item $\left(\sqrt{25}+\sqrt{3}\right)^2 = 28 + 2\sqrt{25}\cdot\sqrt{3} =28 +2\sqrt{75}$.
					\\
					$\left(\sqrt{5}+\sqrt{15}\right)^2 = 20 + 2\sqrt{5}\cdot\sqrt{15} =20 +2\sqrt{75}$.
					\\
					Vì $28 +2\sqrt{75} > 20 +2\sqrt{75}$  nên  $\left(\sqrt{25}+\sqrt{3}\right)^2 >\left(\sqrt{5}+\sqrt{15}\right)^2$ $\Rightarrow$ $\sqrt{25}+\sqrt{3} > \sqrt{5}+\sqrt{15}$.
				\end{enumerate}
			}
		\end{vd}
		%%%==========================Ví dụ 4===========================%%%
		\begin{vd}[So sánh biểu thức số chứa căn bậc hai][4]
			So sánh các số sau:
			\begin{enumEX}{2}
				\item $\dfrac{37-2\sqrt{35}}{5}$ và $6-\sqrt{2}$.
				\item $\sqrt{3}-\sqrt{2}$ và $\sqrt{6}-\sqrt{5}$.
			\end{enumEX}
			\loigiai{%
			\begin{enumerate}
				\item Ta có $\dfrac{37-2\sqrt{35}}{5}>\dfrac{37-2\sqrt{36}}{5}=5$ (1).
				Mặt khác lại có $6-\sqrt{2}<6-\sqrt{1} =5$ (2).
				\\
				Kết hợp (1) và (2) ta được $\dfrac{37-2\sqrt{35}}{5}>6-\sqrt{2}$
				\item Ta có $\left(\sqrt{3}+\sqrt{2}\right) \cdot \left(\sqrt{3}-\sqrt{2}\right)=1$ $\Rightarrow$ $\sqrt{3}-\sqrt{2}=\dfrac{1}{\sqrt{3}+\sqrt{2}}$
				\\
				Tương tự $\sqrt{6}-\sqrt{5}=\dfrac{1}{\sqrt{6}+\sqrt{5}}$
				\\
				Vì $\sqrt{3}+\sqrt{2}<\sqrt{6}+\sqrt{5}$ nên $\Rightarrow$ $\dfrac{1}{\sqrt{3}+\sqrt{2}} > \dfrac{1}{\sqrt{6}+\sqrt{5}} $
				\\
				hay $\sqrt{3}-\sqrt{2}>\sqrt{6}-\sqrt{5}$.
			\end{enumerate}
			}
		\end{vd}
	\end{dang}
\phan{Bài tập vận dụng}
