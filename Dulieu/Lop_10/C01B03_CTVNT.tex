\section{Cấu trúc lớp vỏ electron}
\begin{Muctieu}
	\begin{itemize}
		\item  Trình bày và so sánh được nô hình của Rutherford - Bohr với mô hình hiện đại mô tả sự chuyển động của electron trong nguyên tử
		\item  Nêu được klái niệm về orbital nguyên tử $(\mathrm{AO})$, mô tả được hìnl dạng của $\mathrm{AO}(\mathrm{s}$, số lượng electron trong 1 AO .
		\item  Trìh bày được khái niệm lớp, phân lớp electron và mối quan hệ về số lượng phân trong một lớp. Liên hệ được về số lượng AO trong một phân lớp, trong một lớp.
		\item  Viết được cấu lình electron nguyên tử theo lớp, phân lớp electron và theo ô orbital biết số liệu nguyên tử $Z$ của 20 nguyên tố đầu tiên trong bảng tuần hoàn.
		\item  Dựa vào đặc điểm cấu hình electron lớp ngoài cùng của nguyên tử, dự đoán được chất hoá học cơ bản (kim loại hay pli kim) của nguyên tố tương ưng.
	\end{itemize}
\end{Muctieu}
\subsection{Nội dung bài học}
 \subsubsection{Sự chuyển động của electron trong nguyên tử}
\Noibat[][][\faArrowCircleORight]{Các mô hình nguyên tử}
\begin{figure}[!htp]
	\begin{center}
		\subcaptionbox{Mô hình nguyên tử Borh-Rutherford\label{atomic_borh}}[0.4\textwidth]
		{\begin{tikzpicture}[declare function={R=2;r=0.3*R;}]
				\tikzset{%
					pics/.cd,
					hinhcau/.style args={#1}{%
						code={\path[pic actions] circle (#1 pt);}
					},
				}
				%%%Khai bao gốc tạo độ
				\path (0,0) coordinate (O);
				%%%Hạt nhân
				\foreach \g/\c/\d in {0/\maunhan/0,40/\mauphu/4,80/\mauphu/8,120/\maunhan/8,160/\mauphu/7,220/\maunhan/7,280/\mauphu/7,340/\maunhan/8,40/\maunhan/8}{
					\path ($(O)+(\g:\d pt)$) pic[ball color =\c] {hinhcau={4.5}};
				}
				%%% Quỹ đạo e
				\foreach \goc/\p in {30/30,90/50,150/90}{
					\path[left color=\mycolor,right color =\maunhan,line width=2pt,even odd rule,rotate around={\goc:(O)}] 
					(O) ellipse ({R} and {r}) (O) ellipse ({0.98*R} and {0.96*r})
					;
					\path ($(O)+(180:{0.99*R})$) arc (180:\p:{0.99*R} and {0.98*r}) coordinate (G);
					\path (G) pic[ball color =\mauphu,rotate around={\goc:(O)}] {hinhcau={3}};
				}
		\end{tikzpicture}}
		\subcaptionbox{Mô hình nguyên tử hiện đại\label{atomic_modern}}[0.4\textwidth]
		{\begin{tikzpicture}[declare function={R=2;r=0.3*R;}]
				\tikzset{%
					pics/.cd,
					hinhcau/.style args={#1}{%
						code={\path[pic actions] circle (#1 pt);}
					},
				}
				%%%Khai bao gốc tạo độ
				\path (0,0) coordinate (O);
				%%%vong tron trong
				\path[fill opacity=0.52,fill=\mauphu!90!black,path fading=fade in one](O) circle (2.2cm);
				%%%Dam may e
				\foreach \i in {1,...,2000}
				{
					\pgfmathsetmacro{\angle}{random()*360}
					\pgfmathsetmacro{\u}{random()}
					\pgfmathsetmacro{\radius}{2.18 * pow(\u, 2)}  
					\pgfmathsetmacro{\x}{\radius*cos(\angle)}
					\pgfmathsetmacro{\y}{\radius*sin(\angle)}
					\path (\x,\y) pic[fill=\mauphu,path fading=fade in three] {hinhcau={0.65}};
				}
				%%%Hạt nhân
				\path[fill=\maudam!70!white,path fading=fade in four](O) circle (0.48cm);
				\foreach \g/\c/\d in {0/\maunhan/0,40/\mauphu/4,80/\mauphu/8,120/\maunhan/8,160/\mauphu/7,220/\maunhan/7,280/\mauphu/7,340/\maunhan/8,40/\maunhan/8}{
					\path ($(O)+(\g:\d pt)$) pic[ball color =\c] {hinhcau={4.5}};
				}
				%%%vong tron ngoài
				\path[fill opacity=0.50,fill=\mauphu,path fading=fade in one](O) circle (2.2cm);
		\end{tikzpicture}}
		\caption{Hai Mô hình nguyên tử }\label{fig:atomic_model}
	\end{center}
\end{figure}
\begin{hoivadap}
	\begin{cauhoi}
		Quan sát hình \ref{atomic_borh} và hình \ref{atomic_modern} so sánh điểm giống và khác nhau giữa mô hình Rutherford - Bohr với mô hình hiện đại mô tả sự chuyển động của electron trong nguyên tử.
	\end{cauhoi}
	\loigiai{
		\begin{longtable}{|C{0.25\textwidth}|p{0.33\textwidth}|p{0.33\textwidth}|}
			\caption{So sánh hai mô hình nguyên tử}\label{tab:ssmhnt}\\
			\hline\rowcolor{\maunhan!20}
			\makecell[c]{\qagfont{Tiêu chí}} & \makecell[c]{\qagfont{Mô hình Rutherford-Bohr}} & \makecell[c]{\qagfont{Mô hình hiện đại}} \\
			\hline
			\endfirsthead
			\multicolumn{3}{c}%
			{{\bfseries \tablename\ \thetable{} -- tiếp theo}} \\
			\hline\rowcolor{\maunhan!20}
			\makecell[c]{\qagfont{Tiêu chí}} & \makecell[c]{\qagfont{Mô hình Rutherford-Bohr}} & \makecell[c]{\qagfont{Mô hình hiện đại}} \\
			\hline
			\endhead
			\hline \multicolumn{3}{|r|}{{Tiếp theo ở trang sau}} \\ \hline
			\endfoot
			\hline
			\endlastfoot
			\multirow[c]{2.5}{*}{\makecell[c]{\textcolor{\maunhan}{\textbf{Cấu trúc nguyên tử}}}} & Hạt nhân ở trung tâm, electron chuyển động quanh hạt nhân theo các quỹ đạo tròn & Hạt nhân ở trung tâm, electron chuyển động trong các obitan (đám mây electron) \\
			\hline
			\multirow[c]{2.0}{*}{\makecell[c]{\textcolor{\maunhan}{\textbf{Quỹ đạo electron}}}} & Các quỹ đạo tròn cố định với bán kính xác định & Obitan - vùng không gian có xác suất tìm thấy electron cao \\
			\hline
			\multirow[c]{2.5}{*}{\makecell[c]{\textcolor{\maunhan}{\textbf{Năng lượng electron}}}} & Electron chỉ tồn tại ở các mức năng lượng xác định (trạng thái dừng) & Electron có thể tồn tại ở nhiều mức năng lượng khác nhau trong một obitan \\
			\hline
			\multirow[c]{2.5}{*}{\makecell[c]{\textcolor{\maunhan}{\textbf{Chuyển động}}\\\textcolor{\maunhan}{\textbf{của electron}}}} & Chuyển động tròn quanh hạt nhân & Chuyển động phức tạp, không thể xác định chính xác vị trí và vận tốc cùng lúc \\
			\hline
			\textcolor{\maunhan}{\textbf{Nguyên lý xác định vị tríelectron}} & Có thể xác định chính xác vị trí và vận tốc của electron & Tuân theo nguyên lý bất định Heisenberg \\
			\hline
			\textcolor{\maunhan}{\textbf{Sự mô tả electron}} & Hạt & Lưỡng tính sóng-hạt \\
			\hline
			\textcolor{\maunhan}{\textbf{Số lượng electron tối đa trên một lớp}} & 2n\textsuperscript{2} (n là số lớp) & Tuân theo nguyên lý Pauli và quy tắc Hund \\
			\hline
			\multirow[c]{2}{*}{\makecell[c]{\textcolor{\maunhan}{\textbf{Hình dạng obitan}}}} & Không đề cập & Có nhiều hình dạng khác nhau (s, p, d, f) \\
			\hline
			\multirow[c]{2}{*}{\makecell[c]{\textcolor{\maunhan}{\textbf{Spin của electron}}}} & Không đề cập & Được xem xét và ảnh hưởng đến cấu hình electron \\
			\hline
			\multirow[c]{2}{*}{\makecell[c]{\textcolor{\maunhan}{\textbf{Giải thích phổ}}\\ \textcolor{\maunhan}{\textbf{nguyên tử}}}} & Giải thích được phổ của nguyên tử hydro & Giải thích được phổ của tất cả các nguyên tử \\
		\end{longtable}
	}
\end{hoivadap}
\Noibat[][][\faArrowCircleORight]{Tìm hiểu về orbital nguyên tử}
\Noibat{Khái niệm}
\vspace{0.5cm}
\begin{tomtat}
	Orbital nguyên tử (ki hiệu là AO) là khu vực không gian xung quanh hạt nhân nguyên tử mà xác suất tìm thấy electron trong khu vực đó là lớn nhất (khoảng $90 \%$ ).
\end{tomtat}
\begin{hopvidu}[\maunhan]
	\begin{itemize}
		\item Các AO thường gặp là s, p, d, f.
		\item Orbital nguyên tử có một số hình dạng khác nhau. Ví dụ: AO hình cầu, còn gọi là $\mathrm{AO} \mathrm{s} ; \mathrm{AO}$ hình số tám nổi, còn gọi là AO p (tùy theo vị tri của AO p trên hệ trục toạ độ Descartes (Đề-các), sẽ gọi là $\mathrm{AO} \mathrm{p}_{\mathrm{x}}, \mathrm{P}_{\mathrm{y}}$ và $\mathrm{p}_z$ )(xem hình \ref{fig:hinhdangAO}).
	\end{itemize}
\end{hopvidu}
\begin{hopvidu}[\mycolor]
	\begin{center}
		\begin{tikzpicture}[declare function={d=1cm;r=.55*d;h=.125*d;R=.36*d;k=0.65*d;}]
			\tikzstyle{linestyle} = [line width=1pt,\maunhan!80]
			\tikzstyle{myshapestyle} = [line width=1pt,opacity=.90,ball color =\mauphu!90]
			\tikzset{
				pics/.cd,
				AOs/.style args={#1/#2}{code={%
						\if\relax\detokenize{#1}\relax
						\def\ballcolor{red}
						\else
						\def\ballcolor{#1}
						\fi,
						\if\relax\detokenize{#2}\relax
						\def\opacity{0.8}
						\else
						\def\opacity{#2}
						\fi
						\draw[linestyle] ([xshift=-1.8*R]0*d,0)--([xshift=1.8*R]0*d,0);
						\fill[myshapestyle,ball color = \ballcolor,opacity=\opacity] (0*d,0) circle (R);
				}},
				AOp/.style args={#1/#2}{code={%
						\draw[linestyle,pic actions] (0,{-1.5*d - h})--(0,{1.5*d + h}) node [pos=#2,above,font=\sffamily\bfseries] {#1};
						\path[myshapestyle,pic actions] (0,0)..controls +(0:{.25*r}) and +(0:r)..(0,d)--
						(0,d)..controls +(180:r) and +(180:{.25*r})..(0,0);
						\path[myshapestyle] (0,-d)..controls +(180:r) and +(180:{.25*r})..(0,0)--
						(0,0)..controls +(0:{.25*r}) and +(0:r)..(0,-d);
				}}
			}
			\path (0*k,0) coordinate (A)
			(4*k,0) coordinate (B)
			(9*k,0) coordinate (C)
			(13*k,0) coordinate (D)
			;
			\path (A) pic [local bounding box=AOsa] {AOs={red}/{}};
			\path (B) pic[local bounding box=AOPx,rotate around={-45:(B)},<-,>=stealth]  {AOp={x}/{0}};
			\path (C) pic [local bounding box=AOPy,rotate around={-90:(C)},-latex] at (C) {AOp={y}/{1}};
			\path (D)  pic [local bounding box=AOPz,-latex] at (D) {AOp={z}/{1}};
			\foreach \p/\n in {
				A/AOs,B/AOpx,C/AOpy,D/AOpz
			}{
				\path ($(\p)+ (0,-2)$) node [inner sep =0pt, outer sep =0pt,font=\bfseries\sffamily] {\n};
			}
		\end{tikzpicture}
	\end{center}
	\captionof{figure}{Hình dạng các AO s và AOp\label{fig:hinhdangAO}}
\end{hopvidu}
\Noibat{Ô orbital}
\vspace{0.5cm}
\begin{tomtat}
	\begin{itemize}
		\item Một AO được biểu diễn bằng một ô vuông, gọi là ô orbital \raisebox{-3pt}{\squarerow[][0.5][\mycolor]{1}}
		\item Trong 1 orbital chỉ chứa tối đa 2 electron có chiều tự quay ngược nhau (nguyên lí loại trừ Pauli (Pau-li)). Nếu orbital có 1 electron thì biểu diễn bằng 1 mũi tên đi lên (\raisebox{-3pt}{\squarerow[1u][0.5][\mycolor]{1}}), nếu orbital có 2 electron thì được biểu diễn bằng 2 mũi tên ngược chiều nhau, mũi tên đi lên viết trước (\raisebox{-3pt}{\squarerow[2ud][0.5][\mycolor]{1}}).
	\end{itemize}
\end{tomtat}
\subsubsection{Lớp và phân lớp electron}
\Noibat[][][]{Lớp electron}\\
\begin{tomtat}
	\begin{enumerate}
		\item Thứ tự các lớp electron được ghi bằng các số nguyên $n=1, 2, 3, ... , 7$.
		\item Các electron thuộc cùng một lớp có mức năng lương gần bằng nhau
	\end{enumerate}
\end{tomtat}
\vspace{0.5cm}
\begin{center}
	\captionof{table}{\textbf{Tên gọi các lớp từ 1 đến 7}\label{lopelectron}}
	\begin{tabular}{C{0.1\linewidth}*{7}{C{0.1\linewidth}}}
		\textbf{n}&1&2&3&4&5&6&7\\
		\textbf{Tên lớp} &K&L&M&N&O&P&Q
	\end{tabular}
\end{center}
\Noibat[][][]{Phân lớp electron}\\
\begin{tomtat}
	\begin{enumerate}
		\item  Mỗi lớp electron phân chia thành các phân lớp được kí hiệu bằng các chữ cái viết thường: s, p, d, f.
		\item  Các electron trên cùng một phân lớp có mức năng lượng bằng nhau.
		\item  Số phân lớp trong mỗi lớp bằng số thứ tự của lớp $(n \leq 4)$ :
	\end{enumerate}
\end{tomtat}
\begin{vidu}
	\begin{itemize}
		\item Lớp thứ nhất (lớp K) có 1 phân lớp , đó là phân lớp 1s
		\item Lớp thứ hai (lớp L) có 2 phân lớp , đó là phân lớp 2s và 2p
		\item Lớp thứ ba (lớp M) có 3 phân lớp , đó là các phân lớp 3s, 3p và 3d
		\item Lớp thứ 4 (lớp N) có 4 phân lớp , đó là các phân lớp 4s, 4p, 4d, 4f
	\end{itemize}
\end{vidu}
\Noibat{Số lượng Orbital trong một lớp, phân lớp}
\Noibat[][][\faAngleRight][][0.25]{Số lượng Orbital trong một phân lớp}
\begin{tomtat}
	Trong một phân lớp, các orbital có cùng mức năng lượng.
	\begin{itemize}
		\item Phân lớp s có 1 AO s
		\item Phân lớp p có 3 AO px, py, pz
		\item Phân lớp d có 5 AO 
		\item Phân lớp f có 7 AO 
	\end{itemize}
\end{tomtat}
\Noibat[][][\faAngleRight][][0.25]{Số lượng Orbital trong một lớp}
\begin{tomtat}
	Số obitan trong lớp electron thứ n là $n^2$ orbital
	\begin{itemize}
		\item Lớp K (n=1) có $1^2 =1$ AO đó là AO 1s.
		\item Lớp L (n=2) có $2^2 =4$ AO đó là 1 AO 2s và 3 AO 2p.
		\item Lớp M (n=3) có $3^2 =9$ AO đó là 1 AO 3s, 3 AO 3p và  5 AO 3d.
		\item Lớp N (n=4) có $4^2 =16$ AO đó là 1 AO 4s, 3 AO 4p, 5AO 4d và 7 AO 4f.
	\end{itemize}
\end{tomtat}
\begin{hoivadap}
	\begin{cauhoi}
		Hãy cho biết số electron  tối đa có trong một phân lớp , một lớp?
	\end{cauhoi}
	\loigiai{
		Số AO có trong các phân lớp s,p,d,f  tương ứng là 1, 3, 5, 7 và mỗi AO chứa tối đa 2 electron do đó
		\begin{itemize}[wide=0.65cm]
			\item Phân lớp s chứa tối đa $2\cdot1 =2$ electron
			\item Phân lớp p có chứ tối đa $2\cdot3 =6$  electron
			\item Phân lớp d chứa tối đa $2\cdot5 =10$  electron
			\item Phân lớp f có tối đa $2\cdot7 =14$  electron
		\end{itemize}
		Lớp n có $n^2$ AO do đó số electron tối đa có trong lớp electron thứ n là $2n^2$
	}
\end{hoivadap}
\subsubsection{Cấu hình electron của nguyên tử}
\Noibat{Năng lượng của electron trong nguyên tử}
\Noibat[][][\faArrowCircleORight]{Trật tự mức năng lượng AO}\\
\begin{tomtat}
	\begin{itemize}
		\item Khi số hiệu nguyên tử $Z$ tăng, các mức năng lượng $AO$ tăng dần theo trình tự sau:
	\end{itemize}
	\centering\boxct[\maunhan][3pt][\color{\maunhan}\qagfont]{1s 2s 2p 3s 3p 4s 3d 4p 5s 4d 5s 4d 5p 6s 4f 5d 6p 4f 5d 6p 7s 5f 6d \ldots}
\end{tomtat}
\Noibat[][][\faArrowCircleORight]{Nguyên lý và quy tắc phân bố electron trong nguyên tử}
\begin{ghinho}
	\indam{Nguyên lý Pau-ly.}Trên một obitan chỉ có thể nhiều nhất là hai electron và hai electron này chuyển động tự quay khác chiều nhau xung quanh trục riêng của mỗi electron.
\end{ghinho}

Như vậy theo nguyên lý Pau-li thì:
\begin{itemize}[wide=0.65cm]
	\item lớp n có tối đa $2n^2$ electron
	\item Số electron tối đa trên các phân lớp s, p, d, f lần lượt là $2$, $6$, $10$, $14$ electron
\end{itemize}

\begin{hopvidu}[\maunhan]
	\begin{itemize}[wide=0.65cm,leftmargin=0.65cm]
		\item Để biểu thị phân lớp 1s có 2 electron  ta dùng kí hiệu $1s^2$.Trong đó:số 1 chỉ lớp n=1. Chữ s chỉ orbital s. Số 2 ở phía trên bên phải số electron có trong AO s.
		\item Phân lớp: $s^2$,$p^6$, $d^{10}$, $f^{14}$ có đủ electron tối đa gọi là \indam{phân lớp bão hòa}.
		\item Phân lớp $s^1$,$p^3$, $d^5$, $f^7$ có nửa số electron tối đa gọi là \indam{phân lớp bán bão hòa}.
		\item Phân lớp chưa đủ số electron tối đa gọi là \indam{phân lớp chưa bão hòa}.Ví dụ $s^1$, $p^3$, $p^5$, $d^9$, $f^{11}$.
	\end{itemize}
\end{hopvidu}

\begin{note}
	Người ta biểu thị chiều tự quay khác nhau quanh trục riêng của hai electron bằng 2 mũi tên nhỏ:Một mũi tên có chiều đi lên  và một mũi tên có chiều đi xuống . 
	\begin{itemize}[wide=0.65cm,leftmargin=0.65cm]
		\item Trong 1 orbital đã có 2 electron, thì 2 electron này gọi là \indam{electron ghép đôi} .Khi biểu diễn mũi tên bên trái phải vẽ hướng lên và mũi tên bên phải vẽ hướng xuống (\squarerow[2ud][0.5][\maunhan][-4pt]{1})
		\item Trong một Orbital chỉ có 1 electron, thì electron này gọi là  \indam{electron độc thân}. Khi biểu diễn mũi tên bắt buộc phải vẽ chiều hướng lên (\squarerow[1u][0.5][\maunhan][-4pt]{1})
	\end{itemize}
\end{note}
\begin{ghinho}
	\indam{Nguyên lý vững bền.} Các electron trong nguyên tử ở trạng thái cơ bản lần lượt chiếm các mức năng lượng từ thấp đến cao.
\end{ghinho}
\begin{vidu}
	Nguyên tử helium ($Z=2$) có 2 electron. Theo nguyên lý pau - li hai electron này cùng chiếm orbital 1s có mức năng lượng thấp nhất. Do đó sự phân bố electron trên orbital của He là \[1s^2 \xrightarrow \squarerow[2ud][0.5][\maunhan][-4pt]{1}\]
	Nguyên tử Boron ($Z=5$) có 5 electron. 2 electron đầu tiên chiếm AO 1s có năng lượng thấp nhất, 2 eletcron tiếp theo chiếm AO 2s và electron còn lại chiếm AO 2p . Do đó sự phân bố electron trên orbital của B là \[1s^22s^22p^1 \xrightarrow \underset{1s^2}{\squarerow[2ud][0.5][\maunhan][-4pt]{1}}\; \underset{2s^2}{\squarerow[2ud][0.5][\maunhan][-2pt]{1}}\hspace{-0.8pt} \underset{2p^1}{\squarerow[1u][0.5][\maunhan][0pt]{3}}\].
\end{vidu}
\begin{ghinho}
	\indam{Quy tắc Hund.} Trong cùng một phân lớp, các electron sẽ phân bố trên các obitan sao cho số electron độc thân là tối đa và các electron này phải có chiều tự quay giống nhau.
\end{ghinho}
\begin{vidu}
	Sự phân bố electron trên các orbital của carbon và nitrogen như sau: \[ C(Z=6):\quad \underset{1s^2}{\squarerow[2ud][0.5][\maunhan][-4pt]{1}}\; \underset{2s^2}{\squarerow[2ud][0.5][\maunhan][-2pt]{1}}\hspace{-0.8pt} \underset{2p^2}{\squarerow[1u,1u][0.5][\maunhan][0pt]{3}} \hspace{2cm}
	N(Z=7):\quad \underset{1s^2}{\squarerow[2ud][0.5][\maunhan][-4pt]{1}}\; \underset{2s^2}{\squarerow[2ud][0.5][\maunhan][-2pt]{1}}\hspace{-0.8pt} \underset{2p^3}{\squarerow[1u,1u][0.5][\maunhan][0pt]{3}}
	\].
\end{vidu}
\begin{hoivadap}
	\begin{cauhoi}
		Trong các trường hợp (1) và (2) dưới đây trường hợp nào phân bố electron tuân theo quy tắc hund%
		\begin{center}
			$\underset{(1)}{\squarerow[2ud,1u][0.65][\maunhan]{3}}$
			\hspace{3cm} 
			$\underset{(2)}{\squarerow[1u,1u,1u][0.65][\maunhan]{3}}$
		\end{center}
	\end{cauhoi}
\end{hoivadap}
\Noibat{Viết cấu hình e}

Cấu hình electron của nguyên tử biểu diễn sự phân bố electron trên các phân lớp thuộc các lớp khác nhau.

Quy ước cách biểu diễn sự phân bố electron trên các phân lớp thuộc các lớp như sau:
\begin{center}
	\tikz[baseline=(char.base)]{\node[fill=\mycolor!20, rounded corners,inner sep=3pt,outer sep=0pt] (char) {
			\begin{tikzpicture}[declare function={d=3;}]
				\path (0,0) coordinate (A) node [font=\color{\maunhan}\fontsize{25pt}{0pt}\bfseries\fontfamily{qag}\selectfont, inner sep=0pt, outer sep=0pt] (lop) {1};
				\path (lop.south east) node [anchor=south west,font=\color{\mycolor}\fontsize{18pt}{0pt}\bfseries\fontfamily{qag}\selectfont, inner sep=0pt, outer sep=0pt](pl){S};
				\path (pl.north east) node [anchor=west,font=\color{\mauphu}\fontsize{14pt}{0pt}\bfseries\fontfamily{qag}\selectfont, inner sep=2pt, outer sep=0pt](e){2};
				\node [font=\color{\maunhan}\fontsize{12pt}{0pt}\bfseries\fontfamily{qag}\selectfont,left=1cm of lop, anchor= east] (sttl) {Số thứ tự lớp};
				\node [font=\color{\mycolor}\fontsize{12pt}{0pt}\bfseries\fontfamily{qag}\selectfont,right =1cm of pl, anchor= west, yshift=-5pt] (khpl) {Kí hiệu phân lớp};
				\node [font=\color{\mauphu}\fontsize{9pt}{0pt}\bfseries\fontfamily{qag}\selectfont,above =0.5cm of e, anchor= south] (soe) {Số e trên phân lớp};
				\draw[-stealth,\maunhan] (sttl.east)--(lop.west);
				\draw[-stealth,\mauphu] (soe.south)--(e.north);
				\draw[-stealth,\mycolor] (khpl.west)--([yshift=-5pt]pl.east);
			\end{tikzpicture}
		};}
\end{center}
\newpage
\begin{paracol}{2}
	\begin{tomtat}
		Các bước viết cấu hình electron
		\begin{cacbuoc}
			\item Xác định số electron của nguyên tử.
			\item Điền electron theo thứ tự các mức năng lượng từ thấp đến cao (theo dãy Klechkovski xem hình \ref{fig:QTKlecchkowsky}). Điền electron bão hoà phân lớp trước rồi mới điền tiếp vào phân lớp sau.
			\item Đổi lại vị trí các phân lớp sao cho số thứ tự lớp (n) tăng dần từ trái qua phải.
		\end{cacbuoc}
	\end{tomtat}
	\begin{vidu}
		$\mathrm{Ca}(\mathrm{Z}=20)$ Thứ tự mức năng lượng orbital : $1\mathrm{s}^22\mathrm{s}^22\mathrm{p}^63 \mathrm{s}^23\mathrm{p}^64\mathrm{s}^2$
		\\
		Cấu hình electron: $1\mathrm{s}^2 2 \mathrm{s}^2 2 \mathrm{p}^63 \mathrm{s}^23\mathrm{p}^6 4\mathrm{s}^2$
		hoặc viết gọn là: $[\mathrm{Ar}]4\mathrm{s}^2$
	\end{vidu}
	\switchcolumn
	%%%===Quy tắc Klechkowski=====%%%%
	\begin{Bancobiet}
		{Trật tự  năng lượng các AO được mô tả theo quy tắc đường chéo (quy tắc Klechkowski)}
		\begin{center}
			\begin{tikzpicture}[line join=round,line cap=round,line width=1pt]
				\tikzstyle{mynode} =[
				font=\color{white}\bfseries\fontfamily{qag}\selectfont,
				inner sep =4pt,
				outer sep =4pt,
				align =center,
				circle,
				text width =0.5cm,
				minimum width = 0.8cm,
				minimum height =0.8cm
				]
				\tikzstyle{mymatrix} = [
				matrix of nodes,
				nodes={mynode},
				column sep=3mm-\pgflinewidth,
				row sep = 3mm-\pgflinewidth,
				ampersand replacement=\&
				]
				\matrix(mt) [mymatrix]
				{%
					1s \&    \&    \&    \\
					2s \& 2p \&    \&    \\
					3s \& 3p \& 3d \&    \\
					4s \& 4p \& 4d \& 4f \\
					5s \& 5p \& 5d \&    \\
					6s \& 6p \&    \&    \\
					7s \& 	 \&    \&    \\
				};%
				\foreach \x/\y/\t/\u in {1/1/1/1,2/1/2/1,2/2/3/1,3/2/4/1,3/3/5/1,4/3/6/1,4/4/7/1}{
					\draw[-stealth,\maunhan!90!black] ($(mt-\x-\y.north east)+(45:1.0mm)$)--($(mt-\t-\u.south west)+(-135:2mm)$);
				}%
				\foreach \x/\y/\c/\n in {%
					1/1/\mycolor/1s,2/1/\mycolor/2s,3/1/\mycolor/3s,4/1/\mycolor/4s,5/1/\mycolor/5s,6/1/\mycolor/6s,7/1/\mycolor/7s,2/2/\mauphu/2p,3/2/\mauphu/3p,4/2/\mauphu/4p,5/2/\mauphu/5p,6/2/\mauphu/6p,3/3/\maunhan/3d,4/3/\maunhan/4d,5/3/\maunhan/5d,4/4/violet/4f
				}{%
					\path (mt-\x-\y) node [mynode,fill=\c!80!white] {\n};
				}
			\end{tikzpicture}%
			\captionof{figure}{Quy tắc Klechkowski\label{fig:QTKlecchkowsky}}
		\end{center}
		`\end{Bancobiet}
\end{paracol}
\Noibat {Biểu diễn cấu hình e theo orbital}
\vspace{0.3cm}
\begin{tomtat}
	\begin{cacbuoc}
		\item Viết cấu hình electron của nguyên tử.
		\item Biểu diễn mỗi AO bằng một ô vuông (ô orbital hay ô lượng tử), các AO trong cùng phân lớp thì viết liền nhau, các AO khác phân lớp thì viết tách nhau. Thứ tự các ô orbital từ trái sang phải theo thứ tự như ở cấu hình electron.
		\item Điền electron vào từng ô orbital theo thứ tự lớp và phân lớp, mỗi electron biểu diễn bằng một mũii tên. Trong mỗi phân lớp, electron được phân bố sao cho số electron độc thân là lớn nhất, electron được điền vào các ô orbital theo thứ tự từ trái sang phải. Trong một ô orbital, electron đầu tiên được biểu diễn bằng mũii tên quay lên, electron thứ hai được biểu diễn bằng mũi tên quay xuống.
	\end{cacbuoc}
\end{tomtat}
\begin{vidu}
	Cấu hình electron của nguyên tử Aluminium có $Z=13: 1s^22s^22p^63s^23p^1$ có thể được biểu diễn theo ô orbital như sau:
	\[\underset{\mathsf{1s^2}}{\squarerow[2ud][0.65][\maunhan]{1}}\;\;\;
	\underset{\mathsf{2s^2}}{\squarerow[2ud][0.65][\maunhan]{1}}\;
	\underset{\mathsf{2p^6}}{\squarerow[2ud,2ud,2ud][0.65][\maunhan]{3}}\;\;\;
	\underset{\mathsf{3s^2}}{\squarerow[2ud][0.65][\maunhan]{1}}\;
	\underset{\mathsf{3p^1}}{\squarerow[1u][0.65][\maunhan]{3}}
	\]
\end{vidu}
\Noibat {Đặc điểm cấu hình e lớp ngoài cùng}
\begin{center}
	\begin{tabular}{|l|c|c|c|c|}
		\hline\rowcolor{\mycolor!20} \indam{Số e lớp ngoài cùng} & \indam{1,2,3 e} & \indam{4 e} & \indam{5, 6, 7e} & \indam{8e (He, 2e)} \\
		\hline\rowcolor{\mycolor!20} \indam{Loại nguyên tố} & \indam{Kim loại} & \indam{KL hoặc PK} & \indam{Phi kim} & \indam{Khí hiếm} \\
		\hline
	\end{tabular}
\end{center}
\subsection{Các dạng bài tập}
 %%%=========Dạng 1=============%%%
\begin{dang}{Câu hỏi lý thuyết}
\end{dang}
\Noibat[][][\faCoffee]{Kiểu hỏi 1: Lý thuyết về mô hình nguyên tử}
%%%Ví dụ mẫu Dạng 1%%%
\Noibat[\maunhan][][\faBookmark]{Ví dụ mẫu}
\LGTNVD
%%%==============Vidu1==============%%%
\begin{vdex}[Trắc nghiệm nhiều lựa chọn]
	Theo mô hình Bohr, điều gì xảy ra khi một electron hấp thụ một photon có năng lượng chính xác bằng hiệu năng lượng giữa hai mức?
	\choice
	{\True Electron sẽ chuyển lên quỹ đạo có năng lượng cao hơn}
	{Electron sẽ thoát khỏi nguyên tử}
	{Electron sẽ vẫn ở nguyên quỹ đạo cũ}
	{Nguyên tử sẽ ion hóa}
	\loigiai{
		Khi electron hấp thụ photon có năng lượng bằng hiệu năng lượng giữa hai mức:
		\begin{itemchoice}
			\itemch \textbf{Đúng}. Electron sẽ hấp thụ photon và chuyển lên quỹ đạo có năng lượng cao hơn tương ứng.
			\itemch \textbf{Sai}. Electron chỉ thoát khỏi nguyên tử khi hấp thụ năng lượng lớn hơn năng lượng ion hóa.
			\itemch \textbf{Sai}. Nếu hấp thụ photon có năng lượng phù hợp, electron sẽ chuyển lên quỹ đạo cao hơn.
			\itemch \textbf{Sai}. Ion hóa chỉ xảy ra khi electron hấp thụ đủ năng lượng để thoát khỏi nguyên tử hoàn toàn.
		\end{itemchoice}
	}
\end{vdex}
\LGvdTF
%%%==============Vidu2==============%%%
\begin{vd}[Trắc nghiệm đúng sai]
	Trong mô hình nguyên tử Bohr, điều nào sau đây là đúng về các quỹ đạo electron?
	\choiceTF[t]
	{\True Các quỹ đạo electron là những vòng tròn cố định}
	{\True Mỗi quỹ đạo tương ứng với một mức năng lượng xác định}
	{Electron có thể tồn tại ở bất kỳ vị trí nào giữa các quỹ đạo}
	{Số quỹ đạo trong một nguyên tử bằng số nguyên tử của nó}
	\loigiai{
		Theo mô hình nguyên tử Bohr:
		\begin{itemchoice}
			\itemch \textbf{Đúng}. Bohr mô tả electron chuyển động trên các quỹ đạo tròn cố định.
			\itemch \textbf{Đúng}. Mỗi quỹ đạo có một mức năng lượng xác định, được gọi là trạng thái dừng.
			\itemch \textbf{Sai}. Electron chỉ tồn tại trên các quỹ đạo cố định, không thể ở giữa các quỹ đạo.
			\itemch \textbf{Sai}. Số quỹ đạo không bằng số nguyên tử. Trong mô hình Bohr, số quỹ đạo có thể là nhiều (lý thuyết là vô hạn), phụ thuộc vào trạng thái kích thích của nguyên tử.
		\end{itemchoice}
	}
\end{vd}
\Noibat[][][\faCoffee]{Kiểu hỏi 2: Lý thuyết về cấu trúc lớp vỏ electron}
%%%Ví dụ mẫu Dạng 1%%%
\Noibat[\maunhan][][\faBookmark]{Ví dụ mẫu}
%%%==============Cau_EX1==============%%%
\begin{vdex}
	Lớp electron thứ 3 có bao nhiêu phân lớp?
	\choice
	{$1$}
	{$2$}
	{\True$3$}
	{$4$}
	\loigiai{Số phân lớp bằng số thứ tự lớp}
\end{vdex}
%%%==============HetCau_EX1==============%%%

%%%==============Cau_EX2==============%%%
\begin{vdex}
	Phát biểu nào sao đây đúng?
	\choice
	{\True Số phân lớp electron có trong lớp N là 4}
	{Số phân lớp electron có trong lớp $M$ là 4}
	{Số orbital có trong lớp $N$ là 9}
	{Số orbital có trong lớp $M$ là 8}
	\loigiai{Lớp N ứng với n = 4 nên có 4 phân lớp electron.Lớp M ứng với n = 3 nên có 3 phân lớp electron.}
\end{vdex}
%%%==============HetCau_EX2==============%%%
\Noibat[][][\faCoffee]{Kiểu hỏi 3: Lý thuyết về cấu hình  electron}
%%%Ví dụ mẫu Dạng 1%%%
\Noibat[\maunhan][][\faBookmark]{Ví dụ mẫu}
%%%==============Cau_EX1==============%%%
\begin{vd}
	Sự phân bố electron trong một orbital dựa vào nguyên lí hay quy tắc nào sau đây?
	\choice
	{Nguyên lí vững bền}
	{Quy tắc Hund}
	{\True Nguyên lí Pauli}
	{Quy tắc Pauli}
	\loigiai{
		Sự phân bố electron trong một orbital dựa vào nguyên lý pau-li
	}
\end{vd}
%%%==============HetCau_EX1==============%%%

%%%==============Cau_EX2==============%%%
\begin{vdex}
	Sự phân bố electron trên các phân lớp thuộc các lớp electron dựa vào nguyên lí hay quy tắc nào sau đây?
	\choice
	{Nguyên lí vững bền và nguyên lí Pauli}
	{\True Nguyên lí vững bền và quy tắc Hund}
	{Nguyên lí Pauli và quy tắc Hund}
	{Nguyên lí vững bền và quy tắc Pauli}
	\loigiai{
		Sự phân bố electron trên các phân lớp thuộc các lớp electron dựa vào Nguyên lí vững bền và quy tắc Hund
	}
\end{vdex}
%%%==============HetCau_EX2==============%%%

%%%==============Cau_EX3==============%%%
\begin{vdex}
	Sự phân bố electron vào các lớp và phân lớp căn cứ vào
	\choice
	{nguyên tử khối tăng dần}
	{điện tích hạt nhân tăng dần}
	{số khối tăng dần}
	{\True mức năng lượng electron}
	\loigiai{}
\end{vdex}
%%%==============HetCau_EX3==============%%%


%%%==============Bài tập tự luyện dạng 1==============%%%
\Noibat[][][\faBank]{Bài tập tự luyện dạng \thedang}
%%%=========Câu hỏi trắc nghiệm 1 phương án=========%%%
\phan{Câu hỏi trắc nghiệm 1 phương án}
\Opensolutionfile{ansex}[Ans/LGEX-Hoa10_C01_CHE_BTTL01]
\Opensolutionfile{ans}[Ans/Ans-Hoa10_C01_CHE_BTTL01]
%\hienthiloigiaiex
%\tatloigiaiex
\luuloigiaiex
%%%==============Ex1==============%%%
\begin{ex}%[0H1N1-3]
	Trong mô hình nguyên tử hiện đại, orbital nguyên tử được mô tả như thế nào?
	\choice
	{\True Là vùng không gian có xác suất tìm thấy electron cao nhất}
	{Là quỹ đạo cố định mà electron chuyển động xung quanh hạt nhân}
	{Là lớp vỏ electron có hình cầu bao quanh hạt nhân}
	{Là đường đi của electron khi chuyển động quanh hạt nhân}
	\loigiai{
		Về orbital nguyên tử trong mô hình nguyên tử hiện đại:
		\begin{itemchoice}
			\itemch \textbf{Đúng}. Orbital là vùng không gian 3 chiều xung quanh hạt nhân, nơi có xác suất tìm thấy electron cao nhất.
			\itemch \textbf{Sai}. Đây là mô tả trong mô hình Bohr, không phải mô hình hiện đại.
			\itemch \textbf{Sai}. Lớp vỏ electron là khái niệm đơn giản hóa, không phản ánh đúng bản chất của orbital.
			\itemch \textbf{Sai}. Electron không có đường đi xác định trong mô hình hiện đại do tính chất sóng-hạt của nó.
		\end{itemchoice}
	}
\end{ex}

%%%==============Ex2==============%%%
\begin{ex}%[0H1H1-3]
	Theo mô hình Bohr, năng lượng của electron trong nguyên tử hydro phụ thuộc vào yếu tố nào?
	\choice
	{\True Số lượng tử chính n}
	{Số lượng tử phụ l}
	{Số lượng tử từ m}
	{Spin của electron}
	\loigiai{
		Về năng lượng của electron trong mô hình Bohr:
		\begin{itemchoice}
			\itemch \textbf{Đúng}. Trong mô hình Bohr, năng lượng của electron chỉ phụ thuộc vào số lượng tử chính n.
			\itemch \textbf{Sai}. Số lượng tử phụ l không được đề cập trong mô hình Bohr.
			\itemch \textbf{Sai}. Số lượng tử từ m không được sử dụng trong mô hình Bohr.
			\itemch \textbf{Sai}. Spin của electron chưa được biết đến trong thời điểm Bohr đề xuất mô hình của mình.
		\end{itemchoice}
	}
\end{ex}

%%%==============Ex3==============%%%
\begin{ex}%[0H1H1-3]
	Trong mô hình nguyên tử hiện đại, nguyên lý bất định Heisenberg phát biểu điều gì?
	\choice
	{\True Không thể xác định đồng thời chính xác vị trí và động lượng của electron}
	{Electron luôn chuyển động trên các quỹ đạo cố định}
	{Năng lượng của electron chỉ phụ thuộc vào khoảng cách từ nó đến hạt nhân}
	{Electron có thể được tìm thấy ở bất kỳ đâu trong nguyên tử với xác suất như nhau}
	\loigiai{
		Về nguyên lý bất định Heisenberg trong mô hình nguyên tử hiện đại:
		\begin{itemchoice}
			\itemch \textbf{Đúng}. Nguyên lý này chỉ ra giới hạn trong việc xác định đồng thời vị trí và động lượng của electron.
			\itemch \textbf{Sai}. Đây là quan điểm của mô hình Bohr, không phải mô hình hiện đại.
			\itemch \textbf{Sai}. Trong mô hình hiện đại, năng lượng của electron phụ thuộc vào nhiều yếu tố hơn.
			\itemch \textbf{Sai}. Xác suất tìm thấy electron không đồng đều trong toàn bộ nguyên tử.
		\end{itemchoice}
	}
\end{ex}

%%%==============Ex4==============%%%
\begin{ex}%[0H1H1-3]
	Trong mô hình nguyên tử hiện đại, số lượng tử spin ms có thể nhận những giá trị nào?
	\choice
	{\True +1/2 và -1/2}
	{0 và 1}
	{-1, 0, và +1}
	{Bất kỳ giá trị nào từ -1 đến +1}
	\loigiai{Số lượng tử spin ms chỉ có thể nhận hai giá trị là +1/2 và -1/2.}
\end{ex}

%%%==============Ex5==============%%%
\begin{ex}%[0H1H1-3]
	Theo mô hình Bohr, điều gì xảy ra khi electron chuyển từ trạng thái kích thích về trạng thái cơ bản?
	\choice
	{\True Electron phát ra một photon}
	{Electron hấp thụ một photon}
	{Nguyên tử trở nên trung hòa về điện}
	{Không có gì xảy ra, electron vẫn giữ nguyên năng lượng}
	\loigiai{
		Về sự chuyển dời electron trong mô hình Bohr:
		\begin{itemchoice}
			\itemch \textbf{Đúng}. Khi electron chuyển từ trạng thái năng lượng cao xuống thấp, nó phát ra một photon.
			\itemch \textbf{Sai}. Electron hấp thụ photon khi chuyển lên trạng thái kích thích, không phải khi về cơ bản.
			\itemch \textbf{Sai}. Trạng thái điện của nguyên tử không thay đổi trong quá trình này.
			\itemch \textbf{Sai}. Electron mất một lượng năng lượng bằng với năng lượng của photon phát ra.
		\end{itemchoice}
	}
\end{ex}

%%%==============Ex6==============%%%
\begin{ex}%[0H1H1-3]
	Trong mô hình nguyên tử hiện đại, nguyên lý Pauli phát biểu điều gì?
	\choice
	{\True Trong một nguyên tử, không có hai electron nào có bộ 4 số lượng tử giống nhau}
	{Các electron luôn chuyển động theo cặp trong orbital}
	{Mỗi orbital chỉ có thể chứa tối đa một electron}
	{Các electron trong cùng một lớp có cùng mức năng lượng}
	\loigiai{
		Về nguyên lý Pauli trong mô hình nguyên tử hiện đại:
		\begin{itemchoice}
			\itemch \textbf{Đúng}. Nguyên lý Pauli khẳng định rằng không có hai electron nào trong một nguyên tử có thể có cùng bộ 4 số lượng tử.
			\itemch \textbf{Sai}. Mặc dù electron thường ghép cặp trong orbital, đây không phải là nội dung của nguyên lý Pauli.
			\itemch \textbf{Sai}. Mỗi orbital có thể chứa tối đa hai electron với spin ngược nhau.
			\itemch \textbf{Sai}. Các electron trong cùng một lớp có thể có mức năng lượng khác nhau, phụ thuộc vào orbital cụ thể.
		\end{itemchoice}
	}
\end{ex}

%%%==============Ex7==============%%%
\begin{ex}%[0H1H1-3]
	Trong mô hình Bohr, bán kính quỹ đạo của electron trong nguyên tử hydro tỉ lệ với yếu tố nào sau đây?
	\choice
	{\True Bình phương của số lượng tử chính n}
	{Số lượng tử chính n}
	{Căn bậc hai của số lượng tử chính n}
	{Nghịch đảo của số lượng tử chính n}
	\loigiai{
		Về bán kính quỹ đạo trong mô hình Bohr:
		\begin{itemchoice}
			\itemch \textbf{Đúng}. Bán kính quỹ đạo r tỉ lệ với $n^2$, trong đó n là số lượng tử chính.
			\itemch \textbf{Sai}. Mối quan hệ không phải là tỉ lệ thuận đơn giản với n.
			\itemch \textbf{Sai}. Bán kính không tỉ lệ với căn bậc hai của n.
			\itemch \textbf{Sai}. Bán kính tăng khi n tăng, không phải giảm.
		\end{itemchoice}
	}
\end{ex}

%%%==============Ex8==============%%%
\begin{ex}%[0H1H1-3]
	Trong mô hình nguyên tử hiện đại, orbital p có hình dạng như thế nào?
	\choice
	{\True Hình số 8 (hai thuỳ)}
	{Hình cầu}
	{Hình tròn phẳng}
	{Hình bông hoa bốn cánh}
	\loigiai{
		Về hình dạng của orbital p trong mô hình nguyên tử hiện đại:
		\begin{itemchoice}
			\itemch \textbf{Đúng}. Orbital p có hình dạng số 8 với hai thuỳ đối xứng qua hạt nhân.
			\itemch \textbf{Sai}. Hình cầu là đặc trưng của orbital s, không phải p.
			\itemch \textbf{Sai}. Không có orbital nào có hình tròn phẳng trong mô hình hiện đại.
			\itemch \textbf{Sai}. Hình bông hoa bốn cánh gần giống với một số orbital d, không phải p.
		\end{itemchoice}
	}
\end{ex}
%%%==============Cau_EX9==============%%%
\begin{ex}%[0H1H1-3]
	Lớp electron thứ 4 có bao nhiêu orbital tối đa?
	\choice
	{$8$}
	{$16$}
	{\True$32$}
	{$64$}
	\loigiai{Số orbital tối đa trong lớp n là $n^2$. Với lớp thứ 4, $n = 4$, nên số orbital tối đa là $4^2 = 16$.}
\end{ex}
%%%==============HetCau_EX9==============%%%

%%%==============Cau_EX10==============%%%
\begin{ex}%[0H1H1-3]
	Phân lớp nào sau đây không tồn tại trong lớp M?
	\choice
	{$3s$}
	{$3p$}
	{$3d$}
	{\True$3f$}
	\loigiai{Lớp M ứng với $n = 3$, chỉ có các phân lớp s, p, và d. Phân lớp f chỉ xuất hiện từ lớp N ($n = 4$) trở đi.}
\end{ex}
%%%==============HetCau_EX10==============%%%

%%%==============Cau_EX11==============%%%
\begin{ex}%[0H1H1-3]
	Số electron tối đa trong phân lớp 3p là bao nhiêu?
	\choice
	{$2$}
	{$4$}
	{\True$6$}
	{$8$}
	\loigiai{Phân lớp p có 3 orbital, mỗi orbital chứa tối đa 2 electron. Vậy số electron tối đa trong phân lớp 3p là $3 \cdot 2 = 6$.}
\end{ex}
%%%==============HetCau_EX11==============%%%

%%%==============Cau_EX12==============%%%
\begin{ex}%[0H1V1-3]
	Trong các phát biểu sau, phát biểu nào đúng?
	\choice
	{Phân lớp 2d có thể tồn tại}
	{Lớp L có 3 phân lớp electron}
	{\True Số electron tối đa trong lớp N là 32}
	{Phân lớp 4f có 5 orbital}
	\loigiai{Số electron tối đa trong lớp N (n = 4) là $2n^2 = 2 \cdot 4^2 = 32$. Các phát biểu khác đều sai: không có phân lớp 2d, lớp L chỉ có 2 phân lớp, và phân lớp 4f có 7 orbital.}
\end{ex}
%%%==============HetCau_EX12==============%%%
%%%==============Cau_EX13==============%%%
\begin{ex}%[0H1V1-3]
	Trong một nguyên tử, electron ở lớp nào có năng lượng cao nhất?
	\choice
	{Lớp K}
	{Lớp L}
	{Lớp M}
	{\True Lớp ngoài cùng}
	\loigiai{Electron ở lớp ngoài cùng của nguyên tử có năng lượng cao nhất vì chúng ở xa hạt nhân nhất, chịu lực hút tĩnh điện yếu nhất từ hạt nhân và dễ dàng tham gia vào các phản ứng hóa học.}
\end{ex}
%%%==============HetCau_EX13==============%%%

%%%==============Cau_EX14==============%%%
\begin{ex}%[0H1N1-3]
	Orbital nguyên tử là
	\choice
	{đám mây chứa electron có dạng hình cầu}
	{đám mây chứa electron có dạng hình số 8 nổi}
	{\True khu vực không gian xung quanh hạt nhân mà tại đó xác suất có mặt electron lớn nhất}
	{quỹ đạo chuyển động của electron quay quanh hạt nhân có kích thước và năng lượng xác định}
	\loigiai{
		Orbital là khu vực không gian xung quanh hạt nhân mà tại đó xác suất có mặt electron lớn nhất.
	}
\end{ex}
%%%==============HetCau_EX14==============%%%
%%%==============Cau_EX15==============%%%
\begin{ex}%[0H1V1-3]
	Cách biểu diễn electron trong AO nào sau đây không tuân theo nguyên lí Pau-li?
	\choice
	{$\squarerow[1u][0.65][\mycolor][-4pt]{1}$}
	{$\squarerow[1d][0.65][\mycolor][-4pt]{1}$}
	{\True$\squarerow[2ud][0.65][\mycolor][-4pt]{1}$}
	{$\squarerow[2uu][0.65][\mycolor][-4pt]{1}$}
	\loigiai{
		Theo nguyên lý pau-li thì mỗi một orbital có chứ tối đa 2 electron.
		\begin{itemize}
			\item Nếu AO chứa 1 e thì vẽ mũi tên hướng lên
			\item Nếu AO chứa 2 e thì mũi tên bên trái hướng lên và mũi tên bên phải hướng xuống.
		\end{itemize}
	}
\end{ex}
%%%==============HetCau_EX15==============%%%
%%%==============Cau_EX16==============%%%
\begin{ex}%[0H1V1-3]
	Sự phân bố electron theo ô orbital nào dưới đây là đúng?
	\choice
	{$\squarerow[2ud][0.65][\mycolor][-4pt]3$}
	{\True $\squarerow[1u,1u,1u][0.65][\mycolor][-4pt]3$}
	{$\squarerow[2ud,2ud][0.65][\mycolor][-4pt]3$}
	{$\squarerow[2ud,1d,1u][0.65][\mycolor][-4pt]3$}
	\loigiai{
		Theo quy tắc hund trên một phân lớp các electron phân bố sao cho tổng số e độc thân là lớn nhất và các electron độc thân phải có chiều tự quay giống nahu
	}
\end{ex}
%%%==============HetCau_EX16==============%%%

%%%==============Cau_EX17==============%%%
\begin{ex}%[0H1V1-3]
	Số electron tối đa có trong lớp M
	\choice
	{$6$}
	{$9$}
	{$14$}
	{\True$18$}
	\loigiai{
		Lớp thứ n có $2n^2$ electron. Lớp M ứng với n=3 có $2\cdot3^2=18$ (electron).
	}
\end{ex}
%%%==============HetCau_EX17==============%%%
%%%==================EX18==================%%%
\begin{ex}%[0H1V1-3]
	Các electron của nguyên tử nguyên tố X được phân bố trên ba lớp, lớp thứ ba có 6 electron. Số đơn vị điện tích hạt nhân của nguyên tử nguyên tố X là
	\choice
	{$6$}
	{$8$}
	{$14$}
	{\True $16$}
	\loigiai{
		Nguyên tố X phân bố electron trên 3 lớp. Theo nguyên lý nững bền sau khi điền đủ số electron tối đa ở lớp 1 (2 e) , ở lớp 2 (8 e) sẽ điền tiếp 6 electron còn lại vào lớp 3. Do đó Nguyên tố X có tổng cộng $2+8+6=16$ (electron).
	}
\end{ex}
%%%==============HetCau_EX18==============%%%

%%%==============Cau_EX19==============%%%
\begin{ex}%[0H1V1-3]
	Nguyên tố X có $Z=17$. Electron lớp ngoài cùng của nguyên tử nguyên tố X thuộc lớp
	\choice
	{K}
	{L}
	{\True M}
	{N}
	\loigiai{
		Nguyên tố X có cấu hình electron là : $1s^22s^22p^63s^23p^5$ E lectron cuối cùng thuộc phân lớp $3p^5$ có số thứ tự lớp $n=3$ hay lớp thứ M. 
	}
\end{ex}
%%%==============HetCau_EX19==============%%%

\Closesolutionfile{ansex}
\Closesolutionfile{ans}


\phan{Câu hỏi trắc nghiệm đúng sai}
%%%%====================%%%
\Opensolutionfile{ansex}[Ans/LGTF-Hoa10_C01_CHE_BTTL01]
\Opensolutionfile{ansbook}[Ansbook/AnsTF-Hoa10_C01_CHE_BTTL01]
\Opensolutionfile{ans}[Ans/Tempt-Hoa10_C01_CTNT_BTTL01]
\luulgEXTF
%%%==============EX01==============%%%
\begin{ex}%[0H1V1-3]
	Đâu là những hạn chế của mô hình nguyên tử Bohr?
	\choiceTF[t]
	{\True Không giải thích được phổ của các nguyên tử phức tạp}
	{Không dự đoán được sự tồn tại của các đồng vị}
	{\True Không giải thích được liên kết hóa học}
	{\True Không phù hợp với nguyên lý bất định Heisenberg}
	\loigiai{
		Hạn chế của mô hình nguyên tử Bohr:
		\begin{itemchoice}
			\itemch \textbf{Đúng}. Mô hình Bohr chỉ giải thích tốt phổ của nguyên tử hydro và các ion giống hydro.
			\itemch \textbf{Sai}. Sự tồn tại của đồng vị không liên quan trực tiếp đến mô hình Bohr.
			\itemch \textbf{Đúng}. Mô hình này không đề cập đến cơ chế hình thành liên kết hóa học.
			\itemch \textbf{Đúng}. Mô hình Bohr xác định chính xác vị trí và động lượng của electron, trái với nguyên lý bất định Heisenberg.
		\end{itemchoice}
	}
\end{ex}
%%%==============EX02==============%%%
\begin{ex}%[0H1V1-3]
	Về sự chuyển dời của electron trong mô hình Bohr, phát biểu nào sau đây là chính xác?
	\choiceTF[t]
	{\True Khi electron chuyển từ quỹ đạo cao xuống quỹ đạo thấp hơn, nó phát ra photon}
	{Electron có thể chuyển giữa các quỹ đạo mà không cần hấp thụ hoặc phát ra năng lượng}
	{\True Năng lượng của photon phát ra bằng hiệu năng lượng giữa hai mức}
	{Electron luôn chuyển xuống mức năng lượng thấp nhất khi bị kích thích}
	\loigiai{
		Về sự chuyển dời của electron trong mô hình Bohr:
		\begin{itemchoice}
			\itemch \textbf{Đúng}. Khi electron chuyển từ quỹ đạo cao xuống thấp, nó giải phóng năng lượng dưới dạng photon.
			\itemch \textbf{Sai}. Mọi sự chuyển dời của electron đều liên quan đến việc hấp thụ hoặc phát ra năng lượng.
			\itemch \textbf{Đúng}. Năng lượng của photon phát ra chính xác bằng hiệu năng lượng giữa hai mức electron chuyển đổi.
			\itemch \textbf{Sai}. Electron có thể chuyển đến bất kỳ mức năng lượng cao hơn nào khi được kích thích đủ.
		\end{itemchoice}
	}
\end{ex}
%%%==================EX03==================%%%
\begin{ex}%[0H1V1-3]
	Về sự phân bố electron vào các lớp và phân lớp, hãy chọn những phát biểu đúng:
	\choiceTF[t]
	{\True Electron được sắp xếp vào các orbital theo nguyên lý vững bền}
	{Số electron tối đa trong một lớp luôn bằng $2n^2$, với n là số lớp e}
	{\True Phân lớp d bắt đầu được điền electron từ lớp thứ 3 trở đi}
	{\True Trong cùng một lớp, phân lớp s có mức năng lượng thấp nhất}
	\loigiai{
		Về sự phân bố electron vào các lớp và phân lớp:
		\begin{itemchoice}
			\itemch \textbf{Đúng}. Nguyên lý vun đắp quy định thứ tự điền electron vào các orbital.
			\itemch \textbf{Sai}. Công thức $2n^2$ chỉ đúng cho 4 lớp đầu tiên. Từ lớp thứ 5 trở đi, số electron tối đa không theo quy luật này.
			\itemch \textbf{Đúng}. Phân lớp d xuất hiện từ lớp $n = 3$ trở đi.
			\itemch \textbf{Đúng}. Trong cùng một lớp, thứ tự mức năng lượng tăng dần là $s < p < d < f$.
		\end{itemchoice}
	}
\end{ex}
%%%==================EX04==================%%%
\begin{ex}%[0H1H1-3]
	Xét các phát biểu sau về cấu hình electron, chọn những phát biểu đúng:
	\choiceTF[t]
	{\True Số electron tối đa trong phân lớp p là 6}
	{Phân lớp f chứa tối đa 10 electron}
	{\True Tổng số orbital trong một phân lớp luôn là số lẻ}
	{Số electron tối đa trong một orbital luôn là 1}
	\loigiai{
		Về cấu hình electron:
		\begin{itemchoice}
			\itemch \textbf{Đúng}. Phân lớp p có 3 orbital, mỗi orbital chứa tối đa 2 electron, nên tổng cộng là 6 electron.
			\itemch \textbf{Sai}. Phân lớp f chứa tối đa 14 electron (7 orbital, mỗi orbital 2 electron).
			\itemch \textbf{Đúng}. Số lượng orbital trên các phân lớp $s, p, d, f$ lần lượt là $1, 3, 5, 7$.
			\itemch \textbf{Sai}. Số electron tối đa trong một orbital là 2, tuân theo nguyên lý Pauli.
		\end{itemchoice}
	}
\end{ex}
%%%==================EX05==================%%%
\begin{ex}%[0H1H1-3]
	Về quy tắc Hund trong việc phân bố electron, hãy chọn những phát biểu đúng:
	\choiceTF[t]
	{\True Các electron sẽ chiếm hết các orbital cùng năng lượng trước khi ghép đôi}
	{\True Các electron trong các orbital cùng năng lượng sẽ có spin cùng chiều}
	{Quy tắc Hund chỉ áp dụng cho các nguyên tố thuộc nhóm p}
	{Electron luôn ghép đôi trong cùng một orbital trước khi điền vào orbital khác}
	\loigiai{
		Về quy tắc Hund:
		\begin{itemchoice}
			\itemch \textbf{Đúng}. Đây là nội dung cơ bản của quy tắc Hund.
			\itemch \textbf{Đúng}. Các electron đơn trong các orbital cùng năng lượng sẽ có spin cùng chiều để đạt trạng thái bền vững nhất.
			\itemch \textbf{Sai}. Quy tắc Hund áp dụng cho tất cả các nguyên tố, không chỉ riêng nhóm p.
			\itemch \textbf{Sai}. Điều này trái với quy tắc Hund, electron sẽ điền đơn vào các orbital trước khi ghép đôi.
		\end{itemchoice}
	}
\end{ex}
%%%==================EX06==================%%%
\begin{ex}%[0H1H1-3]
	Về sự phân bố electron trong nguyên tử, hãy chọn những phát biểu đúng:
	\choiceTF[t]
	{Nguyên tử luôn ở trạng thái cơ bản trong mọi điều kiện}
	{\True Cấu hình electron của ion được xác định sau khi thêm hoặc bớt electron từ nguyên tử trung hòa}
	{\True Trong trạng thái cơ bản, electron được sắp xếp sao cho tổng năng lượng thấp nhất}
	{\True Các electron trong cùng một phân lớp có mức năng lượng bằng nhau.}
	\loigiai{
		Về sự phân bố electron trong nguyên tử:
		\begin{itemchoice}
			\itemch \textbf{Sai}. Nguyên tử có thể ở trạng thái kích thích khi được cung cấp năng lượng.
			\itemch \textbf{Đúng}. Cấu hình electron của ion được xác định bằng cách thêm hoặc bớt electron từ nguyên tử trung hòa.
			\itemch \textbf{Đúng}. Trạng thái cơ bản là trạng thái có năng lượng thấp nhất.
			\itemch \textbf{Đúng}. Các electron trong cùng một phân lớp có mức năng lượng bằng nhau, các electron thuộc cùng một lớp có mức năng lượng gần bằng nhau.
		\end{itemchoice}
	}
\end{ex}
%%%==============EX07==============%%%
\begin{ex}%[0H1H1-3]
	Trong mô hình Bohr, điều nào sau đây là đúng về bán kính quỹ đạo electron?
	\choiceTF[t]
	{\True Bán kính quỹ đạo tỉ lệ với bình phương của số lượng tử chính n}
	{Bán kính quỹ đạo không phụ thuộc vào số lượng tử chính}
	{\True Quỹ đạo có năng lượng cao hơn có bán kính lớn hơn}
	{Bán kính quỹ đạo tỉ lệ nghịch với số proton trong hạt nhân}
	\loigiai{
		Về bán kính quỹ đạo electron trong mô hình Bohr:
		\begin{itemchoice}
			\itemch \textbf{Đúng}. Bán kính quỹ đạo $r = n^2 \cdot a_0$ , với n là số lượng tử chính, $a_0$ là bán kính Borh (hằng số).
			\itemch \textbf{Sai}. Bán kính quỹ đạo phụ thuộc trực tiếp vào số lượng tử chính n.
			\itemch \textbf{Đúng}. Quỹ đạo có n lớn hơn sẽ có bán kính lớn hơn và năng lượng cao hơn.
			\itemch \textbf{Sai}. Bán kính quỹ đạo tỉ lệ nghịch với số nguyên tử Z, không phải tỉ lệ nghịch với số proton.
		\end{itemchoice}
	}
\end{ex}
%%%==============EX08==============%%%
\begin{ex}%[0H1V1-3]
	Cho cấu hình electron của nguyên tử X là $1s^22s^22p^63s^23p^4$.
	\choiceTF[t]
	{\True X có số hiệu nguyên tử $Z = 16$}
	{lớp M có 4 electron}
	{\True X thuộc nguyên tố p}
	{ phân lớp có năng lượng cao nhất có cấu hình electron theo AO là \squarerow[2ud,2ud][0.65][\mycolor][-4pt]{3}}
	\loigiai{
		Về nguyên tố X có cấu hình electron $1s^²2s^²2p^⁶3s^²3p^4$:
		\begin{itemchoice}
			\itemch \textbf{Đúng}.Dựa vào cấu hình electron ta thấy X có 16 electron do đó X có $Z=16$.
			\itemch \textbf{Sai}.X có 3 lớp electron, lớp M (n=3) hay lớp thứ 3 có 6 electron.
			\itemch \textbf{Đúng}.Trong cấu hình của X electron cuối cùng điền vào phân lớp p nên X là nguyên tố p.
			\itemch \textbf{Sai}.Theo quy tắc Hund 3 electron độc thân sẽ chiếm 3 AO của phân lớp 3p trước và electron cuối cùng phải tham gia ghép đôi nê cấu hình đúng phải là \squarerow[2ud,1u,1u][0.65][\mycolor][-4pt]{3}.
		\end{itemchoice}
	}
\end{ex}
%%%==============EX09==============%%%

\Closesolutionfile{ans}
\Closesolutionfile{ansbook}
\Closesolutionfile{ansex}
%\bangdapanTF{AnsTF-Hoa10_C01_CHE_BTTL01}

%%%=========Dạng 2=============%%%
\begin{dang}{Bài tập về cấu hình electron}\end{dang}
%%%bài toán 1%%%
\Noibat[][][\faCoffee]{Bài toán 1: Viết cấu hình electron }
\begin{pp}
	\begin{cacbuoc}
		\item xác định số electron
		\item Phân bố electron vào các lớp và phân lớp theo thứ tự mức năng lượng:
		\[1s<2s<2p<3s<3p<4s<3d<...\] sao cho phân lớp s tối đa 2e, phân lớp p tối đa 6e , phân lớp d 10 e và phân lớp f 14 e. Khi phân bố e trên một phân lớp các electron độc thân phải chiếm các AO trước sau đó mới ghép đôi cho đủ số e tôi đa mói chuyển sang phân lớp khác có ngăng lượng cao hơn.
		\item Đảo lại  sao cho sô thứ tự lớp tăng dần và thứ tự các phân lớp là s,p,d,f
	\end{cacbuoc}
\end{pp}
%%%Ví dụ mẫu bài toán 1%%%
\Noibat[\maunhan][][\faBookmark]{Ví dụ mẫu}
%%%=======Bắt đầu ví dụ mau 1========%%%
\hienthiloigiaivd
\begin{vdex}
	Viết cấu hình electron của $Na$ ($Z=11$), $Ca$ ($Z=19$), $Cl$ ($Z=17$) và $Mn$ ($Z=25$).
	\loigiai{
		\begin{itemize}
			\item $Na$ ($Z=11$): $1s^22s^22p^63s^1$
			\item $Ca$ ($Z=20$): $1s^22s^22p^63s^23p^64s^2$
			\item $Cl$ ($Z=17$): $1s^22s^22p^63s^23p^5$
			\item $Mn$ ($Z=25$): $1s^22s^22p^63s^23p^63d^54s^2$
		\end{itemize}
	}
\end{vdex}
%%%=======Kết thúc ví dụ mau 1========%%%
\begin{vdex}
	Biểu diễn cấu hình electron của các nguyên tử có $X=6$ và $Y=15$ theo ô orbital.
	\loigiai{
		\begin{itemize}
			\item Cấu hình e của $X$ là $1s^22s^22p^2$ biểu diễn theo ô orbital như sau:
			\[
			\underset{\mathsf{1s^2}}{\squarerow[2ud]{1}}\;\;
			\underset{\mathsf{2s^2}}{\squarerow[2ud]{1}}\,
			\underset{\mathsf{2p^2}}{\squarerow[1u,1u]{3}}
			\]
			\item Cấu hình e của $Y$ là $1s^22s^22p^63s^23p^3$ biểu diễn theo ô orbital như sau:
			\[
			\underset{\mathsf{1s^2}}{\squarerow[2ud]{1}}\;\;
			\underset{\mathsf{2s^2}}{\squarerow[2ud]{1}}\,
			\underset{\mathsf{2p^6}}{\squarerow[2ud,2ud,2ud]{3}}\;\;
			\underset{\mathsf{3s^2}}{\squarerow[2ud]{1}}\,
			\underset{\mathsf{3p^3}}{\squarerow[1u,1u,1u]{3}}
			\]
		\end{itemize}
	}
\end{vdex}
%%%============bài toán 2=======================%%%
\Noibat[][][\faCoffee]{Bài toán 2: Dựa vào cấu hình electron giải thích một số tính chất}
\begin{pp}
	Các electron ở lớp ngoài cùng quyết định tính chất hóa học của một nguyên tố.
	\begin{itemize}
		\item Đối với nguyên tử của các nguyên tố, số electron lớp ngoài cùng tối đa là 8 đó là các nguyên tử khí hiếm (trừ He có 2 e ở lớp ngoài cùng) chúng hầu như không tham gia vào phản ứng hóa học.
		\item Các nguyên tử có 1, 2, 3 electron ở lớp ngoài cùng là các nguyên tử kim loại ( trừ H, He và B)
		\item Các nguyên tử có 5,6,7 electron ở lớp ngoài cùng thường là các nguyên tử phi kim.
		\item Các nguyên tử có 4 electron  ở lóp ngoài cùng  có thể là nguyên tử kim loại hay phi kim.
	\end{itemize}
\end{pp}
%%%Ví dụ mẫu bài toán 2%%%
\Noibat[\maunhan][][\faBookmark]{Ví dụ mẫu}
%%%==============Bat dau_VDM1==============%%%
\begin{vdex}
	Phosphorus (P) là một nguyên tố quan trọng trong cơ thể sống, đặc biệt là trong cấu trúc xương và răng. Hãy viết cấu hình electron của nguyên tử phosphorus $(Z=15)$ theo ô orbital và giải thích việc áp dụng các nguyên lý vững bền, nguyên lý Pauli và quy tắc Hund.
	\loigiai{
		Cấu hình electron của phosphorus $(Z=15):1s^22s^22p^63s^23p^3$
		\\
		Biểu diễn theo ô orbital:
		\[
		\underset{\mathsf{1s^2}}{\squarerow[2ud]{1}}\;\;
		\underset{\mathsf{2s^2}}{\squarerow[2ud]{1}}\;
		\underset{\mathsf{2p^6}}{\squarerow[2ud,2ud,2ud]{3}}\;\;
		\underset{\mathsf{3s^2}}{\squarerow[2ud]{1}}\;
		\underset{\mathsf{3p^3}}{\squarerow[1u,1u,1u]{3}}
		\]
		Giải thích:
		\begin{enumerate}[1.]
			\item Nguyên lý vững bền: Electron được sắp xếp vào các orbital có mức năng lượng thấp nhất trước (1s, 2s, 2p, 3s, 3p).
			\item Nguyên lý Pauli: Mỗi orbital chứa tối đa 2 electron có spin ngược nhau.
			\item Quy tắc Hund: Trong orbital 3p, 3 electron được phân bố vào 3 orbital khác nhau với spin cùng chiều để đạt trạng thái năng lượng thấp nhất.
		\end{enumerate}
	}
\end{vdex}
%%%==============Bat dau_VDM2==============%%%
\begin{vdex}
	Calcium $(Z=20)$ là thành phần quan trọng trong xương và răng. Viết cấu hình electron của nguyên tử calcium và giải thích tại sao calcium là một kim loại kiềm thổ.
	\loigiai{
		Cấu hình electron của calcium $(Z=20)$: $1s^22s^22p^63s^23p^64s^2$
		\\
		Calcium là một kim loại kiềm thổ vì:
		\begin{itemize}
			\item Nó có 2 electron ở lớp ngoài cùng ($4s^2$), đặc trưng cho nhóm IIA (nhóm 2).
			\item Các electron này dễ dàng bị mất đi để tạo thành ion $Ca^{2+}$, cho phép calcium tham gia vào các phản ứng hóa học đặc trưng của kim loại.
			\item Cấu trúc electron này tạo ra tính kim loại mạnh, nhưng không mạnh bằng kim loại kiềm (có 1 electron lớp ngoài cùng).
		\end{itemize}
	}
\end{vdex}
%%%===============Bài toán 3====================%%%
\Noibat[][][\faCoffee]{Bài toán 3: Cấu hình electron của ion}
\begin{pp}
	Nguyên tử nhường , nhận electron để trở thành ion
	\begin{itemize}
		\item $X + ne \xrightarrow X^{n-}$
		\item $Y  \xrightarrow X^{m+} + me$
	\end{itemize}
	Cách viết cấu hình electron cho ion 
	\begin{cacbuoc}
		\item  Viết cấu hình electron của nguyên tử trung hoà $\mathrm{X}^0$.
		\item Thêm (nếu viết cho $\mathrm{X}^{\mathrm{n-}}$ ) hoặc bớt (nếu viết cho $\mathrm{X}^{\mathrm{n+}}$ ) n electron trên phân lớp ngoài cùng của cấu hình electron $\mathrm{X}^0$.
	\end{cacbuoc}
\end{pp}
%%%Ví dụ mẫu bài toán 3%%%
\Noibat[\maunhan][][\faBookmark]{Ví dụ mẫu}
\begin{vdex}
	Viết cấu hình electron của các ion sau: $\text{Ca}^{2+}$, $\text{Fe}^{3+}$, $\text{O}^{2-}$
	\loigiai{
		\begin{itemize}
			\item $\text{Ca}^{2+}$:
			\begin{itemize}
				\item Ca có Z = 20, cấu hình electron: $1s^2 2s^2 2p^6 3s^2 3p^6 4s^2$
				\item $\text{Ca}^{2+}$ mất 2 electron ở lớp ngoài cùng
				\item Cấu hình electron của $\text{Ca}^{2+}$: $1s^2 2s^2 2p^6 3s^2 3p^6$
			\end{itemize}
			\item $\text{Fe}^{3+}$:
			\begin{itemize}
				\item Fe có Z = 26, cấu hình electron: $1s^2 2s^2 2p^6 3s^2 3p^6  3d^6 4s^2$
				\item $\text{Fe}^{3+}$ mất 2 electron ở lớp 4s và 1 electron ở lớp 3d
				\item Cấu hình electron của $\text{Fe}^{3+}$: $1s^2 2s^2 2p^6 3s^2 3p^6 3d^5$
			\end{itemize}
			\item $\text{O}^{2-}$:
			\begin{itemize}
				\item O có Z = 8, cấu hình electron: $1s^2 2s^2 2p^4$
				\item $\text{O}^{2-}$ nhận thêm 2 electron ở lớp 2p
				\item Cấu hình electron của $\text{O}^{2-}$: $1s^2 2s^2 2p^6$
			\end{itemize}
		\end{itemize}
		Nhận xét:
		\begin{itemize}
			\item $\text{Ca}^{2+}$ có cấu hình electron giống khí hiếm Ar
			\item $\text{Fe}^{3+}$ có cấu hình electron nửa bão hòa ở lớp 3d
			\item $\text{O}^{2-}$ có cấu hình electron giống khí hiếm Ne
		\end{itemize}
	}
\end{vdex}
%%%===============Bài toán 4====================%%%
\Noibat[][][\faCoffee]{Bài toán 4: Xác định  nguyên tố dựa vào số e trên các phân lớp}
\begin{pp}
	\begin{itemize}
		\item Số electron tối đa trên các phân lớp  là s, p, d,f lần lượt là 2, 6,10,14 electron.
		\item Điền electron theo thứ tự múc năng lượng sao cho đủ số electron trên các phan lớp theo yêu cầu của đề bài
	\end{itemize}
\end{pp}
%%%Ví dụ mẫu bài toán 4%%%
\Noibat[\maunhan][][\faBookmark]{Ví dụ mẫu}
%%%=============Ví dụ mẫu 1================%%%
\begin{vdex}
	Một nguyên tử X có tổng số electron ở phân lớp p là 11. X là nguyên tố nào?
	\choice
	{K}
	{\True Cl}
	{Si}
	{Ca}
	\loigiai{Vì tổng số electron trên phân lớp p là 11 nên có 6 electron phân bố vào phân lớp 2p và 5 elctron còn lại sẽ phân bố vào phân lớp 3p do đó X có cấu hình electron là $1s^22s^22p^63s^23p^5$ $\Rightarrow$  X là nguyên tố Cl}
\end{vdex}
%%%=============Ví dụ mẫu 2================%%%
\begin{vdex}
	Một nguyên tử R có tổng số electron ở các phân lớp d là 5. R là nguyên tố nào?
	\choice
	{Fe}
	{Cu}
	{\True Mn}
	{Zn}
	\loigiai{Tổng số electron ở các phân lớp d là 5, tương ứng với cấu hình electron $1s^22s^22p^63s^23p^63d^5$ $\Rightarrow$ R là nguyên tố Mn (Mangan)}
\end{vdex}





%%%==============Bài tập tự luyện dạng 2==============%%%
\Noibat[][][\faBank]{Bài tập tự luyện dạng \thedang}
%%%=========Câu hỏi trắc nghiệm 1 phương án=========%%%
\phan{Câu hỏi trắc nghiệm 1 phương án}
\Opensolutionfile{ansex}[Ans/LGEX-Hoa10_C01_CTNT_BTTL02]
\Opensolutionfile{ans}[Ans/Ans-Hoa10_C01_CTNT_BTTL02]
%\hienthiloigiaiex
%\tatloigiaiex
\luuloigiaiex
\begin{ex}%[0H1V2-5]
	Một nguyên tử Y có tổng số electron ở phân lớp s là 8. Y là nguyên tố nào?
	\choice
	{Na}
	{Mg}
	{\True Ca}
	{K}
	\loigiai{Vì tổng số electron trên phân lớp s là 8 nên cấu hình electron của Y là $1s^22s^22p^63s^23p^64s^2$ $\Rightarrow$ Y là nguyên tố Ca (Canxi)}
\end{ex}

\begin{ex}%[0H1V2-5]
	Nguyên tử của nguyên tố Z có tổng số electron ở các phân lớp p là 15. Z là nguyên tố nào?
	\choice
	{\True As}
	{P}
	{S}
	{Br}
	\loigiai{Tổng số electron ở các phân lớp p là 15, tương ứng với cấu hình $1s^22s^22p^63s^23p^63d^{10}4s^24p^3$ $\Rightarrow$ Z là nguyên tố As (Asen)}
\end{ex}

\begin{ex}%[0H1V2-5]
	Nguyên tử của nguyên tố M có số electron ở phân lớp ngoài cùng là 6. M là nguyên tố nào?
	\choice
	{N}
	{\True O}
	{C}
	{S}
	\loigiai{Số electron ở phân lớp ngoài cùng là 6, tương ứng với cấu hình electron $1s^22s^22p^4$ $\Rightarrow$ M là nguyên tố O (Oxi)}
\end{ex}

\begin{ex}%[0H1V2-5]
	Một nguyên tử R có tổng số electron ở các phân lớp d là 5. R là nguyên tố nào?
	\choice
	{Fe}
	{Cu}
	{\True Mn}
	{Zn}
	\loigiai{Tổng số electron ở các phân lớp d là 5, tương ứng với cấu hình electron $1s^22s^22p^63s^23p^63d^5$ $\Rightarrow$ R là nguyên tố Mn (Mangan)}
\end{ex}

\begin{ex}%[0H1V2-5]
	Nguyên tử của nguyên tố T có tổng số electron ở các phân lớp s và p là 18. T là nguyên tố nào?
	\choice
	{P}
	{S}
	{\True Ar}
	{Cl}
	\loigiai{Tổng số electron ở các phân lớp s và p là 18, tương ứng với cấu hình electron $1s^22s^22p^63s^23p^6$ $\Rightarrow$ T là nguyên tố Ar (Argon)}
\end{ex}
\begin{ex}%[0H1V2-5]
	Cấu hình electron của X là $1s^22s^22p^63s^23p^3$. Phát biểu nào sau đây \textbf{sai}.
	\choice
	{\True Lớp M có 8 electron}
	{X thuộc nguyên tố  phi kim}
	{Cấu hìnhn electron theo orbital của X có 3 electron độc thân}
	{X có khả năng nhận thêm 3 electron khi tham gia phản ứng hóa học.}
	\loigiai{%
		\begin{\itemchoice}
			\itemch \textbf{Sai}. Lớp M (n=3) có 5 electron.
			\itemch \textbf{Đúng}. X có 5 electron ở lớp cuối cùng nên nguyên tử nguyên tố  X có tính phi kim.
			\itemch \textbf{Đúng}. Theo quy tắc Hund 3 electron ở phân lớp 3p mỗi electron sẽ chiếm 1 AO trước tiên để số electron độc thân là lớn nhất (3 elctron độc thân).
			\itemch X có 5 electron ở lớp cuối cùng theo quy tắc bát tử sẽ nhận thêm 3 electron để đạt cấu hình bền giống khi hiếm liền sau 
		\end{\itemchoice}
	}
\end{ex}
\begin{ex}%[0H1H2-5]
	Nguyên tố Y có Z=25. Vậy Y thuộc nguyên tố nào?
	\choice
	{s}
	{\True p}
	{d}
	{f}
	\loigiai{%
		Sự phân bố electron của nguyên tử $Y$ theo mức năng lượng là : $1s^22s^22p^63s^23p^63d^54s^2$. Nhận thấy electron có mức năng lượng cao nhất thuộc phân lớp 3d $\Rightarrow$ là nguyên tố d.
	}
\end{ex}
\begin{ex}%[0H1H2-1]
	Cấu hình electron của $Al^{3+}$ là
	\choice
	{$1s^22s^22p^63s^23p^1$}
	{$1s^22s^22p^63s^23p^4$}
	{\True $1s^22s^22p^6$}
	{$1s^22s^22p^63s^23p^6$}
	\loigiai{%
		Cấu hình electron của $Al$ là $Mg(Z=12):1s^22s^22p^63s^23p^1$
		\\
		$Al\xrightarrow Al^{3+}  + 3e$ (nhôm nhường 3e ở lớp ngoài cùng)
		$\Rightarrow$ Cấu hình electron của $Al^{3+}$ là
		$1s^22s^22p^6$
	}
\end{ex}

\Closesolutionfile{ans}
\Closesolutionfile{ansex}
%\bangdapan{Ans-Hoa10_C01_CTNT_BTTL02}

%%%============Câu hỏi đúng sai================%%%
\phan{Câu hỏi trắc nghiệm đúng sai}
%%%%====================%%%
\Opensolutionfile{ansex}[Ans/LGTF-Hoa10_C01_CHET_BTTL02]
\Opensolutionfile{ansbook}[Ansbook/AnsTF-Hoa10_C01_CHE_BTTL02]
\Opensolutionfile{ans}[Ans/Tempt-Hoa10_C01_CHE_BTTL02]
\luulgEXTF
%%%=============EXTF01============%%%
\begin{ex}%[0H1H1-3]
	Đối với cấu hình electron của các nguyên tố, điều nào sau đây là đúng?
	\choiceTF[t]
	{\True Electron được sắp xếp theo nguyên lý vững bền}
	{\True Số electron tối đa trong một phân lớp là $2n^2$, với n là số lớp e}
	{Các orbital luôn được lấp đầy hoàn toàn trước khi chuyển sang orbital tiếp theo}
	{Cấu hình electron của một nguyên tố luôn giống hệt với cấu hình của nguyên tố trước nó cộng thêm một electron}
	\loigiai{
		Về cấu hình electron của các nguyên tố:
		\begin{itemchoice}
			\itemch \textbf{Đúng}. Electron được sắp xếp theo nguyên lý vững bền (nguyên lý Aufbau), điền vào các orbital có mức năng lượng thấp nhất trước.
			\itemch \textbf{Đúng}. Công thức 2n^2 xác định số electron tối đa trong một lớp, với n là số lượng tử chính.
			\itemch \textbf{Sai}. Theo quy tắc Hund, các orbital cùng mức năng lượng sẽ được điền một electron trước khi ghép đôi.
			\itemch \textbf{Sai}. Mặc dù đúng cho nhiều trường hợp, nhưng có ngoại lệ, ví dụ như với các nguyên tố chuyển tiếp.
		\end{itemchoice}
	}
\end{ex}
%%%=============EXTF_02=============%%%
\begin{ex}%[0H1H1-3]
	Đối với cấu hình electron của các nguyên tố, điều nào sau đây là đúng?
	\choiceTF[t]
	{\True Electron được sắp xếp theo nguyên lý vững bền}
	{\True Số electron tối đa trong một lớp là $2n^2$, với n là số lượng tử chính}
	{Các orbital luôn được lấp đầy hoàn toàn trước khi chuyển sang orbital tiếp theo}
	{\True Cấu hình electron tuân theo quy tắc Hund khi điền electron vào các orbital cùng mức năng lượng}
	\loigiai{
		\begin{itemchoice}
			\itemch \textbf{Đúng}. Electron được sắp xếp theo nguyên lý vững bền (nguyên lý Aufbau).
			\itemch \textbf{Đúng}. Công thức $2n^2$ xác định số electron tối đa trong một lớp.
			\itemch \textbf{Sai}. Theo quy tắc Hund, các orbital cùng mức năng lượng sẽ được điền một electron trước khi ghép đôi.
			\itemch \textbf{Đúng}. Quy tắc Hund áp dụng khi điền electron vào các orbital cùng mức năng lượng.
		\end{itemchoice}
	}
\end{ex}
%%%=============EXTF_3=============%%%
\begin{ex}%[0H1H1-3]
	Về phân lớp electron
	\choiceTF[t]
	{\True Phân lớp s có tối đa 2 electron}
	{\True Phân lớp p có tối đa 6 electron}
	{\True Phân lớp d có tối đa 10 electron}
	{Phân lớp f có tối đa 12 electron}
	\loigiai{
		\begin{itemchoice}
			\itemch \textbf{Đúng}. Phân lớp s có 1 orbital, mỗi orbital chứa tối đa 2 electron.
			\itemch \textbf{Đúng}. Phân lớp p có 3 orbital, mỗi orbital chứa tối đa 2 electron, nên tổng cộng là 6.
			\itemch \textbf{Đúng}. Phân lớp d có 5 orbital, mỗi orbital chứa tối đa 2 electron, nên tổng cộng là 10.
			\itemch \textbf{Sai}. Phân lớp f có 7 orbital, mỗi orbital chứa tối đa 2 electron, nên tổng cộng là 14, không phải 12.
		\end{itemchoice}
	}
\end{ex}
%%%=============EXTF_4=============%%%
\begin{ex}%[0H1H1-3]
	Về cấu hình electron của nguyên tử
	\choiceTF[t]
	{\True Số electron trong nguyên tử trung hòa bằng số proton}
	{\True Tổng số electron trong các phân lớp bằng số hiệu nguyên tử}
	{Electron luôn được điền vào orbital có năng lượng thấp nhất trước, không có ngoại lệ}
	{\True Cấu hình electron của ion dương có ít electron hơn nguyên tử trung hòa}
	\loigiai{
		\begin{itemchoice}
			\itemch \textbf{Đúng}. Trong nguyên tử trung hòa, số electron bằng số proton.
			\itemch \textbf{Đúng}. Tổng số electron trong các phân lớp bằng số hiệu nguyên tử (số proton).
			\itemch \textbf{Sai}. Có một số ngoại lệ, ví dụ như Cr và Cu, do sự ổn định của orbital bán đầy hoặc đầy.
			\itemch \textbf{Đúng}. Ion dương hình thành khi nguyên tử mất electron.
		\end{itemchoice}
	}
\end{ex}
%%%=============EXTF_5=============%%%
\begin{ex}%[0H1H1-3]
	Về sự sắp xếp electron trong nguyên tử, điều nào sau đây là chính xác?
	\choiceTF[t]
	{\True Electron được sắp xếp vào các orbital theo thứ tự tăng dần của năng lượng}
	{\True Nguyên lý Pauli quy định rằng mỗi orbital chỉ chứa tối đa 2 electron có spin ngược nhau}
	{\True Quy tắc Hund áp dụng khi điền electron vào các orbital cùng mức năng lượng}
	{Cấu hình electron của một nguyên tố luôn giống hệt cấu hình của nguyên tố trước nó cộng thêm một electron}
	\loigiai{
		\begin{itemchoice}
			\itemch \textbf{Đúng}. Đây là nguyên tắc cơ bản của sự sắp xếp electron.
			\itemch \textbf{Đúng}. Nguyên lý Pauli là một trong những nguyên tắc quan trọng trong cấu hình electron.
			\itemch \textbf{Đúng}. Quy tắc Hund quy định cách điền electron vào các orbital cùng mức năng lượng.
			\itemch \textbf{Sai}. Có những trường hợp ngoại lệ, đặc biệt là ở các nguyên tố chuyển tiếp.
		\end{itemchoice}
	}
\end{ex}
%%%=============EXTF_6=============%%%
\begin{ex}%[0H1V1-3]
	Về cấu hình electron của các nguyên tố chuyển tiếp, điều nào sau đây là đúng?
	\choiceTF[t]
	{\True Electron được thêm vào phân lớp d của lớp trước}
	{\True Có thể có sự chuyển electron từ phân lớp s sang d để đạt trạng thái bền hơn}
	{Tất cả các nguyên tố chuyển tiếp đều có cấu hình electron kết thúc ở phân lớp d}
	{\True Các nguyên tố chuyển tiếp thường có nhiều trạng thái oxi hóa}
	\loigiai{
		\begin{itemchoice}
			\itemch \textbf{Đúng}. Đặc điểm của nguyên tố chuyển tiếp là điền electron vào phân lớp d của lớp trước.
			\itemch \textbf{Đúng}. Ví dụ như Cr: $[Ar]3d^5 4s^1$ thay vì $[Ar]3d^4 4s^2$.
			\itemch \textbf{Sai}. Một số nguyên tố chuyển tiếp có cấu hình kết thúc ở s, ví dụ Cu: [Ar]3d^10 4s^1.
			\itemch \textbf{Đúng}. Do có nhiều electron ở phân lớp d, chúng có thể tạo ra nhiều trạng thái oxi hóa khác nhau.
		\end{itemchoice}
	}
\end{ex}
%%%=============EXTF_7=============%%%
\begin{ex}%[0H1H1-3]
	Về mối quan hệ giữa cấu hình electron và vị trí trong bảng tuần hoàn, điều nào sau đây là chính xác?
	\choiceTF[t]
	{\True Số lớp electron xác định số thứ tự chu kỳ}
	{\True Số electron hóa trị xác định số thứ tự nhóm đối với các nguyên tố nhóm A}
	{Tất cả các nguyên tố có cấu hình electron kết thúc ở $ns^2$ đều là kim loại kiềm thổ}
	{\True Nguyên tố có cấu hình electron kết thúc ở $p^6$ là khí hiếm}
	\loigiai{
		\begin{itemchoice}
			\itemch \textbf{Đúng}. Số lớp electron tương ứng với số thứ tự chu kỳ trong bảng tuần hoàn.
			\itemch \textbf{Đúng}. Đối với nguyên tố nhóm A, số electron hóa trị tương ứng với số thứ tự nhóm.
			\itemch \textbf{Sai}. Không phải tất cả, ví dụ Zn ($[Ar]3d^10 4s^2$) không phải là kim loại kiềm thổ.
			\itemch \textbf{Đúng}. Cấu hình electron kết thúc ở p^6 là đặc trưng của khí hiếm (trừ He: $1s^2$).
		\end{itemchoice}
	}
\end{ex}
%%%=============EXTF_8=============%%%
\begin{ex}%[0H1V1-3]
	Về cấu hình electron của các ion, nhận định nào sau đây là chính xác?
	\choiceTF[t]
	{\True Ion dương có ít electron hơn nguyên tử trung hòa}
	{\True Ion âm có nhiều electron hơn nguyên tử trung hòa}
	{\True Các ion của khí hiếm thường có cấu hình electron giống khí hiếm}
	{Tất cả các ion đều có cấu hình electron ổn định như khí hiếm}
	\loigiai{
		\begin{itemchoice}
			\itemch \textbf{Đúng}. Ion dương hình thành khi nguyên tử mất electron.
			\itemch \textbf{Đúng}. Ion âm hình thành khi nguyên tử nhận thêm electron.
			\itemch \textbf{Đúng}. Nhiều ion tạo thành có xu hướng đạt cấu hình electron ổn định giống khí hiếm.
			\itemch \textbf{Sai}. Không phải tất cả các ion đều có cấu hình electron giống khí hiếm, đặc biệt là các ion của kim loại chuyển tiếp.
		\end{itemchoice}
	}
\end{ex}
%%%=============EXTF_9=============%%%
\begin{ex}%[0H1V2-5]
	Nguyên tố X có cấu hình electron $[Ar]3d^10 4s^2$. Nhận định nào sau đây về X là đúng?
	\choiceTF[t]
	{\True X là nguyên tố thuộc nhóm IIB (nhóm 12)}
	{\True X có số oxi hóa phổ biến là +2}
	{X là một kim loại kiềm thổ}
	{\True X có 2 electron hóa trị}
	\loigiai{
		\begin{itemchoice}
			\itemch \textbf{Đúng}. Cấu hình này tương ứng với Zn, thuộc nhóm IIB (nhóm 12).
			\itemch \textbf{Đúng}. Zn thường có số oxi hóa +2 trong hợp chất.
			\itemch \textbf{Sai}. Zn không phải là kim loại kiềm thổ, mà là kim loại chuyển tiếp.
			\itemch \textbf{Đúng}. Phân lớp 3d của Zn đã bão hòa e do đó electron hóa trị bằng số electron ở phân lớp ngoài cùng 4s.
		\end{itemchoice}
	}
\end{ex}
%%%=============EXTF_10=============%%%
\begin{ex}%[0H1H2-5]
	Nguyên tố Y có cấu hình electron $[Xe]4f^14 5d^10 6s^2 6p^2$. Nhận định nào sau đây về Y là đúng?
	\choiceTF[t]
	{\True Y là nguyên tố thuộc nhóm IVA (nhóm 14)}
	{\True Y nằm ở chu kỳ 6 của bảng tuần hoàn}
	{\True Y có thể tạo hợp chất khí với hidro}
	{Y có tính phi kim mạnh}
	\loigiai{
		\begin{itemchoice}
			\itemch \textbf{Đúng}. Cấu hình này tương ứng với Pb (chì), thuộc nhóm IVA (nhóm 14).
			\itemch \textbf{Đúng}. Pb nằm ở chu kỳ 6 của bảng tuần hoàn.
			\itemch \textbf{Đúng}. Pb có thể tạo hợp chất khí với hidro, ví dụ như PbH4 (plumbane).
			\itemch \textbf{Sai}. Pb không có tính phi kim mạnh. Nó là một kim loại yếu (hoặc á kim), có xu hướng hình thành các hợp chất ion và cộng hóa trị.
		\end{itemchoice}
	}
\end{ex}
\Closesolutionfile{ans}
\Closesolutionfile{ansbook}
\Closesolutionfile{ansex}
%\bangdapanTF{AnsTF-Hoa10_C01_CTNT_BTTL02}
%%%================Phần tự luận================%%%
\phan{Bài tập tự luận}
\Opensolutionfile{ansbth}[Ans/LGBT-Hoa10_C01_CHE_BTTL02]
\Opensolutionfile{ansbt}[Ans/AnsBT-Hoa10_C01_CHE_BTTL02]
\luuloigiaibt
%\hienthiloigiaibt
%%%=============BT_1=============%%%
\begin{bt}%[0H1V2-4]
	Sắt $(Z=26)$ là nguyên tố phổ biến trong vỏ Trái Đất và có nhiều ứng dụng trong công nghiệp. Hãy viết cấu hình electron của nguyên tử sắt và giải thích tại sao nó là một kim loại chuyển tiếp.
	\loigiai{
		Cấu hình electron của sắt $(Z=26)$: $1s^22s^22p^63s^23p^64s^23d^6$
		Sắt là một kim loại chuyển tiếp vì:
		\begin{itemize}
			\item Nó có electron ở orbital d ($3d^6$) chưa được điền đầy.
			\item Cấu hình electron lớp ngoài cùng là $4s^23d^6$ , điều này cho phép sắt có nhiều trạng thái oxi hóa khác nhau.
			\item Sự hiện diện của electron ở orbital d cho phép sắt tạo ra các hợp chất có màu và có tính chất từ tính.
			\item Khả năng tạo phức chất đa dạng do sự tương tác giữa các orbital d với các phối tử.
		\end{itemize}
	}
\end{bt}
%%%============BT02=====================%%%
\begin{bt}%[0H1C2-4]
	Đồng $(Z=29)$ là một kim loại được sử dụng rộng rãi trong dây dẫn điện. Hãy viết cấu hình electron của nguyên tử đồng và giải thích tại sao cấu hình electron của đồng lại khác biệt so với quy luật điền electron thông thường.
	\loigiai{
		Cấu hình electron của đồng $(Z=29)$: $1s^22s^22p^63s^23p^64s^13d^{10}$
		\\
		Cấu hình electron của đồng khác biệt vì:
		\begin{itemize}
			\item Theo quy luật điền electron thông thường, cấu hình dự đoán sẽ là $4s^23d^9$.
			\item Tuy nhiên, cấu hình thực tế là $4s^13d^{10}$, với một electron từ orbital 4s chuyển sang 3d.
			\item Điều này xảy ra vì orbital 3d đầy $(d^{10})$ tạo ra trạng thái năng lượng thấp hơn so với cấu hình $4s^23d^9$.
			\item Hiện tượng này được gọi là "hiệu ứng orbital d nửa đầy hoặc đầy", tạo ra sự ổn định đặc biệt cho nguyên tử.
		\end{itemize}
	}
\end{bt}
%%%============BT03=====================%%%
\begin{bt}%[0H1V2-4]
	Nguyên tố R có $Z=13$ và nguyên tố S có $Z=9$.
	\begin{itemize}
		\item Viết cấu hình electron nguyên tử của nguyên tố R và S.
		\item Khi nguyên tử của nguyên tố R nhường đi ba electron và nguyên tử của nguyên tố S nhận thêm một electron thì lớp electron ngoài cùng của chúng có đặc điểm gì?
	\end{itemize}
	\loigiai{
		\begin{itemize}
			\item Cấu hình electron:
			\begin{itemize}
				\item R ($Z=13$): $1s^2 2s^2 2p^6 3s^2 3p^1$
				\item S ($Z=9$): $1s^2 2s^2 2p^5$
			\end{itemize}
			\item Khi R nhường 3 electron và S nhận 1 electron:
			\begin{itemize}
				\item $\text{R}^{3+}$: $1s^2 2s^2 2p^6$ (như Ne)
				\item $\text{S}^-$: $1s^2 2s^2 2p^6$ (như Ne)
			\end{itemize}
			\item Đặc điểm:
			\begin{itemize}
				\item Cả hai ion đều có 8 electron ở lớp ngoài cùng.
				\item Cả hai đều đạt cấu hình electron bền vững của khí hiếm (Ne).
			\end{itemize}
		\end{itemize}
	}
\end{bt}
%%%==============BT4==============%%%
\begin{bt}%[0H1V2-6]
	Nguyên tử X có tổng số electron ở các phân lớp p là 11. Viết cấu hình electron của X
	\loigiai{
		\begin{itemize}
			\item Tổng số electron ở các phân lớp p là 11, ta có: $2p^6 3p^5$
			\item Các phân lớp s trước đó phải được điền đầy đủ: $1s^2 2s^2 3s^2$
			\item Cấu hình electron đầy đủ: $1s^2 2s^2 2p^6 3s^2 3p^5$
			\item Tổng số electron: $2 + 2 + 6 + 2 + 5 = 17$
			\item Vậy Z = 17, đây là nguyên tố Cl (Clo)
		\end{itemize}
	}
\end{bt}
%%%============BT05=================%%%
\begin{bt}%[0H1V2-6]
	Nguyên tử B có 5 electron ở phân lớp d. Viết cấu hình electron của B và cho biết B là kim loại hay phi kim.
	\loigiai{
		\begin{itemize}
			\item Phân lớp d có 5 electron, ta có: $3d^5$
			\item Các phân lớp trước đó phải được điền đầy đủ: $1s^2 2s^2 2p^6 3s^2 3p^6 4s^2$
			\item Cấu hình electron đầy đủ: $1s^2 2s^2 2p^6 3s^2 3p^63d^5 4s^2$
			\item Tổng số electron: $2 + 2 + 6 + 2 + 6 + 2 + 5 = 25$
			\item Vậy Z = 25, đây là nguyên tố Mn (Mangan)
			\item Mn là một kim loại chuyển tiếp
		\end{itemize}
	}
\end{bt}
%%%============BT06=================%%%
\hienthiloigiaibt
\begin{bt}%[0H1C2-6]
	Một ion $M^{3+}$  có tổng số hạt prôton, nơtron, electron là 79,trong đó số hạt mang điện nhiều hơn số hạt không mang điện là 19 Viết cấu hình e của nguyên tử M
	\loigiai{Ta có 
		\\
		{\renewcommand{\arraystretch}{0.7}
			$\begin{array}{ccccc}
				M&-&3e&\xrightarrow&M^{3+}\\
				\downarrow[3]&&\phantom{XXX}&&\downarrow[3]\\
				(2Z+N)&&&&79
			\end{array}$
		}
		$\Rightarrow$ $\heva{&2Z+N-3=79\\&(2Z-3)-N=19}$ $\Rightarrow$ $\heva{&Z=26\\&N=30}$
		\\[3mm]
		Cấu hình của M: $1s^22s^22p^63s^23p^63d^64s^2$ hoặc viết gọn $[Ar]3d^64s^2$
	}
\end{bt}
%%%=============BT_07=============%%%
\begin{bt}%[0H1V2-4]
	Một ion $X^{2-}$ có tổng số hạt prôton, nơtron, electron là 20, trong đó số hạt không mang điện nhiều hơn số hạt mang điện là 2. Viết cấu hình electron của nguyên tử X.
	\loigiai{Ta có
		\\
		{\renewcommand{\arraystretch}{0.7}
			$
			\begin{array}{ccccc}
				X&+&2e&\xrightarrow&X^{2-}\\
				\downarrow[3]&&\phantom{XXX}&&\downarrow[3]\\
				(2Z+N)&&&&20
			\end{array}
			$
		}
		$\Rightarrow$ $\heva{&2Z+N+2=20\\&N-(2Z+2)=2}$ $\Rightarrow$ $\heva{&Z=8\\&N=8}$
		\\[3mm]
		Cấu hình của X: $1s^22s^22p^4$ (nguyên tử Oxy)
	}
\end{bt}
%%%=============BT_08=============%%%
\begin{bt}%[0H1V2-6]
	Một nguyên tử Y có tổng số hạt prôton, nơtron, electron là 74, trong đó số hạt mang điện nhiều hơn số hạt không mang điện là 18. Viết cấu hình electron của nguyên tử Y.
	\loigiai{Ta có $\heva{&2Z+N=74\\&2Z-N=18}$
		$\Rightarrow$ $\heva{&Z=23\\&N=28}$
		\\[3mm]
		Cấu hình của Y: $1s^22s^22p^63s^23p^63d^34s^2$ (nguyên tử Vanadi)
	}
\end{bt}
%%%=============BT_09=============%%%
\begin{bt}%[0H1C2-6]
	Một ion $Z^+$ có tổng số hạt prôton, nơtron, electron là 33, trong đó số hạt mang điện ít hơn số hạt không mang điện là 1. Viết cấu hình electron của nguyên tử Z.
	\loigiai{Ta có
		\\
		{\renewcommand{\arraystretch}{0.7}
			$
			\begin{array}{ccccc}
				Z&-&1e&\xrightarrow&Z^+\\
				\downarrow[3]&&\phantom{XXX}&&\downarrow[3]\\
				(2Z+N)&&&&33
			\end{array}
			$
		}
		$\Rightarrow$ $\heva{&2Z+N-1=33\\&N-(2Z-1)=1}$ $\Rightarrow$ $\heva{&Z=11\\&N=12}$
		\\[3mm]
		Cấu hình của Z: $1s^22s^22p^63s^1$ (nguyên tử Natri)
	}
\end{bt}
%%%=============BT_10=============%%%
\begin{bt}%[0H1V2-4]
	Một nguyên tử W có tổng số hạt prôton, nơtron, electron là 24, trong đó số hạt mang điện nhiều hơn số hạt không mang điện là 2. Viết cấu hình electron của nguyên tử W.
	\loigiai{Ta có $\heva{&2Z+N=24\\&2Z-N=2}$
		$\Rightarrow$ $\heva{&Z=8\\&N=8}$
		\\[3mm]
		Cấu hình của W: $1s^22s^22p^4$ (nguyên tử Oxy)
	}
\end{bt}
%%%=============BT_11=============%%%
\begin{bt}%[0H1C2-6]
	Một ion $M^{3+}$ có tổng số hạt prôton, nơtron, electron là 79, trong đó số hạt mang điện nhiều hơn số hạt không mang điện là 19. Viết cấu hình electron của nguyên tử M.
	\loigiai{Ta có
		\\
		{\renewcommand{\arraystretch}{0.7}
			$
			\begin{array}{ccccc}
				M&-&3e&\xrightarrow&M^{3+}\\
				\downarrow[3]&&\phantom{XXX}&&\downarrow[3]\\
				(2Z+N)&&&&79
			\end{array}
			$
		}
		$\Rightarrow$ $\heva{&2Z+N-3=79\\&(2Z-3)-N=19}$ $\Rightarrow$ $\heva{&Z=26\\&N=30}$
		\\[3mm]
		Cấu hình của M: $1s^22s^22p^63s^23p^63d^64s^2$ (nguyên tử Sắt)
	}
\end{bt}
\Closesolutionfile{ansbt}
\Closesolutionfile{ansbth}
%\bangdapanSA{AnsBT-Hoa10_C01_CTNT_BTTL02}