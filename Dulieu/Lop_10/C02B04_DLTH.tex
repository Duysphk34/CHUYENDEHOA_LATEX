\section[Định luật tuần hoàn]{Định luật tuần hoàn. Ý nghĩa của bảng tuần hoàn các nguyên~tố~hóa~học}
\vspace{1.25cm}
\begin{Muctieu}
	\begin{itemize}
		\item  Phát biểu được định luật tuần hoàn.
		\item  Trình bày được ý nghĩa của bảng tuần hoàn các nguyên tố hoá học: Mối liên hệ giữa vị trí (trong bảng tuần hoàn các nguyên tố hoá học) với tính chất và ngược lại.
	\end{itemize}
\end{Muctieu}
\begin{kd}
	Định luật tuần hoàn đóng vai trò như thế nào trong việc dự đoán tính chất các chất
\end{kd}
\subsection{Nội dung bài học}
	\subsubsection{Định luật tuần hoàn}
	\vspace{0.25 cm}
	\begin{tomtat}
		\lq\lq Tính chất của các nguyên tố và đơn chất, cũng như thành phần và tính chất của các hợp chất tạo nên từ các nguyên tố đó biến đồi tuần hoàn theo chiều tăng của điện tích hạt nhân nguyên tử \rq\rq.
	\end{tomtat}
	\subsubsection{Ý nghĩa của bảng tuần hoàn}
