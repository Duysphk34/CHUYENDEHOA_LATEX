\def\x{210}
\setcounter{bt}{0}
\setcounter{ex}{0}
%%%Tùy chọn 1: Kì thi(có thể bỏ trống)
%%%Tùy chọn 3: lớp (có thể bỏ trống)
%%%Tùy chọn 4: Sở/Phòng (có thể bỏ trống)
%%%Tùy chọn 5: Ngày thi (không được bỏ trống)
\begin{name}[][][][]{Trường THCS }{2023 - 2024}
\end{name}
\sttde{1}
\Opensolutionfile{ansbt}[LOIGIAITN/KTGIUAKI1/LGTN_DE1]
\Opensolutionfile{ans}[Ans/KTGIUAKI1/DAPAN_DE1]
%%%============EX_1==============%%%
\begin{ex}
	Đối tượng nào sau đây là đối tượng nghiên cứu của hóa học?
	\choice{Sự quay của Trái Đất}
	{Sự sinh trưởng và phát triển của thực vật}
	{Chất và sự biến đổi về chất}
	{Tác dụng của thuốc với cơ thể người}
\end{ex}
%%%============EX_2==============%%%
\begin{ex}
	Cho các phương pháp: lý thuyết, thực hành, vẽ hình họa, mỹ thuật. Có bao nhiêu phương pháp được sử dụng để học tập hóa học?
	\choice{1}
	{2}
	{3}
	{4}
\end{ex}
%%%============EX_3==============%%%
\begin{ex}
	Ngành nào sau đây không liên quan đến hóa học?
	\choice{Mĩ phẩm}
	{Năng lượng}
	{Dược phẩm}
	{Vũ trụ}
\end{ex}
%%%============EX_4==============%%%
\begin{ex}
	Trong hạt nhân nguyên tử có chứa những loại hạt nào?
	\choice{proton, neutron}
	{electron, neutron}
	{electron, proton}
	{proton, neutron, electron}
\end{ex}
%%%============EX_5==============%%%
\begin{ex}
	Hạt nào sau đây mang điện tích âm?
	\choice{Proton}
	{Hạt nhân}
	{Electron}
	{Neutron}
\end{ex}
%%%============EX_6==============%%%
\begin{ex}
	Khối lượng của một proton bằng
	\choice{0,00055 amu}
	{0,1 amu}
	{$1 \mathrm{amu}$}
	{0,0055 amu}
\end{ex}
%%%============EX_7==============%%%
\begin{ex}
	Nguyên tố hóa học là những nguyên tử có cùng
	\choice{số neutron}
	{nguyên tử khối}
	{số khổi}
	{số proton}
\end{ex}
%%%============EX_8==============%%%
\begin{ex}
	Số hiệu nguyên tử (Z) của nguyên tố hóa học không bằng giá trị nào sau đây?
	\choice{Số hạt proton}
	{Số hạt electron}
	{Số điện tích dương}
	{Số hạt neutron}
\end{ex}
%%%============EX_9==============%%%
\begin{ex}
	Đồng vị là những nguyên tử có
	\choice{cùng số proton, khác số neutron}
	{cùng số neutron}
	{cùng số khối}
	{cùng số proton, cùng số neutron}
\end{ex}
%%%============EX_10==============%%%
\begin{ex}
	Lớp Kcó mấy phân lớp?
	\choice{1}
	{3}
	{5}
	{7}
\end{ex}
%%%============EX_11==============%%%
\begin{ex}
	Số electron tối đa trong lớp Mlà bao nhiêu?
	\choice{2}
	{8}
	{32}
	{18}
\end{ex}
%%%============EX_12==============%%%
\begin{ex}
	Phân lớp nào sau đây kí hiệu sai?
	\choice{$1 \mathrm{~s}$ }
	{$3 p$}
	{$3 \mathrm{~d}$}
	{$2 \mathrm{~d}$}
\end{ex}
%%%============EX_13==============%%%
\begin{ex}
	Sự phóng xạ là quá trình xảy ra do yếu tố nào?
	\choice{Sự tác động của bên ngoài}
	{Sự tác động của con người}
	{Sự tự phát}
	{Do từ trường trái đất}
\end{ex}
%%%============EX_14==============%%%
\begin{ex}
	Trong bảng tuần hoàn, số thứ tự của ô nguyên tố không được tính bằng
	\choice{số proton}
	{số electron}
	{số hiệu nguyên tử}
	{số khối}
\end{ex}
%%%============EX_15==============%%%
\begin{ex}
	Cho biết, khối lượng của một proton bằng $1 \mathrm{amu}$ của một electron bằng $0,00055 \mathrm{amu}$ Tỉ lệ về khối lượng giữa hạt proton và hạt electron có giá trị bằng khoảng
	\choice{181,8}
	{1818}
	{18,18}
	{1,818}
\end{ex}
%%%============EX_16==============%%%
\begin{ex}
	Kích thước hạt nhân so với kích thước nguyên tử bằng khoảng bao nhiêu lần?
	\choice{$10^6$ lần}
	{$10^7$ lần}
	{$10^{-4}-10^{-3}$ lần}
	{$10^{-5}-10^{-4}$ lần}
\end{ex}
%%%============EX_17==============%%%
\begin{ex}
	Một nguyên tử có chứa 8 proton trong hạt nhân. Số hiệu nguyên tử của nguyên tử này là
	\choice{8}
	{9}
	{16}
	{4}
\end{ex}
%%%============EX_18==============%%%
\begin{ex}
	Nguyên tử $X$ có chứa 7 proton và 8 neutron. Kí hiệu nguyên tử của $X$ là
	\choice{$_7^8 X$}
	{$_7^{15} X$}
	{$_8^7 X$}
	{$_{15}^7 X$}
\end{ex}
%%%============EX_19==============%%%
\begin{ex}
	Cặp nguyên tử nào sau đây là đồng vị của nhau?
	\choice{$_6^{12} X,_5^{10} Y$}
	{$_1^1 M,_2^4 G$}
	{$_8^{16} D,_8^{17} E$}
	{$_9^{17} \mathrm{~L},_1^3 \mathrm{~T}$}
\end{ex}
%%%============EX_20==============%%%
\begin{ex}
	Cho các nguyên tử với các giá trị trong bảng sau:\par
	\begin{center}
		\begin{tabular}{lllll}
		\hline Nguyên tử & X& Y& G& T\\
		\hline Tổng hạt $(p, n, e)$ & 82 & 24 & 40 & 26 \\
		\hline Số khối & 56 & 16 & 27 & 18 \\
		\hline
		\end{tabular}\par
	\end{center}
	Những nguyên từ nào là đồng vị của nhau?
	\choice{Xvà $Y$}
	{$Y$ và $G$}
	{Gvà T}
	{Yvà T}
\end{ex}
%%%============EX_21==============%%%
\begin{ex}
	Electron chuyển từ lớp gần hạt nhân ra lớp xa hạt nhân thì sẽ
	\choice{thu năng lượng}
	{giải phóng năng lượng}
	{không thay đổi năng lượng}
	{vừa thu vừa giải phóng năng lượng}
\end{ex}
%%%============EX_22==============%%%
\begin{ex}
	Theo em, xác suất tìm thấy electron trong toàn phần không gian bên ngoài đám mây là khoảng bao nhiêu phần trăm?
	\choice{$0 \%$}
	{$100 \%$}
	{khoảng $90 \%$}
	{khoảng $50 \%$}
\end{ex}
%%%============EX_23==============%%%
\begin{ex}
	Kí hiệu cấu hình electron nào sau đây viết sai?
	\choice{$2 s^2$}
	{$3 p^5$}
	{$1 \mathrm{~s}^3$}
	{$3 d^2$}
\end{ex}
%%%============EX_24==============%%%
\begin{ex}
	Cấu hình electron nào sau đây là của nguyên tử Oxygen $(Z=8)$ ?
	\choice{$1 s^2 2 s^3 2 p^3$}
	{$1 s^2 2 s^4 2 p^2$}
	{$1 s^2 2 s^1 2 p^5$}
	{$1 s^2 2 s^2 2 p^4$}
\end{ex}
%%%============EX_25==============%%%
\begin{ex}
	Cho các cấu hình electron sau:
	(1) $1 s^2$
	(2) $1 s^2 2 s^2 2 p^3$
	(3) $1 s^2 2 s^2 2 p^6$
	(4) $1 s^2 2 s^2 2 p^6 3 s^2 3 p^1$
	(5) $1 s^2 2 s^2 2 p^6 3 s^2$
	(6) $1 s^2 2 s^2 2 p^6 3 s^2 3 p^6 4 s^1$
	Có bao nhiêu cấu hình electron trong các cấu hình cho trên là của nguyên tử kim loại?
	\choice{2}
	{3}
	{4}
	{5}
\end{ex}
\Closesolutionfile{ans}
\Closesolutionfile{ansbt}
\label{\x}