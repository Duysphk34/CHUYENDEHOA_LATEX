\begin{dang}{Lý thuyết về cấu tạo nguyên tử}
	\Noibat[][][\faCamera]{Phương pháp giải}
	\begin{itemize}
		\item Nắm vững về cấu tạo nguyên tử
		\item Nắm vững kết quả thí nghiệm của Thomson,Rutherford
	\end{itemize}
\end{dang}
\Noibat[\maunhan][][\faBookmark]{Ví dụ mẫu}
%%%%============VDEX1==============%%%
\begin{vdex}
	Các hạt cơ bản của hầu hết các nguyên tử là?
	\choice
	{electron}
	{electron và proton}
	{proton và notron}
	{\True electron, proton và notron}
	\loigiai{}
\end{vdex}
%%%%============VDEX2==============%%%
\begin{vdex}
	Hạt nhân của hầu hết các nguyên tử gồm có?
	\choice
	{electron}
	{electron và proton}
	{\True proton và notron}
	{electron, proton và notron}
	\loigiai{}
\end{vdex}
%%%%============VDEX3==============%%%
\begin{vdex}
	Trong thí nghiệm của Thomson, phát biểu nào sau đây sai với kết quả thí nghiệm ta quan sát được?
	\choice
	{Tia âm cực là các chùm hạt electron di chuyển từ cực âm sang cực dương}
	{Tia âm cực là chùm hạt mang điện tích âm}
	{\True	Tia âm cực bị lệch về phía bản cực âm của nguồn điện}
	{Tia âm cực bị lệch hướng khi ta đặt nó trong từ trường}
	\loigiai{}
\end{vdex}

%%%%============Bài tập tự luyện dạng 1==============%%%
\Noibat[][][\faBank]{Bài tập tự luyện dạng \thedang}
%%%=========Câu hỏi trắc nghiệm 1 phương án=========%%%
\phan{Câu hỏi trắc nghiệm 1 phương án}
\Opensolutionfile{ansex}[Ans/LGEX-Hoa10_C01B01_CTNT_BTTL01]
\Opensolutionfile{ans}[Ans/Ans-Hoa10_C01B01_CTNT_BTTL01]
%\hienthiloigiaiex
%\tatloigiaiex
%\luuloigiaiex
%%%=============EX_1=============%%%
\begin{ex}%[0H1N1-1]
	Hạt mang điện dương trong hạt nhân nguyên tử là
	\choice
	{Electron}
	{\True Proton}
	{Neutron}
	{Photon}
	\loigiai{
		Proton là hạt mang điện dương trong hạt nhân nguyên tử. Electron mang điện âm, neutron không mang điện.
	}
\end{ex}

%%%=============EX_2=============%%%
\begin{ex}%[0H1N1-1]
	Số proton trong hạt nhân nguyên tử được gọi là
	\choice
	{Số khối}
	{Số neutron}
	{\True Số hiệu nguyên tử}
	{Số electron}
	\loigiai{
		Số hiệu nguyên tử (kí hiệu Z) chính là số proton trong hạt nhân nguyên tử. Nó xác định danh tính của nguyên tố hóa học.
	}
\end{ex}

%%%=============EX_3=============%%%
\begin{ex}%[0H1H1-1]
	Nguyên tử trung hòa về điện có số
	\choice
	{Proton lớn hơn số electron}
	{Electron lớn hơn số proton}
	{\True Proton bằng số electron}
	{Neutron bằng số proton}
	\loigiai{
		Trong nguyên tử trung hòa về điện, số proton (mang điện dương) bằng số electron (mang điện âm), do đó tổng điện tích bằng 0.
	}
\end{ex}

%%%=============EX_4=============%%%
\begin{ex}%[0H1H1-1]
	Số khối A của một nguyên tử được tính bằng
	\choice
	{Số proton - số neutron}
	{Số electron + số neutron}
	{\True Số proton + số neutron}
	{Số proton + số electron}
	\loigiai{
		Số khối A = số proton + số neutron. Đây là tổng số hạt trong hạt nhân nguyên tử.
	}
\end{ex}

%%%=============EX_5=============%%%
\begin{ex}%[0H1H1-1]
	Nguyên tử ${^{23}_{11}Na}$ có bao nhiêu neutron?
	\choice
	{11}
	{\True 12}
	{23}
	{34}
	\loigiai{
		Trong ${^{23}_{11}Na}$, số khối $A = 23$, số hiệu nguyên tử $Z = 11$.
		Số neutron $= A - Z = 23 - 11 = 12$
	}
\end{ex}

%%%=============EX_6=============%%%
\begin{ex}%[0H1N1-1]
	Đồng vị là các nguyên tử có cùng
	\choice
	{Số khối}
	{Số neutron}
	{\True Số proton}
	{Số electron}
	\loigiai{
		Đồng vị là các nguyên tử có cùng số proton (số nguyên tử) nhưng khác nhau về số neutron (và do đó khác nhau về số khối).
	}
\end{ex}

%%%=============EX_7=============%%%
\begin{ex}%[0H1H1-1]
	Hạt nào sau đây không có trong hạt nhân nguyên tử?
	\choice
	{Proton}
	{Neutron}
	{\True Electron}
	{Proton và neutron}
	\loigiai{
		Hạt nhân nguyên tử chỉ chứa proton và neutron. Electron quay xung quanh hạt nhân trong lớp vỏ nguyên tử.
	}
\end{ex}

%%%=============EX_8=============%%%
\begin{ex}%[0H1H2-1]
	Trong một nguyên tử, nếu số proton là 8 và số neutron là 9, số khối của nguyên tử đó là bao nhiêu?
	\choice
	{8}
	{9}
	{16}
	{\True 17}
	\loigiai{
		Số khối $A = \text{số proton} + \text{số neutron}$
		Trong trường hợp này: $A = 8 + 9 = 17$
	}
\end{ex}

%%%=============EX_9=============%%%
\begin{ex}%[0H1H1-1]
	Đơn vị nào thường được sử dụng để đo kích thước của nguyên tử?
	\choice
	{Milimet ($mm$)}
	{\True Picomet ($pm$)}
	{decimet ($dm$)}
	{centimet ($cm$)}
	\loigiai{
		Kích thước nguyên tử thường được đo bằng đơn vị picomet (pm). $1 pm = 10^-12 m$, phù hợp với kích thước cực nhỏ của nguyên tử.
	}
\end{ex}

%%%=============EX_10=============%%%
\begin{ex}%[0H1N1-1]
	Hạt nào sau đây có khối lượng gần bằng khối lượng của proton?
	\choice
	{Electron}
	{\True Neutron}
	{Positron}
	{Alpha}
	\loigiai{
		Neutron có khối lượng gần bằng khối lượng của proton. Electron có khối lượng rất nhỏ so với proton. Positron là phản hạt của electron. Hạt alpha là hạt gồm 2 proton và 2 neutron.
	}
\end{ex}
%%%=============EX_11=============%%%
\begin{ex}%[0H1H1-1]
	Theo mô hình bánh pudding mận của Thomson, phát biểu nào sau đây là đúng?
	\choice
	{%
		Nguyên tử có cấu tạo rỗng gồm hạt nhân mang điện tích dương và vỏ là các electron chuyển động xung quanh hạt nhân.
	}
	{%
		Nguyên tử có cấu tạo rỗng gồm hạt nhân mang điện tích dương và vỏ là các electron chuyển dộng xung quanh hạt nhân theo những quỹ đạo có kích thước và năng lượng cố định
	}
	{%
		\True	nguyên tử bao gồm các electron nằm rải rác trong một đám mây hình cầu mang điện tích dương.
	}
	{%
		các electron  quay quanh hạt nhân không theo một quỹ đạo xác định, mà chúng tạo thành các đám mây điện tích mà tại đó xác suất tìm thấy electron là lớn nhất
	}
	\loigiai{
		Theo mô hình bánh pudding mận của Thomson nguyên tử bao gồm các electron nằm rải rác trong một đám mây hình cầu mang điện tích dương
	}
\end{ex}

%%%=============EX_12=============%%%
\begin{ex}%[0H1H1-1]
	Cho các phát biểu sau:
	\begin{enumerate}[(1)]
		\item Tất cả các hạt nhân nguyên tử đều được cấu tạo từ các hạt proton và neutron.
		\item Khối lượng nguyên tử tập trung phần lớn ở lớp vỏ.
		\item Trong nguyên tử, số electron bằng số proton.
		\item Trong hạt nhân nguyên tử, hạt mang điện là proton và electron.
		\item Trong nguyên tử, hạt electron có khối lượng không đáng kể so với các hạt còn lại.
	\end{enumerate}
	Số phát biểu đúng là
	\choice
	{%
		1
	}
	{%
		\True 2
	}
	{%
		3
	}
	{%
		4
	}
	\loigiai{%
		Phát biểu đúng là:
		Trong hạt nhân nguyên tử, hạt mang điện là proton và electron.\\
		Trong nguyên tử, hạt electron có khối lượng không đáng kể so với các hạt còn lại
	}
\end{ex}

%%%=============EX_13=============%%%
\begin{ex}%[0H1V1-1]
	Điều nào sau đây đúng theo mô hình nguyên tử của Thomson?
	\choice
	{%
		Nguyên tử không trung hòa về điện
	}
	{%
		\True Nguyên tử là quả cầu mang điện tích dương có chứa các electron bên trong
	}
	{%
		Điện tích âm và điện tích dương trong nguyên tử có độ lớn bằng nhau
	}
	{%
		Không có điều nào ở trên
	}
	\loigiai{
		Nguyên tử là quả cầu mang điện tích dương có chứa các electron bên trong
	}
\end{ex}

%%%=============EX_14=============%%%
\begin{ex}%[0H1V1-1]
	Trong hiện tượng xả điện qua khí ở áp suất thấp, sự tỏa sáng màu trong ống xuất hiện là kết quả của:
	\choice
	{%
		\True va chạm giữa các hạt mang điện được phát ra từ cực âm và nguyên tử của khí
	}
	{%
		va chạm giữa các electron khác nhau của các nguyên tử trong khí
	}
	{%
		kích thích các electron trong các nguyên tử
	}
	{%
		va chạm giữa các nguyên tử của khí
	}
	\loigiai{
		sự tỏa sáng màu trong ống xuất hiện là kết quả của va chạm giữa các hạt mang điện được phát ra từ cực âm và nguyên tử của khí
	}
\end{ex}

%%%=============EX_15=============%%%
\begin{ex}%[0H1N1-1]
	Mô hình đầu tiên về nguyên tử được đưa ra bởi:
	\choice
	{%
		N. Bohr
	}
	{%
		E. Goldstein
	}
	{%
		Rutherford
	}
	{%
		\True J.J. Thomson
	}
	\loigiai{
		Mô hình nguyên tử đầu tiên được đưa ra bởi JJ Thomson. Theo ông, nguyên tử bao gồm một quả cầu mang điện tích dương với các electron mang điện tích âm được nhúng trong đó.
	}
\end{ex}

%%%=============EX_16=============%%%
\begin{ex}%[0H1V1-1]
	Nếu đường kính của nguyên tử khoảng $10^2 \mathrm{pm}$ thì đường kính của hạt nhân khoảng
	\choice
	{%
		$10^2 \mathrm{pm}$
	}
	{%
		$10^{-4} \mathrm{pm}$
	}
	{%
		\True	$10^{-2} \mathrm{pm}$
	}
	{
		$10^4 \mathrm{pm}$
	}
	\loigiai{
		Ta có $\dfrac{d_{NT}}{d_{\text{hạt nhân}}} =10^4$ (lần)
		$\Rightarrow d_{\text{hạt nhân}} = \dfrac{d_{NT}}{10^4}=\dfrac{10^2}{10^4}=10^{-2}$ (pm)
	}
\end{ex}
\Closesolutionfile{ans}
\Closesolutionfile{ansex}
%\bangdapan{Ans-Hoa10_C01B01_CTNT_BTTL01}

%%%===========Bài tập tự luyện dạng tự luận dạng 1=============%%%%
\phan{BÀI TẬP TỰ LUẬN}
\Opensolutionfile{ansbth}[Ans/LGBT-Hoa10_C01B01_CTNT_BTTL01]
\Opensolutionfile{ansbt}[Ans/AnsBT-Hoa10_C01B01_CTNT_BTTL01]
%\luuloigiaibt
%%%=============BT_1=============%%%
\begin{bt}%[0H1V1-1]
	Trong thí nghiệm của Rutherford, khi sử dụng các hạt alpha (ion $\mathrm{He}^{2+}$, kí hiệu là $\mathrm{a}$ ) bắn vào lá vàng thì:
	\begin{itemize}
		\item Hầu hết các hạt a xuyên thẳng qua lá vàng.
		\item Một số ít hạt a bị lệch quỹ đạo so với ban đầu.
		\item Một số rất ít hạt a bị bật ngược trở lại.
	\end{itemize}
	Từ kết quả này, em có nhận xét gì về cấu tạo nguyên tử?
	\loigiai{
		Trong thí nghiệm của Rutherford, khi sử dụng các hạt alpha (ion $\mathrm{He}^{2+}$, kí hiệu là a) bắn vào lá vàng thì:
		\begin{itemize}
			\item Hầu hết các hạt a xuyên thẳng qua lá vàng chứng tỏ nguyên tử có cấu tạo rỗng.
			\item Một số ít hạt a bị lệch quỹ đạo so với ban đầu chứng tỏ hạt nhân nguyên tử cùng điện tích dương như hạt hạt alpha (ion $\mathrm{He}^{2+}$, kí hiệu là $ \alpha $).
			\item Một số rất ít hạt a bị bật ngược trở lại chứng tỏ kích thước hạt nhân nhỏ hơn rất nhiều so với kích thước của nguyên tử và khối lượng nguyên tử tập trung chủ yếu ở hạt nhân.
		\end{itemize}
	}
\end{bt}
%%%=============BT_2=============%%%
\begin{bt}%[0H1V1-1]
	Viết lại bảng sau vào vở và điền thông tin còn thiếu vào các ô trống:\par\noindent
	\begin{longtable}{|c|c|c|c|c|c|c|}
		\hline \indam{Nguyên tố} & \indam{Kí hiệu} &  \indam{Z} & \indam{Số e} & \indam{Số p} & \indam{Số n} & \indam{Số khối} \\
		\endfirsthead
		\hline \indam{Nguyên tố} & \indam{Kí hiệu} &  \indam{Z} & \indam{Số e} & \indam{Số p} & \indam{Số n} & \indam{Số khối} \\
		\endhead
		\hline \indam{Carbon} & $\mathrm{C}$ & 6 & 6 & $?$ & 6 & $?$ \\
		\hline \indam{Nitrogen} & $\mathrm{N}$ & 7 & $?$ & 7 & $?$ & 14 \\
		\hline \indam{Oxygen} & $\mathrm{O}$ & 8 & 8 & $?$ & 8 & $?$ \\
		\hline \indam{Sodium (natri)} & $\mathrm{Na}$ & 11 & $?$ & 11 & $?$ & 23 \\
		\hline \indam{Aluminium (nhôm)} & $\mathrm{Al}$ & $?$ & 13 & $?$ & $?$ & 27 \\
		\hline
	\end{longtable}
	\loigiai{%
		\begin{longtable}{|c|c|c|c|c|c|c|}
			\hline\rowcolor{\mycolor!10}
			\indam{Nguyên tố} & \indam{Kí hiệu} &  \indam{Z} & \indam{Số e} & \indam{Số p} & \indam{Số n} & \indam{Số khối} \\
			\hline 
			\endfirsthead
			\hline\rowcolor{\mycolor!10}
			\indam{Nguyên tố} & \indam{Kí hiệu} &  {$\mathbf{Z}$} & \indam{Số e} & \indam{Số p} & \indam{Số n} & \indam{Số khối} \\
			\hline \endhead
			\hline\endfoot
			\indam{Carbon} & $\mathrm{C}$ & 6 & 6 & 6 & 6 & 12 \\
			\hline 
			\indam{Nitrogen} & $\mathrm{N}$ & 7 & 7 & 7 & 7 & 14 \\
			\hline 
			\indam{Oxygen} & $\mathrm{O}$ & 8 & 8 & 8 & 8 & 16 \\
			\hline 
			\indam{Sodium (natri)} & $\mathrm{Na}$ & 11 & 11 & 11 & 12 & 23\\
			\hline 
			\indam{Aluminium (nhôm)} & $\mathrm{Al}$ & 13 & 13 & 13 & 14 &27\\
			\hline
		\end{longtable}
	}
\end{bt}
%%%=============BT_3=============%%%
\begin{bt}%[0H1H1-1]
	Nối tên các nhà khoa học ở cột A với những đóng góp của họ trong việc tìm hiểu cấu trúc nguyên tử ở cột B
	\par\noindent
	\begin{longtable}{|p{0.3\linewidth}|p{0.4\linewidth}|}
		% Nội dung cho đầu trang đầu tiên
		\hline\rowcolor{\mycolor!10}
		Cột A & Cột B \\
		\hline
		\endfirsthead
		% Nội dung cho đầu các trang tiếp theo
		\hline\rowcolor{\mycolor!10}
		Cột A & Cột B \\
		\hline
		\endhead
		% Nội dung cho chân trang (trừ trang cuối)
		\hline
		\endfoot
		% Nội dung cho chân trang cuối cùng
		\hline
		\endlastfoot
		% Nội dung chính của bảng bắt đầu từ đây
		(a) Ernest Rutherford & (i) Tính không thể phân chia của nguyên tử\\
		(b) J.J.Thomson &(ii) Các quỹ đạo dừng\\
		(c) Dalton &(iii) Khái niệm hạt nhân\\
		(d) Neils Bohr &(iv) Phát hiện electron\\
		(e) James Chadwick &(v) Số nguyên tử\\
		(f) E. Goldstein &(vi) Nơtron\\
		(g) Mosley &(vii) Tia âm cực\\
		\hline
	\end{longtable}
	\loigiai{%
		\noindent
		(a) Ernest Rutherford -- (iii) Khái niệm hạt nhân\\
		(b) J.J.Thomson -- (iv) Phát hiện electron\\
		(c) Dalton -- (i) Tính không thể phân chia của nguyên tử\\
		(d) Neils Bohr -- (ii) Các quỹ đạo dừng\\
		(e) James Chadwick -- (vi) Nơtron\\
		(f) E. Goldstein -- (vii) Tia âm cực\\
		(g) Mosley -- (v) Số nguyên tử
	}
\end{bt}
%%%==============BT4==============%%%
\begin{bt}%[0H1V1-1]
	Một loại nguyên tử nitrogen có 7 proton và 7 neutron trong hạt nhân. Dựa vào Bảng \ref{tab:ktntklnt}, hãy tính và so sánh:
	\begin{enumerate}
		\item Khối lượng hạt nhân với khối lượng nguyên tử.
		\item Khối lượng hạt nhân với khối lượng vỏ nguyên tử.
	\end{enumerate}
	\loigiai{Từ bảng \ref{tab:ktntklnt}, ta có:\\
		$m(\text{proton}) = m(\text{neutron}) \approx 1,67 \cdot 10^{-27} \text{ kg} $, $m(\text{electron}) = 9,109 \cdot 10^{-31} \text{ kg}$
		\\
		Nguyên tử nitrogen có 7 proton, 7 neutron và 7 electron (vì là nguyên tử trung hòa).
		
		\begin{enumerate}
			\item So sánh khối lượng hạt nhân với khối lượng nguyên tử:
			\begin{align*}
				\text{Khối lượng hạt nhân} &= 7m(\text{proton}) + 7m(\text{neutron}) \\
				&= 7 \cdot 1,67 \cdot 10^{-27} + 7 \cdot 1,67 \cdot 10^{-27} \\
				&= 23,38 \cdot 10^{-27} \text{ kg}
			\end{align*}
			\begin{align*}
				\text{Khối lượng nguyên tử} &= \text{khối lượng hạt nhân} + \text{khối lượng 7 electron} \\
				&= 23,38 \cdot 10^{-27} + 7 \cdot 9,109 \cdot 10^{-31} \\
				&= 23,38 \cdot 10^{-27} + 0,064 \cdot 10^{-27} \\
				&= 23,444 \cdot 10^{-27} \text{ kg}
			\end{align*}
			
			Ta thấy khối lượng hạt nhân chiếm 99,73\% khối lượng nguyên tử.
			\item So sánh khối lượng hạt nhân với khối lượng vỏ nguyên tử:
			\begin{align*}
				\text{Khối lượng vỏ nguyên tử} &= \text{khối lượng 7 electron} \\
				&= 7 \cdot 9,109 \cdot 10^{-31} = 0,064 \cdot 10^{-27} \text{ kg}
			\end{align*}
			\begin{align*}
				\text{Tỉ lệ khối lượng hạt nhân} : \text{khối lượng vỏ}	&= 23,38 \cdot 10^{-27} : 0,064 \cdot 10^{-27} \\
				&\approx 365{,}3 : 1
			\end{align*}
			Vậy khối lượng hạt nhân gấp khoảng $365{,}3$ lần khối lượng vỏ nguyên tử.
		\end{enumerate}
	}
\end{bt}

\Closesolutionfile{ansbt}
\Opensolutionfile{ansbth}

%%%%%====================Dạng 2=====================%%%
\newpage
\begin{dang}{Bài tập về khối lượng, kích thước nguyên tử}	
	\Noibat[][][\faCamera]{Phương pháp giải}
	\tieumuc{Các công thức liên quan khối lượng}
	\begin{itemize}
		\item $ m _{\text{nguyên tử}=m_{p}+m_{n} + m_{e} } $ (tính chính xác); $ m _{\text{nguyên tử}} \approx  m_{p} + m_{n} \approx m_{\text{hạt nhân}} $ (tính gần đúng)
		\item Khối lượng tính ra kg của 1 nguyên tử carbon-12 là $ 19,926 . 10^{27}~\mathrm{kg}$.
		\item 1 amu được định nghĩa bằng $\dfrac{1}{12}$ khối lượng 1 nguyên tử carbon-12:
		\item$1 \mathrm{amu}=\dfrac{19,926 \cdot 10^{-27} \mathrm{~kg}}{12}=1,661 \cdot 10^{-27} \mathrm{~kg}$
		\item$1 \mathrm{mol}$ chứa $ 6,02.10^{23} $ nguyên tử, phân tử, ion.
	\end{itemize}
	\tieumuc{Các công thức liên quan kích thước}
	\begin{itemize}
		\item Thể tích của hình cầu:
		$ V=\dfrac{4}{3}\pi r^3 $
		\item Phần trăm thể tích các nguyên tử trong tinh thể $ = \dfrac{V_{\text{các nguyên tử}}}{V_{\text{tinh thể}}}\cdot 100\% $
		\item Một số đơn vị đo: 
		$\left\{\begin{array}{l}
			1~\mathrm{nm} = 10^{-9}~\mathrm{m}\\
			1~\mathrm{A^{0}} = 10^{-10}~\mathrm{m}\\
			1~\mathrm{pm} = 10^{-12}~\mathrm{m}	
		\end{array}\right.$
	\end{itemize}
\end{dang}

\Noibat[\maunhan][][\faBookmark]{Ví dụ mẫu}
%%%==========VDM1====================%%%
\begin{vdex}
	Khối lượng của nguyên tử magnesium là $39{,}8271 \cdot 10^{-27} \mathrm{~kg}$. Khối lượng của magnesium theo amu là bao nhiêu? Biết rằng $1amu=19{,}926 \cdot 10^{-27}$ kg.
	\choice
	{\True $ 23{,}978 $}
	{$66{,}133 \cdot 10^{-51}$}
	{$23{,}985 \cdot 10^{-3}$}
	{$ 24{,}000 $}
	\loigiai{
		\begin{align*}
			m_{\text{Mg}} (\text{amu}) &= \frac{m_{\text{Mg}} (\text{kg})}{m_{1 \text{ amu}} (\text{kg})} 
			= \frac{39,8271 \cdot 10^{-27} \text{ kg}}{19,926 \cdot 10^{-27} \text{ kg}} 
			= \frac{39,8271}{19,926} \\
			&= 1,9988 
			\approx 23,978 \text{ amu}
		\end{align*}
		Vậy khối lượng của nguyên tử magnesium là $23{,}978$ amu.\\
		Đáp án đúng là $23{,}978$ amu.
	}    
\end{vdex}
%%%==========VDM2====================%%%
\begin{vdex}
	Khối lượng tuyệt đối của một nguyên tử oxygen bằng $26,5595.10^{-27} \mathrm{~kg}$. Hãy tính khối lượng nguyên tử (theo amu) và khối lượng mol nguyên tử (theo g) của nguyên tử này.
	\loigiai
	{%
		$
		1 \mathrm{amu}=1,661 \cdot 10^{-27} \mathrm{~kg}
		$\\
		Khối lượng của nguyên tử oxygen theo amu là:
		$
		\dfrac{26,5595 \cdot 10^{-27}}{1,661 \cdot 10^{-27}} \approx 15,99~ \mathrm{amu}
		$\\
		$1 \mathrm{mol}$ chứa $ 6,02.10^{23} $ nguyên tử\\
		$\Rightarrow$ Khối lượng mol của oxygen là  $=26,5595.10^{-24}.6,02.10^{23}= 15,99~ \mathrm{gam} $
	}
\end{vdex}
%%%==========VDM3====================%%%
\begin{vdex}
	Nguyên tử helium có 2 proton, 2 neutron và 2 electron. Khối lượng của các electron chiếm bao nhiêu $\%$ khối lượng nguyên tử helium?
	\choice
	{$2,72 \%$}
	{$0,272 \%$}
	{\True$0,0272 \%$}
	{$0,0227 \%$}
	\loigiai{Khối lượng nguyên tử helium là:\\ $ m_{NT} = 2m_{p} + 2m_{n} + 2m_{e} = 2.1,672.10^{-27} + 2.1,675.10^{-27} + 2 .9,109.10^{-31} = 6.696.10^{-27}~\mathrm (kg) $\\
		Phần trăm khối lượng của electron trong nguyên tử helium là:\\
		$ \%m_{e}=\dfrac{2 .9,109.10^{-31}}{5.51941.10^{-27}}.100\%=0,0272 \%$
	}
\end{vdex}
%%%==========VDM4====================%%%
\begin{vdex}
	Khối lượng riêng của canxi kim loại  là $ 1,55 g/cm^3 $. Giả thiết rằng , trong tinh thể canxi các nguyên tử là những hình cầu chiếm $ 74\% $ thể tích tinh thể, phần còn lại là khe rỗng.Bán kính nguyên tử tính theo lý thuyết là
	\choice
	{$0,185~\mathrm{nm}$}
	{\True	$0,196~\mathrm{nm}$}
	{$0,155~\mathrm{nm}$}
	{$0,168~\mathrm{nm}$}
	\loigiai{Lấy 1 mol Ca\\
		Ta có: $ D_{Ca}=\tfrac{m_{Ca}}{V_{\scriptsize\text{tinh thể Ca}}}=\tfrac{M_{Ca}.1}{V_{\scriptsize\text{tinh thể Ca}}}\Rightarrow V_{\scriptsize\text{tinh thể Ca}} = \tfrac{M_{Ca}}{D_{Ca}} ~\mathrm{cm^{3}} $\\
		Thể tích 1 mol Ca là: $ V_{\scriptsize\text{ 1 mol Ca} } = \tfrac{74}{100} \cdot V_{\scriptsize\text{tinh thể Ca}} = \tfrac{74}{100} \cdot \tfrac{M_{Ca}}{D_{Ca}} $\\
		Thể tích một nguyên tử Canxi là:
		$V_{\scriptsize\text{1 NT Ca}} = \tfrac{V_{\scriptsize\text{ 1 mol Ca}}}{6,02.10^{23}}=\tfrac{74.M_{Ca}}{6,02.10^{23}.100.D_{Ca}} $\\
		$ \Rightarrow \tfrac{4}{3}\pi r^{3} = \tfrac{74.M_{Ca}}{6,02.10^{23}.100.D_{Ca}} \Rightarrow \tfrac{4}{3}\pi r^{3} = \tfrac{74.40}{6,02.10^{23}.100.1,55} \Rightarrow r= 1,96.10^{-8}~\mathrm{cm}=0,196 ~\mathrm{nm} $ 
	}
\end{vdex}
%%%==================Bài tập tự luyện dạng 2=============%%%%
\Noibat[\maunhan][][\faBank]{Bài tập tự luyện dạng \thedang}
%%%Phần trắc nghiệm%%%
\phan{Bài tập trắc nghiệm}
\Opensolutionfile{ansex}[Ans/LGEX-Hoa10_C01B01_CTNT_BTTL02]
\Opensolutionfile{ans}[Ans/Ans-Hoa10_C01B01_CTNT_BTTL02]
%\hienthiloigiaiex
%\tatloigiaiex
%%\luuloigiaiex

\begin{ex}%[0H1V2-2]
	Bán kính nguyên tử và khối lượng mol của nguyên tử $ Fe $ lần lượt là $ 1{,}28 A^{0} $ và $ 56  $ gam/mol . Biết rằng trong tinh thể $ Fe $ chỉ chiếm $ 74\% $ về thể tích, còn lại là rỗng. Khối lượng riêng của sắt là
	\choice
	{\True	$ 7{,}83 ~\mathrm{gam /cm^{3}}$}
	{$ 8{,}74 ~\mathrm{gam /cm^{3}}$}
	{$ 4{,}78 ~\mathrm{gam /cm^{3}}$}
	{$ 7{,}48 ~\mathrm{gam /cm^{3}}$}
	\loigiai{Thể tích tinh thể sắt
		\begin{align*}
			V_{tt} &= V_{\text{1 mol}}\times\tfrac{100}{\text{độ chặt khít}}\\
			&=\frac{4}{3}\pi r^3\times N_{A}\times\tfrac{100}{\text{độ chặt khít}}\\
			&=\frac{4}{3}\pi\cdot \left(1{,}28\cdot10^{-8}\right)^3\times 6{,}022\cdot10^{23}\times\tfrac{100}{74}\\
			&\approx 7{,}149 ~\mathrm{cm^3}
		\end{align*}
		Giả sử xét  1 mol Fe, ta có $m_{Fe}=M_{Fe}\cdot1=56$ (gam)\\
		Khối lượng riêng của sắt:
		$D_{Fe} = \dfrac{m_{Fe}}{V_{tt}}= \dfrac{56}{7{,}149}\approx 7{,}83 ~\mathrm{gam /cm^{3}}$
	}
\end{ex}

\Closesolutionfile{ans}
\Closesolutionfile{ansex}
%\bangdapan{Ans-Hoa10_C01B01_CTNT_BTTL02}

%%% BTTL dạng 2Phần tự luận%%%%
\phan{Bài tập tự luận}
\Opensolutionfile{ansbth}[Ans/LGBT-Hoa10_C01B01_CTNT_BTTL02]
\Opensolutionfile{ansbt}[Ans/AnsBT-Hoa10_C01B01_CTNT_BTTL02]
%\luuloigiaibt
%\hienthiloigiaibt
\begin{bt}%[0H1?2-2]
	Nguyên tử aluminium (nhôm) gồm 13 proton và 14 neutron. Tính khối lượng proton, neutron, electron có trong $27 \mathrm{~g}$ nhôm.
	\loigiai{
		Ta có : $ n_{Al}=\dfrac{m_{Al}}{M_{Al}}= \dfrac{27}{27}=1~\mathrm{mol}\\ $	
		$ \Rightarrow $ Khối lượng proton là: $ 13.1,672.10^{-24}.6,02.10^{23} =13,0972 ~\mathrm{gam} $\\
		Khối lượng neutron là: $14 \cdot 1,675 \cdot 10^{-24} \cdot 6,022 \cdot 10^{23}=14,1216(\mathrm{~g})$.\\
		Khối lượng electron là: $13 \cdot 9,109 \cdot 10^{-28} \cdot 6,022 \cdot 10^{23}=7,131 \cdot 10^{-3}(\mathrm{~g})$.\
	}
\end{bt}
%
\begin{bt}%[0H1C2-2]
	Nguyên tử $\mathrm{Fe}$ ở $20^{\circ} \mathrm{C}$ có khối lượng riêng là $7,87 \mathrm{~g} / \mathrm{cm}^3$. Với giả thiết này, tinh thể nguyên tử Fe là những hình cầu chiếm $75 \%$ thể tích tinh thể, phần còn lại là những khe rỗng giữa các quả cầu. Cho biết khối lượng nguyên tử của Fe là 55,847 . Tính bán kính nguyên tử gần đúng của $\mathrm{Fe}$.
	\loigiai
	{%
		\noindent Lấy 1 mol Fe
		Ta có: $ D_{Fe}=\dfrac{m_{Fe}}{V_{\scriptsize\text{tinh thể Fe}}}=\dfrac{M_{Fe}.1}{V_{\scriptsize\text{tinh thể Fe}}}\Rightarrow V_{\scriptsize\text{tinh thể Fe}} = \dfrac{M_{Fe}}{D_{Fe}} ~\mathrm{cm^{3}} $\\
		Thể tích 1 mol Fe là: $ V_{\scriptsize\text{ 1 mol Fe} } = \dfrac{75}{100} \cdot V_{\scriptsize\text{tinh thể Fe}} = \dfrac{75}{100} \cdot \dfrac{M_{Fe}}{D_{Fe}} $\\
		Thể tích một nguyên tử Fe là:
		$V_{\scriptsize\text{1 NT Ca}} = \dfrac{V_{\scriptsize\text{ 1 mol Fe}}}{6,02.10^{23}}=\dfrac{75.M_{Fe}}{6,02.10^{23}.100.D_{Fe}} $\\
		$ \Rightarrow \dfrac{4}{3}\pi r^{3} = \dfrac{75.M_{Fe}}{6,02.10^{23}.100.D_{Fe}} \Rightarrow \dfrac{4}{3}\pi r^{3} = \dfrac{75.55,847}{6,02.10^{23}.100.7,87} \Rightarrow r= 1,28.10^{-8}~\mathrm{cm}=0,128 ~\mathrm{nm} $ 
	}
\end{bt}
%
\begin{bt}%[0H1C2-2]
	Nguyên tử kẽm $(\mathrm{Zn})$ có nguyên tử khối bằng 65 . Thực tế hầu như toàn bộ khối lượng nguyên tử tập trung ở hạt nhân, với bán kinh $r=2 \times 10^{-15} \mathrm{~m}$. Khối lượng riêng của hạt nhân nguyên tử kẽm là bao nhiêu tấn trên một centimet khối (tấn/cm³)?
	\loigiai{
		\noindent Đổi $\mathrm{r}=2 \times 10^{-15} \mathrm{~m}=2 \times 10^{-13} \mathrm{~cm}$.\\
		Thể tích hạt nhân nguyên tử Zn:$ =\dfrac{4}{3}\pi r^{3} =\dfrac{4}{3}\pi (2x10^{-13})^{3}=3,349.10^{-38}~\mathrm{cm^{3}} $\\
		Ta có $1 \mathrm{u}=1,66.10^{-27} \mathrm{~kg}=1,66.10^{-30}$ tấn.\\
		Khối lượng riêng của hạt nhân nguyên tử Zn là:
		$
		d=\dfrac{65.1,66 \cdot 10^{-30}}{3,349 \cdot 10^{-38}}=3,22.10^9\left(\text { tấn } / \mathrm{cm}^3\right. \text { ) }
		$
	}
\end{bt}
\Closesolutionfile{ansbt}
\Closesolutionfile{ansbth}
%\bangdapanSA{AnsBT-Hoa10_C01B01_CTNT_BTTL02}






%%%===================DẠNG 3==============================%%%
\newpage
\begin{dang}{Bài tập về các loại hạt}
\end{dang}
\Noibat[][][\faCamera]{Phương pháp giải}\\
\Noibat[\maunhan][][\faStar]{Các loại hạt của nguyên tử}
\begin{itemize}
	\item	Xét nguyyên tử X. Gọi Z là số proton của Z
	$ \Rightarrow $ Số electron của X là Z.
	Gọi N  là số nơtron của X.
	\begin{itemize}
		\item Số hạt mang điện của nguyên tử X là \indam[dndo]{$ \mathbf= $ số p $\mathbf + $ số e $\mathbf = 2Z +N $}
		\item Số hạt mang điện dương của nguyên tử X là \indam[dndo]{$\mathbf = $ số p $ \mathbf = Z  $}
		\item Số hạt mang điện âm của nguyên tử X là \indam[dndo]{ {$ \mathbf = $} số e $\mathbf = $ số p $\mathbf  = Z  $}
	\end{itemize}
	\item Đối với các nguyên tố có số proton từ 2 đến 82 $ (2<Z<82) $.Ta luôn có : \indam[dndo]{$\mathbf{1<\dfrac{N}{Z} <1,5} $}
	\item Xét hợp chất $ M $ có công thức là $ X_{n}Y_{m} $
	\begin{itemize}
		\item Số proton của $ M $ là $ n.Z_{X} + m.Z_{Y} $
		\item Số electron của $ M $ là $ n.Z_{X} + m.Z_{Y} $
		\item Số nơtron của $ M $ là $ n.N_{X} + m.N_{Y} $
	\end{itemize}
\end{itemize}
\Noibat[\maunhan][][\faStar]{Các loại hạt của ion}\\
\begin{itemize}
	\item Nguyên tử trung hòa về điện khi  mất bớt electron trở thành ion dương (cation)
	\begin{center}
		\tcbox[colback=dndo!15,frame hidden,colframe=dndo]{$X  \longrightarrow X^{n+} + ne $}
	\end{center}
	\begin{itemize}
		\item Số proton của $ X^{n+} = Z $.
		\item Số electron của $ X^{n+} = Z-n $.
		\item Số nơtron của $ X^{n+} = N $.
	\end{itemize}
	\item Nguyên tử trung hòa về điện khi nhận thêm electron trở thành ion âm (anion)
	\begin{center}
		\tcbox[colback=dndo!15,frame hidden,colframe=dndo]{$ X + me \longrightarrow X^{m+} $}
	\end{center}
	\begin{itemize}
		\item Số proton của $ X^{m-} = Z $.
		\item Số electron của $ X^{m-} = Z+m $.
		\item Số nơtron của $ X^{m-} = N $.
	\end{itemize}
\end{itemize}
%%%===========Ví dụ mấu===========
\Noibat[\maunhan][][\faBookmark]{Ví dụ mẫu}
%%%=============VDM1=============%%%
\begin{vdex}
	Nguyên tử nguyên tố X có tổng số hạt cơ bản là 40. Trong đó số hạt mang điện nhiều hơn số hạt không mang điện là 12. Nguyên tố X là:
	\choice
	{\True Al}
	{Na}
	{Ca}
	{F}
	\loigiai{
		Gọi Z là số proton và N là số nơtron có trong nguyên tử X.\\
		Theo đề bài nguyên tử X có tổng số hạt cơ bản là $ 40 $ nên ta có:
		$ P + E + N = 40  $\\
		Vì P=E nên:
		\begin{equation}
			\Rightarrow 2Z + N = 40 \label{eq:1}
		\end{equation} 
		
		Mặt khác số hạt mang điện  nhiều hơn số hạt không mang điện là 12, nên ta có: 
		\begin{equation}
			2Z-N=12 \label{eq:2}
		\end{equation}
		
		Từ \eqref{eq:1} và \eqref{eq:2} ta có hệ phương trình:
		$ \begin{cases}
			2Z+N=40\\
			2Z-N =12
		\end{cases} $
		$ \Rightarrow  
		\begin{cases}
			Z=13\\
			N =14
		\end{cases} $ 
		Vậy X là nguyên tố Al (nhôm)
	}
\end{vdex}

%%%=============VDM_2=============%%%
\begin{vdex}
	Tổng số hạt proton,nơtron, electron trong nguyên tử của nguyên tố X là 46. Biết rằng công thức oxit của X có dạng $ X_{2}O_{5} $.X là nguyên tố
	\choice
	{N}
	{\True P}
	{O}
	{S}
	\loigiai{
		Theo đề bài ta có tổng số hạt của nguyên tử X là 46 $\Rightarrow S=2Z+N=46$\\
		Mặt khác theo điều kiện bền của hạt nhân ta có
		\begin{align*}
			&1\leq \dfrac{N}{Z}\leq 1{,}5\\
			\Rightarrow &3Z \leq 2Z+N\leq 3{,}5Z\\
			\text{hay}\; &3Z \leq S\leq 3{,}5Z\\
			\Leftrightarrow & \dfrac{S}{3} \leq Z\leq \dfrac{S}{3{,}5}\\
			\Leftrightarrow & \dfrac{46}{3} \leq Z\leq \dfrac{46}{3{,}5}\\
			\Leftrightarrow & 13{,}14 \leq Z\leq 15{,}3\\
		\end{align*}
		Vì $Z \in \mathbb{N}$ nên $Z \in \{13; 14; 15\}$
		\\
		Bảng biện luận \par
		{\renewcommand{\arraystretch}{0.7}\begin{tabular}{|C{0.1\linewidth}|*{3}{C{0.15\linewidth}|}}
				\hline\rowcolor{\mycolor!10}
				\rule[-1.5ex]{0cm}{4.5ex}\textbf{X}&\textbf{Al}&\textbf{Si}&\textbf{P}\\
				\hline
				\multirow{2}{*}{$\mathsf{X_2O_5}$}&\rule[1.5ex]{0ex}{1.5ex}$\mathsf{Al_2O_5}$&$\mathsf{Si_2O_5}$&$\mathsf{P_2O_5}$\\
				&\rule[-1.5ex]{0ex}{1.5ex}(loại)&(loại)&(nhận)\\
				\hline
			\end{tabular}\par}
		Vậy X là P và công thức oxit tương ứng là $P_2O_5$
	}
\end{vdex}

%%%=============Bài tập tự luyên dạng 3===========%%%
\Noibat[][][\faBank]{Bài tập tự luyện dang \thedang}
%%%=============Phần trắc nghiệm================%%%
\phan{Bài tập trắc nghiệm}
\Opensolutionfile{ansex}[Ans/LGEX-Hoa10_C01B01_CTNT_BTTL03]
\Opensolutionfile{ans}[Ans/Ans-Hoa10_C01B01_CTNT_BTTL03]
%\hienthiloigiaiex
%\tatloigiaiex
%\luuloigiaiex
%%%=========ex_1=========%%%
\begin{ex}%[0H1H2-1]
	Nguyên tử của một nguyên tố X có tổng số hạt cơ bản là 82. Biết số hạt mang điện nhiều hơn số hạt không mang điện là 22. Tổng số proton và nơtron của X là:
	\choice
	{58}
	{57}
	{\True 56}
	{55}
	\loigiai{Gọi P là số proton, N là số nơtron, E là số electron.
		Ta có:
		$\heva{
			&Z + N + E = 82 &\;\;\text{(1)}\\
			&(Z + E) - N = 22 &\;\;\text{(2)}
		}$ (*)
		\\
		Vì là nguyên tử trung hòa nên $P = E = Z$. Thay vào (*) ta được
		$\heva{
			&2Z + N = 82\\
			&2Z - N = 22
		}$ $\Leftrightarrow$ $\heva{
			&Z = 26\\
			&N = 30
		}$.\\
		Vậy $Z+N=26+30=56$
	}
\end{ex}
%%%=========ex_2=========%%%
\begin{ex}%[0H1V2-1]
	Tổng số hạt trong cation $R^{2+}$ là 58. Trong nguyên tử R số hạt mang điện nhiều hơn số hạt không mang điện là 20 hạt. Số electron của cation $R^{2+}$ là:
	\choice
	{\True 18}
	{22}
	{20}
	{16}
	\loigiai{Gọi P là số proton, N là số nơtron, E là số electron của nguyên tử R.\\
		Ta có:
		$\heva{
			&P + N + (E-2) = 58 &\;\;\text{(1)}\\
			&(P + E) - N = 20 &\;\;\text{(2)}
		}$. (*)
		\\
		Vì là nguyên tử trung hòa nên $P = E = Z$. Thay vào (*) ta được
		$\heva{
			&2Z + N = 60\\
			&2Z - N = 20
		}$ $\Leftrightarrow$ $\heva{
			&Z = 20\\
			&N = 20
		}$.\\
		Vậy số electron của cation $R^{2+}$ là: $Z - 2 = 20 - 2 = 18$
	}
\end{ex}
%%%=========ex_3=========%%%
\begin{ex}%[0H1V2-1]
	Nguyên tử của nguyên tố Y có tổng số hạt là 16. Số electron của nguyên tử Y là:
	\choice
	{7}
	{6}
	{5}
	{8}
	\loigiai{Theo đề bài ta có tổng số hạt của nguyên tử Y là 16 $\Rightarrow S=2Z+N=16$\\
		Mặt khác theo điều kiện bền của hạt nhân ta có
		\begin{align*}
			&1\leq \dfrac{N}{Z}\leq 1{,}5\\
			\Rightarrow &3Z \leq 2Z+N\leq 3{,}5Z\\
			\text{hay}\; &3Z \leq S\leq 3{,}5Z\\
			\Leftrightarrow & \dfrac{S}{3{,}5} \leq Z\leq \dfrac{S}{3}\\
			\Leftrightarrow & \dfrac{16}{3{,}5} \leq Z\leq \dfrac{16}{3}\\
			\Leftrightarrow & 4{,}57 \leq Z\leq 5{,}33\\
		\end{align*}
		Vì $Z \in \mathbb{N}$ nên $Z = 5$
		\\
		Trong nguyên tử trung hòa, số electron bằng số proton (Z).
		\\
		Vậy số electron của nguyên tử Y là 5.
	}
\end{ex}
%%%=========ex_4=========%%%
\begin{ex}%[0H1C2-1]
	Tổng số electron trong ion $AB_{3}^{-}$ là $32$ hạt. Số hạt mang điện trong nguyên tử A nhiều hơn số hạt mang điện trong hạt nhân nguyên tử B là 6 hạt. Số proton của A và B lần lượt là:
	\choice
	{6 và 7}
	{\True 7 và 8}
	{8 và 9}
	{5 và 6}
	\loigiai{%
		Gọi $Z_A$ và $Z_B$ lần lượt là số proton của A và B.\\
		Theo đề bài ta có tổng số electron  trong ion $AB_{3}^{-}$ nên ta có $E_{A}+3E_{B}+1 =32$.\\ Vì $Z=E$ $\Rightarrow$ $Z_{A}+3Z_{B}+1 =32$ (1)
		\\
		Mặt khác ta lại có $2Z_{A}-Z_B=6$ (2)\\
		Từ (1) và (2) ta có hệ phương trình $\heva{&Z_{A}+3Z_{B} =31\\&2Z_{A}-Z_B=6}$ $\Leftrightarrow$ $\heva{&Z_{A}=7\\&Z_{B}=8}$.
		\\
		Vậy số proton của A và B lần lượt là 7 và 8.
	}
\end{ex}
\Closesolutionfile{ans}
\Closesolutionfile{ansex}
%\bangdapan{Ans-Hoa10_C01B01_CTNT_BTTL03}

%%%=============Phần Tự luận================%%%
\phan{Bài tập tự luận}
\Opensolutionfile{ansbth}[Ans/LGBT-Hoa10_C01B01_CTNT_BTTL03]
\Opensolutionfile{ansbt}[Ans/AnsBT-Hoa10_C01B01_CTNT_BTTL03]
%\luuloigiaibt
%\hienthiloigiaibt
\begin{bt}[Bài tập 1.11 SBT hóa 10 KNTT]%[0H1V2-1]
	Hợp kim chứa nguyên tố $\mathrm{X}$ nhẹ và bền, dùng chế tạo vỏ máy bay, tên lửa. Nguyên tố $\mathrm{X}$ còn được sử dụng trong xây dựng, ngành điện và đồ gia dụng. Nguyên tử của nguyên tố $\mathrm{X}$ có tổng số hạt (proton, electron, neutron) là 40 . Tổng số hạt mang điện nhiều hơn tổng số hạt không mang điện là 12 .
	\begin{enumerate}[a)]
		\item Tính số mỗi loại hạt (proton, electron, neutron) trong nguyên tử $\mathrm{X}$.
		\item Tính số khối của nguyên tử $\mathrm{X}$.
	\end{enumerate}
	\loigiai{
		\begin{enumerate}
			\item Gọi Z, P lần lượt là số proton và nơtron của nguyên tử X.\\
			Nguyên tử trung hòa về điện nên $p=$ e.
			Theo bài ra ta có: $p+e+n=40$ hay 
			\begin{equation}
				2 Z+N=40 \label{eq:pt3}
			\end{equation}
			và 
			\begin{equation}
				2 Z-N=12  \label{eq:pt4}
			\end{equation}
			Từ \eqref{eq:pt3} và \eqref{eq:pt4} ta có hệ phương trình:
			$ \left\{
			\begin{array}{l}
				2 Z+N=40\\
				2 Z-N=12
			\end{array}
			\right.$ 
			$\Leftrightarrow  $
			$ \left\{
			\begin{array}{l}
				Z=13\\
				N=14
			\end{array}
			\right. $
			
			$\Rightarrow  p=e=13 $ và $ n=14 $
			\item Số khối của X là:$ A_{X} =Z + N=13+14 =27$
		\end{enumerate}
	}
\end{bt}
\Closesolutionfile{ansbt}
\Closesolutionfile{ansbth}
%\bangdapanSA{AnsBT-Hoa10_C01B01_CTNT_BTTL03}