%%%%==============BT_3/SGK==============%%%
%\setcounter{bt}{2}
%\begin{bt}
%Cho tam giác $ABC$ có hai đường trung tuyến $AM$ và $BN$ cắt nhau tại $G$. Trên tia đối của tia $MA$ lấy điểm $D$ sao cho $MD=MG$. Chứng minh:
%	\begin{enumEX}{3}
%	\item $GA=GD$;
%	\item $\triangle MBG=\triangle MCD$;
%	\item $CD=2GN$.
%	\end{enumEX}
%\end{bt}
%
%%%%==============BT_4/SGK==============%%%
%\begin{bt}
%	Cho tam giác $ABC$ có hai đường trung tuyến $AM$ và $BN$ cắt nhau tại $G$. Gọi $H$ là hình chiếu của $A$ lên đường thẳng $BC$. Giả sử $H$ là trung điểm của đoạn thẳng $BM$. Chứng minh:		
%	\begin{enumEX}{2}
%		\item $\triangle AHB=\triangle AHM$;
%		\item $AG=\dfrac{2}{3} AB$.
%	\end{enumEX}
%\end{bt}
%
%%%%==============BT_5/SGK==============%%%
%\begin{bt}
%Hình 107 là mặt cắt đứng của một ngôi nhà ba tầng có mái dốc. Mỗi tầng cao $3,3\mathrm{m}$. Mặt cắt mái nhà có dạng tam giác $ABC$ cân tại $A$ với đường trung tuyến $AH$ dài $1,2\mathrm{m}$. Tại vị trí $O$ là trọng tâm tam giác $ABC$, người ta làm tâm cho một cửa sổ có dạng hình tròn.
%	\begin{enumEX}{2}
%		\item $AH$ có vuông góc với $BC$ không? Vì sao?
%		\item Vị trí $O$ ở độ cao bao nhiêu mét so với mặt đất.
%	\end{enumEX}
%\end{bt}
%
%%%%==============BT_2Mẫu/SBT_CTST==============%%%
%\begin{bt}
%	Cho tam giác $ABC$ có các đường trung tuyến $AD, BF, CE$ cùng đi qua điểm G. Tính các tỉ số: $\dfrac{AG}{AD}; \dfrac{FG}{GB}; \dfrac{EG}{EC}$.
%\end{bt}
%
%%%%=============BT_3/SBT_CTST==============%%%
%\begin{bt}
%Cho tam giác $ABC$ có hai trung tuyến $AM$ và $CN$ cắt nhau tại $G$.
%	\begin{enumerate}
%		\item Biết $AM=12\mathrm{~cm}$, tính $AG$.
%		\item Biết $GN=3\mathrm{~cm}$, tính $CN$.
%		\item Tìm $x$ biết $AG=3x-4, GM=x$.
%	\end{enumerate}
%\end{bt}
\Opensolutionfile{ans}[Ans/DATAM-VL9CKII-De-01]
\Opensolutionfile{ansbook}[Ans/DATNTF-VL9CKII-De-01]
\LGexTF
\Opensolutionfile{ansex}[LOIGIAITN/LGTNTF-VL9CKII-De-01]

%%%============EX_1==============%%%
\begin{ex}
	\immini{Câu hỏi xác định đúng sai.
		\choiceTF[t]
		{PA sai 1}
		{\True PA đúng 1}
		{PA sai 2}
		{\True PA đúng 2}}{\tikz{\draw circle (3cm);}}
	\loigiai{Nội dung lời giải}
\end{ex}

\begin{ex}
	Câu hỏi xác định đúng sai.
	\choiceTF[tt]
	{\True PA đúng 1}
	{PA sai 1}
	{\True PA đúng 2}
	{PA sai 2}
	\loigiai{Nội dung lời giải}
\end{ex}
\Closesolutionfile{ansex}
\Closesolutionfile{ansbook}
\Closesolutionfile{ans}	