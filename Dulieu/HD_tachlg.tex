%\Opensolutionfile{ansbt}[LOIGIAITN/LGTNCHUONG1]
%\Opensolutionfile{ans}[Ans/DATN]
%%\Writetofile{ansbt}{\protect\thongtin{Lời giải chiết phần trắc nghiệm}}
%\nhanmanh{Phần Trắc Nghiệm}
%	%%%============Phần trắc nghiệm============%%%
%	%%%%%%%%%%%%%%%%%%%%%%%%%%%%%%%%%%%%%%%%%%
%	\begin{ex}
%		Nội dung câu hỏi thứ nhất
%		\choice{\True Nội dung PA sai }
%		{Nội dung PA sai 1 }
%		{Nội dung PA sai 1 }
%		{Nội dung PA sai 3}
%		\loigiai{Nội dung lời giải}
%	\end{ex}
%	\begin{ex}
%		Nội dung câu hỏi thứ nhất
%		\choice{\True Nội dung PA sai }
%		{Nội dung PA sai 1 }
%		{Nội dung PA sai 1 }
%		{Nội dung PA sai 3}
%		\loigiai{Nội dung lời giải}
%	\end{ex}
%	
%	\begin{ex}
%		Nội dung câu hỏi thứ nhất
%		\choice{\True Nội dung PA sai }
%		{Nội dung PA sai 1 }
%		{Nội dung PA sai 1 }
%		{Nội dung PA sai 3}
%		\loigiai{Nội dung lời giải}
%	\end{ex}
%	\begin{ex}
%		Nội dung câu hỏi thứ nhất
%		\choice{\True Nội dung PA sai }
%		{Nội dung PA sai 1 }
%		{Nội dung PA sai 1 }
%		{Nội dung PA sai 3}
%		\loigiai{Nội dung lời giải}
%	\end{ex}
%	\begin{ex}
%		Nội dung câu hỏi thứ nhất
%		\choice{\True Nội dung PA sai }
%		{Nội dung PA sai 1 }
%		{Nội dung PA sai 1 }
%		{Nội dung PA sai 3}
%		\loigiai{Nội dung lời giải}
%	\end{ex}
%	\begin{ex}
%		Nội dung câu hỏi thứ nhất
%		\choice{\True Nội dung PA sai }
%		{Nội dung PA sai 1 }
%		{Nội dung PA sai 1 }
%		{Nội dung PA sai 3}
%		\loigiai{Nội dung lời giải}
%	\end{ex}
%	\begin{ex}
%		Nội dung câu hỏi thứ nhất
%		\choice{\True Nội dung PA sai }
%		{Nội dung PA sai 1 }
%		{Nội dung PA sai 1 }
%		{Nội dung PA sai 3}
%		\loigiai{Nội dung lời giải}
%	\end{ex}
%	\begin{ex}
%		Nội dung câu hỏi thứ nhất
%		\choice{\True Nội dung PA sai }
%		{Nội dung PA sai 1 }
%		{Nội dung PA sai 1 }
%		{Nội dung PA sai 3}
%		\loigiai{Nội dung lời giải}
%	\end{ex}
%	\begin{ex}
%		Nội dung câu hỏi thứ nhất
%		\choice{\True Nội dung PA sai }
%		{Nội dung PA sai 1 }
%		{Nội dung PA sai 1 }
%		{Nội dung PA sai 3}
%		\loigiai{Nội dung lời giải}
%	\end{ex}
%	\begin{ex}
%		Nội dung câu hỏi thứ nhất
%		\choice{\True Nội dung PA sai }
%		{Nội dung PA sai 1 }
%		{Nội dung PA sai 1 }
%		{Nội dung PA sai 3}
%		\loigiai{Nội dung lời giải}
%	\end{ex}
%	\begin{ex}
%		Nội dung câu hỏi thứ nhất
%		\choice{\True Nội dung PA sai }
%		{Nội dung PA sai 1 }
%		{Nội dung PA sai 1 }
%		{Nội dung PA sai 3}
%		\loigiai{Nội dung lời giải}
%	\end{ex}
%	\begin{ex}
%		Nội dung câu hỏi thứ nhất
%		\choice{\True Nội dung PA sai }
%		{Nội dung PA sai 1 }
%		{Nội dung PA sai 1 }
%		{Nội dung PA sai 3}
%		\loigiai{Nội dung lời giải}
%	\end{ex}
%	\begin{ex}
%		Nội dung câu hỏi thứ nhất
%		\choice{\True Nội dung PA sai }
%		{Nội dung PA sai 1 }
%		{Nội dung PA sai 1 }
%		{Nội dung PA sai 3}
%		\loigiai{Nội dung lời giải}
%	\end{ex}
%	\begin{ex}
%		Nội dung câu hỏi thứ nhất
%		\choice{\True Nội dung PA sai }
%		{Nội dung PA sai 1 }
%		{Nội dung PA sai 1 }
%		{Nội dung PA sai 3}
%		\loigiai{Nội dung lời giải}
%	\end{ex}
%	\begin{ex}
%		Nội dung câu hỏi thứ nhất
%		\choice{\True Nội dung PA sai }
%		{Nội dung PA sai 1 }
%		{Nội dung PA sai 1 }
%		{Nội dung PA sai 3}
%		\loigiai{Nội dung lời giải}
%	\end{ex}
%	\begin{ex}
%		Nội dung câu hỏi thứ nhất
%		\choice{\True Nội dung PA sai }
%		{Nội dung PA sai 1 }
%		{Nội dung PA sai 1 }
%		{Nội dung PA sai 3}
%		\loigiai{Nội dung lời giải}
%	\end{ex}
%	\begin{ex}
%		Nội dung câu hỏi thứ nhất
%		\choice{\True Nội dung PA sai }
%		{Nội dung PA sai 1 }
%		{Nội dung PA sai 1 }
%		{Nội dung PA sai 3}
%		\loigiai{Nội dung lời giải}
%	\end{ex}	
%\Closesolutionfile{ans}	
%\Closesolutionfile{ansbt}	
%%%%%=======Phần tự luận==================%%%
%%	\Writetofile{ansbt}{\protect\thongtin{Lời giải chiết phần Tự Luận}}
%\Opensolutionfile{ansbt}[LOIGIAITULUAN/LGTLCHUONG1]
%	\nhanmanh{Phần Tự Luận}
%	\begin{bt}[1]
%		Nội dung Bài tập số 4
%		\begin{enumerate}[a)]
%			\item Nội dung ý 1
%			\item Nội dung ý 2
%			\item Nội dung ý 3
%		\end{enumerate}
%		\loigiai{%
%			\begin{enumerate}[a)]
%				\item Nội dung ý 1
%				\item Nội dung ý 2
%				\item Nội dung ý 3
%			\end{enumerate}
%		}
%	\end{bt}
%	
%	\begin{bt}[1]
%		Nội dung Bài tập số 4
%		\begin{enumerate}[a)]
%			\item Nội dung ý 1
%			\item Nội dung ý 2
%			\item Nội dung ý 3
%		\end{enumerate}
%		\loigiai{%
%			\begin{enumerate}[a)]
%				\item Nội dung ý 1
%				\item Nội dung ý 2
%				\item Nội dung ý 3
%			\end{enumerate}
%		}
%	\end{bt}
%	
%	\begin{bt}[1]
%		Nội dung Bài tập số 4
%		\begin{enumerate}[a)]
%			\item Nội dung ý 1
%			\item Nội dung ý 2
%			\item Nội dung ý 3
%		\end{enumerate}
%		\loigiai{%
%			\begin{enumerate}[a)]
%				\item Nội dung ý 1
%				\item Nội dung ý 2
%				\item Nội dung ý 3
%			\end{enumerate}
%		}
%	\end{bt}
%	
%	\begin{bt}[1]
%		Nội dung Bài tập số 4
%		\begin{enumerate}[a)]
%			\item Nội dung ý 1
%			\item Nội dung ý 2
%			\item Nội dung ý 3
%		\end{enumerate}
%		\loigiai{%
%			\begin{enumerate}[a)]
%				\item Nội dung ý 1
%				\item Nội dung ý 2
%				\item Nội dung ý 3
%			\end{enumerate}
%		}
%	\end{bt}
%\Closesolutionfile{ansbt}	
%\newpage
%\thongtin{ĐÁP ÁN VÀ LỜI GIẢI CHI TIẾT TRẮC NGHIỆM}
%\bangdapan{DATN}
%\input{LOIGIAITN/LGTNCHUONG1.tex}
%\thongtin{LỜI GIẢI CHI TIẾT TỰ LUẬN}
%\input{LOIGIAITULUAN/LGTLCHUONG1.tex}
%
%\begin{bt}[2 điểm][ĐHKA-201]
%	Nội dung vd
%	\loigiai{}
%\end{bt}
%
%

%\begin{tikzpicture}[declare function ={d=1.5cm;},node distance=d,>=stealth]
%	\tikzstyle{mynode} = [font=\color{\maunhan}\bfseries\sffamily]
%	\node [name=Al,mynode] {\rm{Al}};
%	\node [name= nhomsunfua,below=of Al,mynode]{\rm{$Al_2S_3$}};
%	\node [name= nhomhidroxit,below=of nhomsunfua,mynode]{\rm{$Al\left(OH\right)_{3}$}};
%	\node [name= nhomoxit,right=of nhomsunfua]{\rm{$Al_{2}O_{3}$}}; 
%	\node [name= Natrialuminat,right=of nhomoxit]{\rm{$NaAlO_{2}$}}; 
%	\node [name= Alx,above=of Natrialuminat]{\rm{Al}}; 
%	\node [name= nhomclorua,below=of Natrialuminat]{\rm{$AlCl_{3}$}}; 
%	\node [name= nhomhidroxity,right=of Natrialuminat]{\rm{$Al(OH)_{3}$}}; 
%	\draw[->,line join=3pt,line width=1.5pt] (Al)--(nhomsunfua) node[pos=0.5,left]{(1)} ;
%	\draw[->,line join=3pt,line width=1.5pt] (nhomhidroxit)--([xshift=-1cm]nhomoxit|-nhomhidroxit)--(nhomoxit) node[pos=0.05,left,midway]{(5)} ;
%\end{tikzpicture}
