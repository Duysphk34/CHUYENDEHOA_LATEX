%%%==============Tắt tiêu đề chân trang==========%%%
\def\myytb{}
\def\myqrcodeytb{}
\def\myqrcodezalo{}
\def\quetmaqr{}
\def\thamgianhomhoctap{}
\newpage
\setcounter{ex}{0}
%%%======================%%%
\Tieudegiua[\maunhan]{Đề cương ôn tập môn khoa học lớp 5}
\Opensolutionfile{ans}[Ans/Dapan-KH-5-CKII]
\hienthiloigiaiex
\tieumuc{Phần Trắc nghiệm (4 điểm)}
%%%============EX_1==============%%%
\begin{ex}[M1-1 điểm]
	Sự biến đổi hoá học xảy ra trong trường hợp nào dưới đây?
	\choice
	{Hoà tan đường vào nước}
	{\True Thả vôi sống vào nước}
	{Dây cao su bị kéo dãn ra}
	{Cốc thuỷ tinh bị rơi vỡ}
	\loigiai{}
\end{ex}
%%%============EX_2==============%%%
\begin{ex}[M1-1 điểm]
	Cơ quan sinh sản của thực vật có hoa là?
	\choice
	{Củ}
	{Quả}
	{\True Hoa}
	{Lá}
	\loigiai{}
\end{ex}
%%%============EX_3==============%%%
\begin{ex}[M2-1 điểm]
	Nguồn năng lượng chủ yếu của sự sống trên Trái đất là gì?
	\choice
	{\True Mặt trời}
	{Mặt trăng}
	{Gió}
	{Nước}
	\loigiai{}
\end{ex}
%%%============EX_4==============%%%
\begin{ex}[M3-1 điểm]
	Trong các nguồn năng lượng dưới đây, nguồn năng lượng nào không phải là năng lượng sạch?
	\choice
	{Năng lượng mặt trời}
	{Năng lượng nước chảy}
	{Năng lượng gió}
	{\True Năng lượng từ than đá, xăng dầu}
	\loigiai{}
\end{ex}

\tieumuc{Phần Tự luận (6 điểm)}
\hienthiloigiaiex
%%%============EX_1==============%%%
\begin{ex}[1điểm]Thú con mới sinh ra có đặc điểm gì và được thú mẹ chăm sóc như thế nào?
\loigiai{Thú con mới sinh ra đã có hình dạng giống thú trưởng thành và được thú mẹ nuôi bằng sữa cho đến khi tự đi kiếm ăn}
\end{ex}
%%%============EX_2==============%%%
\begin{ex}[1 điểm]Sự biến đổi từ chất này thành chất khác gọi là gì?
\loigiai{Sự biến đổi từ chất này thành chất khác gọi là sự biến đổi hóa học}
\end{ex}
%%%============EX_3==============%%%
\begin{ex}[1 điểm]Tài nguyên thiên nhiên là gì?
\loigiai{Tài nguyên thiên nhiên là những của cải có sẵn trong môi trường tự nhiên. Con người khai thác, sử dụng chúng cho lợi ích của bản thân và cộng đồng}
\end{ex}
%%%============EX_4==============%%%
\begin{ex}[1 điểm]Đặc điểm các loài hoa thụ phấn
\loigiai{\begin{itemize}
		\item Các loài hoa thụ phấn nhờ côn trùng thường có màu sắc sặc sỡ hoặc có hương thơm hấp dẫn côn trùng.
		\item Các loài hoa thụ phấn nhờ gió không có màu sắc đẹp, cánh hoa, đài hoa thường nhỏ hoặc không có
\end{itemize}}
\end{ex}
%%%============EX_5==============%%%
\begin{ex}[1 điểm]Để tránh lãng phí điện, ta cần phải làm gì?
\loigiai{Để tránh lãng phí điện, ta cần:
\begin{itemize}
		\item Chỉ dùng điện khi cần thiết.
		\item Tiết kiệm điện khi đun nấu, sưởi, là quần áo vì những việc này cần nhiều năng lượng điện
\end{itemize}}
\end{ex}
%%%============EX_6==============%%%
\begin{ex}[1 điểm]
Để góp phần bảo vệ môi trường xung quanh, em cần phải làm gì?
\loigiai{Để góp phần bảo vệ môi trường xung quanh, em cần phải làm các việc sau:
\begin{itemize}
		\item Tích cực tham gia trồng cây, chăm sóc và bảo vệ cây xanh ở trường, gia đình và nơi công cộng.
		\item Bỏ giấy rác vào thùng rác, vệ sinh đúng nơi quy định.
		\item  Chăm chỉ làm tổng vệ sinh dọn dẹp cho môi trường xung quanh luôn sạch sẽ
\end{itemize}}
\end{ex}
\Closesolutionfile{ans}


