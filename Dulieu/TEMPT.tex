%%%=============EX_1=============%%%
\begin{ex}
	Ai là người đầu tiên đề xuất quy luật bộ ba trong việc sắp xếp các nguyên tố hóa học?
	\choice
		{Newlands}
		{Mendeleev}
	{\True Döbereiner}
	{Moseley}
\loigiai{
	Johann Wolfgang Döbereiner là người đầu tiên đề xuất quy luật bộ ba vào năm 1829. Ông nhận thấy rằng trong một số bộ ba nguyên tố có tính chất tương tự, khối lượng nguyên tử của nguyên tố giữa xấp xỉ bằng trung bình cộng khối lượng nguyên tử của hai nguyên tố ở hai đầu.
	}
\end{ex}
%%%=============EX_2=============%%%
\begin{ex}
	\lq\lqQuy luật bát âm\rq\rq trong lịch sử phát triển bảng tuần hoàn được đề xuất bởi ai?
	\choice
		{Mendeleev}
	{\True Newlands}
	{Döbereiner}
	{Moseley}
\loigiai{
	John Newlands đề xuất \lq\lqQuy luật bát âm\rq\rq vào năm 1863. Ông nhận thấy rằng khi sắp xếp các nguyên tố theo thứ tự tăng dần của khối lượng nguyên tử, cứ mỗi nguyên tố thứ tám thì có tính chất tương tự với nguyên tố đầu tiên, giống như các nốt nhạc trong âm nhạc.
	}
\end{ex}
%%%=============EX_3=============%%%
\begin{ex}
	Đóng góp quan trọng nhất của Mendeleev trong việc xây dựng bảng tuần hoàn là gì?
\choice
	{Sắp xếp nguyên tố theo số nguyên tử tăng dần}
	{\True Để lại các ô trống và dự đoán tính chất của các nguyên tố chưa phát hiện}
	{Phát hiện ra các đồng vị của nguyên tố}
	{Giải thích cấu trúc electron của nguyên tố}
\loigiai{
	Dmitri Mendeleev đã để lại các ô trống trong bảng tuần hoàn của mình và dự đoán tính chất của các nguyên tố chưa phát hiện. Điều này cho phép ông dự đoán sự tồn tại và tính chất của các nguyên tố như gallium, germanium và scandium, mà sau này đã được phát hiện và chứng minh là chính xác.
}
\end{ex}
%%%=============EX_4=============%%%
\begin{ex}
	Phát hiện nào của Moseley đã cải tiến bảng tuần hoàn của Mendeleev?
\choice
	{Khái niệm về đồng vị}
	{\True Số hiệu nguyên tử đặc trưng cho mỗi nguyên tố}
	{Cấu trúc electron của nguyên tử}
	{Sự tồn tại của các nguyên tố nhân tạo}
\loigiai{
	Henry Moseley phát hiện ra rằng mỗi nguyên tố có một số nguyên tử đặc trưng, và số này tăng dần khi đi từ nguyên tố này sang nguyên tố khác trong bảng tuần hoàn. Phát hiện này đã giải quyết được vấn đề sắp xếp không chính xác của một số cặp nguyên tố trong bảng Mendeleev và xác định chính xác vị trí của các nguyên tố trong bảng tuần hoàn.
}
\end{ex}
%%%=============EX_5=============%%%
\begin{ex}
	Trong bảng tuần hoàn hiện đại, các nguyên tố được sắp xếp theo thứ tự tăng dần của đại lượng nào?
	\choice
		{Khối lượng nguyên tử}
		{\True Số hiệu nguyên tử }
		{Số khối}
		{Số neutron trong hạt nhân}
	\loigiai{
		Trong bảng tuần hoàn hiện đại, các nguyên tố được sắp xếp theo thứ tự tăng dần của số hiệu nguyên tử hay số đơn vị điện tích hạt nhân (= số proton) trong hạt nhân. Điều này dựa trên phát hiện của Moseley và đảm bảo sự sắp xếp chính xác của các nguyên tố.
		}
\end{ex}
%%%=============EX_6=============%%%
\begin{ex}
	Định nghĩa của \lq\lqchu kỳ\rq\rq trong bảng tuần hoàn là gì?
	\choice
		{Một cột dọc trong bảng tuần hoàn}
		{\True Một hàng ngang trong bảng tuần hoàn, trong đó các nguyên tố có cấu hình electron lớp ngoài cùng biến đổi tuần hoàn}
		{Một nhóm các nguyên tố có tính chất hóa học giống nhau}
		{Khoảng cách giữa hai nguyên tố liên tiếp trong bảng}
	\loigiai{
		Một chu kỳ trong bảng tuần hoàn là một hàng ngang, trong đó các nguyên tố được sắp xếp theo thứ tự tăng dần của số nguyên tử. Trong một chu kỳ, cấu hình electron lớp ngoài cùng của các nguyên tố biến đổi tuần hoàn từ 1 đến 8 electron (trừ chu kỳ 1).
		}
\end{ex}
%%%=============EX_7=============%%%
\begin{ex}
	\lq\lq Nhóm\rq\rq trong bảng tuần hoàn được định nghĩa như thế nào?
	\choice
		{Một hàng ngang trong bảng tuần hoàn}
		{Các nguyên tố có cùng số khối}
		{\True Một cột dọc chứa các nguyên tố có cấu hình electron lớp ngoài cùng tương tự nhau}
		{Các nguyên tố có cùng số neutron}
	\loigiai{
		Một nhóm trong bảng tuần hoàn là một cột dọc chứa các nguyên tố có cấu hình electron lớp ngoài cùng tương tự nhau. Do đó, các nguyên tố trong cùng một nhóm thường có tính chất hóa học tương tự nhau.
		}
\end{ex}
%%%=============EX_8=============%%%
\begin{ex}
	Bảng tuần hoàn hiện đại có bao nhiêu chu kỳ?
	\choice
		{6}
		{\True 7}
		{8}
		{18}
	\loigiai{
		Bảng tuần hoàn hiện đại có 7 chu kỳ. Chu kỳ 1 là chu kỳ ngắn nhất với 2 nguyên tố, chu kỳ 2 và 3 có 8 nguyên tố, chu kỳ 4 và 5 có 18 nguyên tố, chu kỳ 6 có 32 nguyên tố, và chu kỳ 7 là chu kỳ chưa hoàn thiện.
		}
\end{ex}
%%%=============EX_9=============%%%
\begin{ex}
	Bảng tuần hoàn hiện đại có bao nhiêu nhóm?
	\choice
		{8}
		{16}
		{\True 18}
		{32}
	\loigiai{
		Bảng tuần hoàn hiện đại có 18 nhóm. Các nhóm được đánh số từ 1 đến 18, trong đó nhóm 1-2 và 13-18 là các nguyên tố chính, nhóm 3-12 là các nguyên tố chuyển tiếp.
		}
\end{ex}
%%%=============EX_10=============%%%
\begin{ex}
	Nguyên tố nào sau đây không phải là nguyên tố họ s?
	\choice
		{Lithium}
		{Beryllium}
		{\True Boron}
		{Sodium}
	\loigiai{
		Boron (B) không phải là nguyên tố họ s. Nó là nguyên tố họ p, thuộc nhóm 13 trong bảng tuần hoàn. Các nguyên tố họ s là những nguyên tố có electron cuối cùng điền vào orbital s, bao gồm các nguyên tố ở nhóm 1 (trừ H) và nhóm 2.
		}
\end{ex}
%%%=============EX_11=============%%%
\begin{ex}
	Các nguyên tố chuyển tiếp thuộc họ nào trong bảng tuần hoàn?
	\choice
		{Họ s}
		{Họ p}
		{\True Họ d}
		{Họ f}
	\loigiai{
		Các nguyên tố chuyển tiếp thuộc họ d trong bảng tuần hoàn. Đây là các nguyên tố mà electron cuối cùng được điền vào orbital d. Chúng chiếm các nhóm từ 3 đến 12 trong bảng tuần hoàn.
		}
\end{ex}
%%%=============EX_12=============%%%
\begin{ex}
	Nguyên tố nào sau đây là một kim loại kiềm?
	\choice
		{Beryllium}
		{Magnesium}
		{\True Potassium}
		{Calcium}
	\loigiai{
		Potassium (K) là một kim loại kiềm. Các kim loại kiềm là các nguyên tố thuộc nhóm 1 của bảng tuần hoàn (trừ Hydrogen), bao gồm Lithium, Sodium, Potassium, Rubidium, Cesium, và Francium.
		}
\end{ex}
%%%=============EX_13=============%%%
\begin{ex}
	Nguyên tố nào sau đây là một khí hiếm?
	\choice
		{Chlorine}
		{Nitrogen}
		{Oxygen}
		{\True Neon}
	\loigiai{
		Neon (Ne) là một khí hiếm. Các khí hiếm (còn gọi là khí trơ) là các nguyên tố thuộc nhóm 18 của bảng tuần hoàn, bao gồm Helium, Neon, Argon, Krypton, Xenon, và Radon.
		}
\end{ex}
%%%=============EX_14=============%%%
\begin{ex}
	Các nguyên tố họ f được gọi là gì?
	\choice
		{Nguyên tố chuyển tiếp}
		{\True Nguyên tố nội chuyển tiếp}
		{Nguyên tố khí hiếm}
		{Nguyên tố halogen}
	\loigiai{
		Các nguyên tố họ f được gọi là nguyên tố nội chuyển tiếp. Chúng bao gồm hai dãy: Lanthanide (từ Lanthanum đến Lutetium) và Actinide (từ Actinium đến Lawrencium). Các nguyên tố này có electron cuối cùng được điền vào orbital f.
		}
\end{ex}
%%%=============EX_15=============%%%
\begin{ex}
	Nguyên tố nào sau đây là một phi kim?
	\choice
		{Sodium}
		{Aluminum}
		{\True Sulfur}
		{Calcium}
	\loigiai{
		Sulfur (S) là một phi kim. Các phi kim thường nằm ở phía trên bên phải của bảng tuần hoàn (trừ các khí hiếm). Chúng có xu hướng nhận electron để tạo thành ion âm trong phản ứng hóa học.
		}
\end{ex}
%%%=============EX_16=============%%%
\begin{ex}
	Nguyên tố nào sau đây là một á kim?
	\choice
		{Oxygen}
		{\True Silicon}
		{Magnesium}
		{Chlorine}
	\loigiai{
		Silicon (Si) là một á kim. Các á kim nằm ở đường chéo giữa kim loại và phi kim trong bảng tuần hoàn. Chúng có tính chất trung gian giữa kim loại và phi kim.
		}
\end{ex}
%%%=============EX_16=============%%%
\begin{ex}
	Nguyên tắc nào được sử dụng để sắp xếp các nguyên tố trong bảng tuần hoàn hiện đại?
	\choice
		{Khối lượng nguyên tử tăng dần}
		{Số proton trong hạt nhân giảm dần}
		{\True Số proton trong hạt nhân tăng dần}
		{Số electron hóa trị giảm dần}
	\loigiai{
		Trong bảng tuần hoàn hiện đại, các nguyên tố được sắp xếp theo thứ tự tăng dần của số proton trong hạt nhân (số nguyên tử). Đây là nguyên tắc cơ bản để xác định vị trí của mỗi nguyên tố trong bảng.
	}
\end{ex}

%%%=============EX_16=============%%%
\begin{ex}
	Các nguyên tố trong cùng một nhóm của bảng tuần hoàn có đặc điểm chung nào?
	\choice
		{Cùng số neutron}
		{Cùng khối lượng nguyên tử}
		{Cùng số electron tổng cộng}
		{\True Cùng cấu hình electron lớp ngoài cùng}
	\loigiai{
		Các nguyên tố trong cùng một nhóm của bảng tuần hoàn có cấu hình electron lớp ngoài cùng giống nhau. Điều này dẫn đến sự tương đồng về tính chất hóa học giữa các nguyên tố trong cùng nhóm.
	}
\end{ex}

%%%=============EX_16=============%%%
\begin{ex}
	Chu kỳ trong bảng tuần hoàn được xác định dựa trên yếu tố nào?
	\choice
		{Số khối của nguyên tố}
		{\True Số lớp electron}
		{Số neutron trong hạt nhân}
		{Bán kính nguyên tử}
	\loigiai{
		Chu kỳ trong bảng tuần hoàn được xác định dựa trên số lớp electron của nguyên tố. Mỗi chu kỳ bắt đầu với một nguyên tố có electron bắt đầu lấp đầy một lớp electron mới.
	}
\end{ex}

%%%=============EX_16=============%%%
\begin{ex}
	Trong bảng tuần hoàn, các nguyên tố chuyển tiếp được xếp ở đâu?
	\choice
		{Nhóm IA và VIIIA}
		{Nhóm IIIA đến VIIIA}
		{\True Giữa nhóm IIA và IIIA}
		{Dưới cùng của bảng}
	\loigiai{
		Các nguyên tố chuyển tiếp được xếp ở giữa nhóm IIA và IIIA trong bảng tuần hoàn. Chúng lấp đầy các obitan d và tạo thành một "khối" riêng được gọi là khối d trong bảng tuần hoàn.
	}
\end{ex}