Xét phản ứng thuận nghịch sau đây:
N2O4 (g) ⇌ 2NO2 (g)
Tại nhiệt độ 25°C, hằng số cân bằng Kc của phản ứng là $4.61 × 10^-3$. Ban đầu, trong một bình kín 1 lít chỉ có N2O4 với nồng độ $0.08 M$. Hệ phản ứng đạt đến trạng thái cân bằng.
1. Tính nồng độ cân bằng của N2O4 và NO2.
2. Tính độ chuyển hóa của N2O4.

# Bộ bài tập về Cân bằng Hóa học

## Phần 1: Khái niệm cơ bản và hằng số cân bằng

1. Định nghĩa trạng thái cân bằng hóa học và nêu các đặc điểm của nó.

2. Phân biệt giữa Kc và Kp. Khi nào chúng bằng nhau?

3. Cho phản ứng: N2 (g) + 3H2 (g) ⇌ 2NH3 (g). Viết biểu thức của Kc và Kp.

4. Đối với phản ứng: 2SO2 (g) + O2 (g) ⇌ 2SO3 (g), Kc = 280 ở 700K. Tính Kp ở cùng nhiệt độ.

5. Giải thích tại sao hằng số cân bằng không có đơn vị.

## Phần 2: Tính toán nồng độ cân bằng

6. Ở 25°C, Kc cho phản ứng N2O4 (g) ⇌ 2NO2 (g) là 0.0421. Nếu nồng độ ban đầu của N2O4 là 0.500 M, tính nồng độ cân bằng của cả hai chất.

7. Phản ứng CO (g) + H2O (g) ⇌ CO2 (g) + H2 (g) có Kc = 23.2 ở 600K. Nếu nồng độ ban đầu của CO và H2O đều là 0.100 M, tính nồng độ cân bằng của tất cả các chất.

8. Ở 1000K, Kc cho phản ứng H2 (g) + I2 (g) ⇌ 2HI (g) là 54.5. Nếu nồng độ ban đầu của H2 và I2 đều là 0.200 M, tính nồng độ cân bằng của HI.

## Phần 3: Nguyên lý Le Châtelier

9. Giải thích nguyên lý Le Châtelier và ý nghĩa của nó trong cân bằng hóa học.

10. Đối với phản ứng tổng hợp amoniac: N2 (g) + 3H2 (g) ⇌ 2NH3 (g) + nhiệt, dự đoán ảnh hưởng của các thay đổi sau đến vị trí cân bằng:
    a) Tăng nhiệt độ
    b) Tăng áp suất
    c) Thêm chất xúc tác
    d) Loại bỏ NH3

11. Ở 25°C, phản ứng: CO2 (g) + H2 (g) ⇌ CO (g) + H2O (g) có ΔH = 41.2 kJ/mol. Làm thế nào để tăng sản lượng CO?

## Phần 4: Bài tập tổng hợp

12. Phản ứng tổng hợp metanol: CO (g) + 2H2 (g) ⇌ CH3OH (g) có Kc = 14.5 ở 500K. Nếu nồng độ ban đầu của CO là 0.300 M và của H2 là 0.400 M, tính nồng độ cân bằng của CH3OH.

13. Ở 700°C, phản ứng: 2SO2 (g) + O2 (g) ⇌ 2SO3 (g) có Kc = 4.8 × 10^4. Trong bình phản ứng 2L, ban đầu có 0.02 mol SO2 và 0.01 mol O2. Tính khối lượng SO3 tạo thành khi đạt cân bằng.

14. Phản ứng: N2O4 (g) ⇌ 2NO2 (g) có Kc = 0.0421 ở 25°C. Nếu độ phân ly của N2O4 là 15%, tính nồng độ ban đầu của N2O4.

15. Ở 1000K, phản ứng: PCl5 (g) ⇌ PCl3 (g) + Cl2 (g) có Kc = 0.0420. Nếu nồng độ ban đầu của PCl5 là 0.500 M, tính phần trăm phân ly của PCl5.


# Bộ bài tập về Cân bằng Hóa học (tiếp theo)

## Phần 5: Cân bằng trong dung dịch

16. Giải thích sự khác nhau giữa cân bằng đồng thể và dị thể. Cho ví dụ minh họa.

17. Tích số tan của AgCl ở 25°C là 1.8 × 10^-10. Tính độ tan của AgCl trong nước tinh khiết ở nhiệt độ này.

18. Tính pH của dung dịch NH3 0.1M, biết Kb của NH3 là 1.8 × 10^-5 ở 25°C.

19. Tính nồng độ ion H+ trong dung dịch CH3COOH 0.1M, biết Ka của CH3COOH là 1.8 × 10^-5 ở 25°C.

20. Một dung dịch đệm được tạo bởi 0.1M CH3COOH và 0.1M CH3COONa. Tính pH của dung dịch này, biết Ka của CH3COOH là 1.8 × 10^-5.

## Phần 6: Cân bằng và nhiệt động học

21. Giải thích mối quan hệ giữa ΔG° và hằng số cân bằng K.

22. Ở 25°C, ΔG° cho phản ứng 2NO (g) + O2 (g) → 2NO2 (g) là -70.9 kJ/mol. Tính Kp của phản ứng này ở 25°C.

23. Hằng số cân bằng cho phản ứng N2 (g) + 3H2 (g) ⇌ 2NH3 (g) là 0.5 ở 500K. Tính ΔG° của phản ứng ở nhiệt độ này.

24. Sử dụng phương trình van 't Hoff để tính toán sự thay đổi của hằng số cân bằng khi nhiệt độ thay đổi từ 300K đến 400K cho phản ứng có ΔH° = -50 kJ/mol.

25. Giải thích tại sao một số phản ứng tỏa nhiệt có hằng số cân bằng giảm khi nhiệt độ tăng.

## Phần 7: Ứng dụng trong công nghiệp

26. Trong quy trình Haber để sản xuất amoniac (N2 + 3H2 ⇌ 2NH3), giải thích tại sao người ta sử dụng:
    a) Nhiệt độ khoảng 450°C (thấp hơn nhiều so với nhiệt độ tối ưu về mặt động học)
    b) Áp suất cao (khoảng 200-300 atm)
    c) Loại bỏ liên tục NH3 khỏi hỗn hợp phản ứng

27. Trong quá trình sản xuất axit sunfuric bằng phương pháp tiếp xúc (2SO2 + O2 ⇌ 2SO3), giải thích tại sao cần sử dụng xúc tác V2O5 và tại sao phản ứng được thực hiện ở nhiệt độ khoảng 450°C.

28. Phản ứng tổng hợp methanol (CO + 2H2 ⇌ CH3OH) là một phản ứng tỏa nhiệt. Nếu mục tiêu là tối đa hóa sản lượng methanol, hãy giải thích tại sao quá trình công nghiệp vẫn sử dụng nhiệt độ cao (khoảng 300°C) và áp suất cao (50-100 atm).

29. Trong quá trình cracking xúc tác để sản xuất olefin từ hydrocarbon mạch dài, giải thích vai trò của nhiệt độ cao và áp suất thấp.

30. Phản ứng chuyển hóa hơi nước (CO + H2O ⇌ CO2 + H2) là một bước quan trọng trong sản xuất hydrogen công nghiệp. Phản ứng này là tỏa nhiệt nhẹ. Hãy đề xuất các điều kiện phản ứng (nhiệt độ, áp suất) và giải thích lý do cho đề xuất của bạn.














