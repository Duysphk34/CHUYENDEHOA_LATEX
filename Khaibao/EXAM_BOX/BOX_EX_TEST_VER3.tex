%%%%===============UPDATE 13/09/2024=====================%%%
\newcommand\circl[2][\mycolor]{\tikz[baseline=(char.base)]{\node[shape=circle,inner sep=1pt,draw=#1,fill=#1!20,text=#1!80!black,font=\bfseries\fontfamily{qag}\selectfont,minimum size=15pt] (char) {#2};
	}
}
\newcommand\circlenum[2][\mycolor]{\tikz[baseline=(char.base)]
	{\node[shape=circle,inner sep=1pt,fill=#1!20,text=#1!80!black,
		font=\footnotesize\bfseries\fontfamily{qag}\selectfont,minimum size=14pt,outer sep=0pt] (char) {#2};}}
\newcommand\circleTrue[2][\mycolor]{\tikz[baseline=(char.base)]{
		\node[shape=circle,inner sep=1pt,fill=#1!20,text=#1!80!black,
		font=\footnotesize\bfseries\fontfamily{qag}\selectfont,minimum size=14pt,outer sep=0pt] (char) {#2};}}
\usepackage{ex_test}
\usepackage{ifthen}
\renewcommand{\FalseEX}{\stepcounter{dapan}{{\circlenum{\Alph{dapan}}}}}
\renewcommand{\TrueEX}{\stepcounter{dapan}{{\circlenum{\Alph{dapan}}}}}
%%%================Lệnh hình phải================%%%
%%%================Lệnh immini================%%%
\def\loaicau{}
\def\tieudehinh{}
\newlength{\widthPB}
\newboolean{TNdungsai}
\newcommand{\hinhphai}[2]{%
	\savebox{\imbox}{#2}%
	\tcbsidebyside[
	sidebyside adapt=right,
	blanker,sidebyside gap=5mm,
	sidebyside align=top seam,
	]{%
		{\tieudehinh}%
		#1
	}{%
		\usebox{\imbox}%
	}%
	%	\xdef\loaicau{\setboolean{TNdungsai}{false}}%
}

\renewcommand{\immini}[3][]{
	\savebox{\imbox}{#3}
	\setlength{\widthPB}{0.55\linewidth}
	\tcbsidebyside[
	sidebyside adapt=right,
	blanker,sidebyside gap=5mm,
	sidebyside align=top seam,
	]{%
		{\tieudehinh}#2
	}{%
		\usebox{\imbox}
	}
}


%%%%%%%%%%%%%%%%%%%%%%Tách lời giải%%%%%%%%%%%%%%%%%%%%%%%%%%%%%%%%%%%%%%%%
\Newassociation{giaibth}{loigiaibth}{ansbth}
\Newassociation{giaibt}{loigiaibt}{ansbt}
\Newassociation{giaiex}{loigiaiex}{ansex}
\NewTColorBox{ansbt}{m}{
	breakable,
	enhanced,
	outer arc=0pt,
	arc=0pt,
	colframe=white,
	frame hidden,
	left=-6pt,right=0pt,top=0pt,
	colback=white,
	attach boxed title to top left,
	boxed title style={
		colback=white,
		outer arc=0pt,
		arc=0pt,
		top=1pt,
		bottom=1pt,
		left=3pt,
		right=3pt,
		colframe=white
	},
	fonttitle=\bfseries\sffamily\selectfont\color{white},
	title={HDBT~#1},
	overlay unbroken and first={
		\path (title.north west) coordinate (A)
		($ (title.south west) +(-4pt,0pt)$) coordinate (B)
		(title.south east) coordinate (C)
		($ (title.north east)+(4pt,0) $) coordinate (D);
		\path[rounded corners=2pt,fill=\mycolor,
		preaction={transform canvas={shift={(-45:2pt)}},left color=\mycolor!45,right color=\mycolor!25}] 
		(A)--(B)--(C)--(D)--cycle;
		\path (A)--(C) node[midway,font=\color{white}\bfseries\sffamily\selectfont](Bai){\textsl{HDBT~#1}};
		\draw[rounded corners=2pt,thick,\mycolor] ([xshift=-3pt]B) coordinate (Bt)
		--([shift={(-2pt,2pt)}]A)--+(\linewidth,0) coordinate (Ct);
		\fill[\mycolor] (Bt) circle (1pt) (Ct) circle (2pt);
	}
}

\NewTColorBox{ansex}{m}{
	breakable,
	enhanced,
	outer arc=0pt,
	arc=0pt,
	colframe=white,
	frame hidden,
	left=-6pt,right=0pt,top=0pt,
	colback=white,
	attach boxed title to top left,
	boxed title style={
		colback=white,
		outer arc=0pt,
		arc=0pt,
		top=1pt,
		bottom=1pt,
		left=3pt,
		right=3pt,
		colframe=white
	},
	fonttitle=\bfseries\sffamily\selectfont\color{white},
	title={HD Câu~#1},
	overlay unbroken and first={
		\path (title.north west) coordinate (A)
		($ (title.south west) +(-4pt,0pt)$) coordinate (B)
		(title.south east) coordinate (C)
		($ (title.north east)+(4pt,0) $) coordinate (D);
		\path[rounded corners=2pt,fill=\mycolor,
		preaction={transform canvas={shift={(-45:2pt)}},left color=\mycolor!45,right color=\mycolor!25}] 
		(A)--(B)--(C)--(D)--cycle;
		\path (A)--(C) node[midway,font=\color{white}\bfseries\sffamily\selectfont](Bai){\textsl{HD Câu~#1}};
		\draw[rounded corners=2pt,thick,\mycolor] ([xshift=-3pt]B) coordinate (Bt)
		--([shift={(-2pt,2pt)}]A)--+(\linewidth,0) coordinate (Ct);
		\fill[\mycolor] (Bt) circle (1pt) (Ct) circle (2pt);
	}
}
\def\ansexhead{\begin{ansex}}
	\def\ansexend{\end{ansex}}
\renewenvironment{loigiaibth}[1]{\begin{ansbt}{#1}}{\end{ansbt}}
\renewenvironment{loigiaiex}[1]{\ansexhead{#1}}{\ansexend}
\def\luuloigiaibt{
	\AtBeginEnvironment{bt}{
		\renewcommand{\loigiai}[1]{%
		\end{btbox}
		\def\btend{}
		\scantokens{%
			\begin{giaibth}
				##1
		\end{giaibth}}
	}
}
}

%%%%%%%%%%%%%%%%%%%%%%%%%%%%%%%%HỘP CÂU CÓ 3 TÙY CHỌN%%%%%%%%%%%%%%%%%%%%%%%%%%%%%
\NewTColorBox{exbox}{+!O{}O{}O{}}{%
enhanced,
breakable,
toprule at break=-\tcboxedtitleheight,
fonttitle=\bfseries\sffamily\color{white},
title={\faClockO\ Câu~\theex},
code={\refstepcounter{ex}}, 
empty,opacityback=0,
colback=white,
colbacktitle=white,
fonttitle=\bfseries,
coltitle=black,
attach boxed title to top left=
{yshift=-2mm-\tcboxedtitleheight,yshifttext=2mm-\tcboxedtitleheight/2},
boxed title style={
frame hidden,
outer arc=0pt,
arc=0pt,
bottom=3pt,
left=0pt,
right=0pt
},
overlay unbroken ={
\path[draw=\mycolor,rounded corners=5pt,thick] (frame.north west) rectangle (frame.south east);
\path
($ (title.north west) +(-2pt,0pt)$) coordinate (A)
($ (title.south west) +(-2pt,3pt)$) coordinate (B)
($ (title.south east)+(3pt,3pt) $) coordinate (C)
($ (title.north east)+(3pt,0) $) coordinate (D)
($ (A)+(0.99*\linewidth,0) $) coordinate (E)
;
\path[fill=\mycolor!15,rounded corners=3pt]
(A)--(B)--([xshift=0.85*\linewidth]C) coordinate (E)--([xshift=0.85*\linewidth]D) coordinate (F)--cycle;
\path[fill=\mycolor!70!black,rounded corners=2pt]
(A)--(B)--([xshift=3pt]C)--([xshift=3pt]D)--cycle;
\path[left color=\mycolor,right color=\mycolor,rounded corners=3pt]
([xshift=-2pt]A)--([xshift=-2pt]B)--(C)--(D)--cycle;
\path ($ (A)!0.5!(B) +(0pt,0)$) node[anchor=west,font=\color{white}\bfseries\sffamily\selectfont]{{\faClockO\ Câu~\theex}};
\path ($ (C)!0.5!(D) +(9pt,0)$) node[anchor=west,font=\color{\mycolor}]{\taosao{#2}};
\path ($ (E)!0.5!(F) +(-9pt,0)$) node[anchor=east,font=\color{\mycolor!70!black}\itshape\bfseries]{#1};
\path ([shift={(-1pt,7pt)}]frame.south east) node[anchor=east,font=\fontsize{9.2 pt}{1pt}\selectfont\color{\mycolor!70!black}\itshape\bfseries]{#3};
},
overlay first={
\path[draw=\mycolor,rounded corners=5pt, thick] (frame.south west) coordinate (FSW)
--(frame.north west) coordinate (FNW)
-- (frame.north east) coordinate (FNE)
--(frame.south east) coordinate (FSE)
(FSW)--(FSE)
;
\path
($ (title.north west) +(-2pt,0pt)$) coordinate (A)
($ (title.south west) +(-2pt,3pt)$) coordinate (B)
($ (title.south east)+(3pt,3pt) $) coordinate (C)
($ (title.north east)+(3pt,0) $) coordinate (D)
($ (A)+(0.99*\linewidth,0) $) coordinate (E)
;
\path[fill=\mycolor!15,rounded corners=3pt]
(A)--(B)--([xshift=0.85*\linewidth]C) coordinate (E)--([xshift=0.85*\linewidth]D) coordinate (F)--cycle;
\path[fill=\mycolor!70!black,rounded corners=2pt]
(A)--(B)--([xshift=3pt]C)--([xshift=3pt]D)--cycle;
\path[left color=\mycolor,right color=\mycolor,rounded corners=3pt]
([xshift=-2pt]A)--([xshift=-2pt]B)--(C)--(D)--cycle;
\path ($ (A)!0.5!(B) +(0pt,0)$) node[anchor=west,font=\color{white}\bfseries\sffamily\selectfont]{{\faClockO\ Câu~\theex}};
\path ($ (C)!0.5!(D) +(9pt,0)$) node[anchor=west,font=\color{\mycolor}]{\taosao{#2}};
\path ($ (E)!0.5!(F) +(-9pt,0)$) node[anchor=east,font=\color{\mycolor!70!black}\itshape\bfseries]{#1};
},
overlay middle ={
\path[draw=\mycolor,thick] (frame.north west) rectangle (frame.south east);
},
overlay  last ={
\path[draw=\mycolor,rounded corners=5pt, thick] (frame.north west) coordinate (FNW)-- (frame.south west) coordinate (FSW)
--(frame.south east) coordinate (FSE)--(frame.north east) coordinate (FNE)
(FNW)--(FNE)
;
\path ([shift={(-1pt,7pt)}]frame.south east) node[anchor=east,font=\fontsize{9.2 pt}{1pt}\selectfont\color{\mycolor!70!black}\itshape\bfseries]{#3};
},
top=\tcboxedtitleheight
}
\def\exhead#1#2#3{\begin{exbox}[#1][#2][#3]}
\def\exend{\end{exbox}}
\RenewDocumentEnvironment{ex}{+!O{}O{}O{}}{
\global\setbool{TNdungsai}{false}
\ifblank{#1}{\def\tuychonone{}}{\def\tuychonone{#1}}
\ifblank{#2}{\def\tuychontwo{0}}{\def\tuychontwo{#2}}
\ifblank{#3}{\def\tuychonthree{}}{\def\tuychonthree{#3}}
\exhead{\tuychonone}{\tuychontwo}{\tuychonthree}
}{\gdef\tieudehinh{}\ignorespacesafterend \exend}
\AtBeginEnvironment{ex}{%
	\setcounter{numTrue}{0}%
	\setcounter{numTrueSol}{0}%
	\global\setbool{TNdungsai}{false}%
	\def\dotEX{.}%
	\setcounter{numChoice}{0}%
	\renewcommand{\loigiai}[1]{	%
		\end{exbox}\def\exend{}\gdef\tieudehinh{}\par\noindent%
		{\color{\maunhan!85!black}\reflectbox{\Large\WritingHand}\ {\fmmfamily\LARGE Lời giải.}} #1
		\ifthenelse{\value{numTrueSol}>0}{
			\hfill \textcolor{\maunhan}{\faKey~\circlenum[\maunhan]{\Alph{numTrueSol}}}
		}{}
	}%}
	} 
%%%%%%%%%%%%%%%%%%%%%%%%%%%%%%%HỘP Bai Tap CÓ 3 TÙY CHỌN%%%%%%%%%%%%%%%%%%%%%%%%%%%%%
\newcounter{bt}
\NewTColorBox{btbox}{+!O{}O{}O{}}{%
enhanced,
breakable,
toprule at break=-\tcboxedtitleheight,
fonttitle=\bfseries\sffamily\color{white},
title={\faClockO\ Bài~\thebt},
code={\refstepcounter{bt}},
empty,opacityback=0,
colback=white,
colbacktitle=white,
fonttitle=\bfseries,
coltitle=black,
attach boxed title to top left=
{yshift=-2mm-\tcboxedtitleheight,yshifttext=2mm-\tcboxedtitleheight/2},
boxed title style={
frame hidden,
outer arc=0pt,
arc=0pt,
top=3pt,
bottom=3pt,
left=0pt,
right=0pt
},
overlay unbroken ={
\path[draw=\mycolor,rounded corners=5pt, thick] (frame.north west) rectangle (frame.south east);
\path
($ (title.north west) +(-2pt,0pt)$) coordinate (A)
($ (title.south west) +(-2pt,3pt)$) coordinate (B)
($ (title.south east)+(3pt,3pt) $) coordinate (C)
($ (title.north east)+(3pt,0) $) coordinate (D)
($ (A)+(0.99*\linewidth,0) $) coordinate (E)
;
\path[fill=\mycolor!15,rounded corners=3pt]
(A)--(B)--([xshift=0.85*\linewidth]C) coordinate (E)--([xshift=0.85*\linewidth]D) coordinate (F)--cycle;
\path[fill=\mycolor!70!black,rounded corners=2pt]
(A)--(B)--([xshift=3pt]C)--([xshift=3pt]D)--cycle;
\path[left color=\mycolor,right color=\mycolor,rounded corners=3pt]
([xshift=-2pt]A)--([xshift=-2pt]B)--(C)--(D)--cycle;
\path ($ (A)!0.5!(B) +(0pt,0)$) node[anchor=west,font=\color{white}\bfseries\sffamily\selectfont]{{\faClockO\ Bài~\thebt}};
\path ($ (C)!0.5!(D) +(9pt,0)$) node[anchor=west,font=\color{\mycolor}]{\taosao{#2}};
\path ($ (E)!0.5!(F) +(-9pt,0)$) node[anchor=east,font=\color{\mycolor!70!black}\itshape\bfseries]{#1};
\path ([shift={(-1pt,7pt)}]frame.south east) node[anchor=east,font=\fontsize{9.2 pt}{1pt}\selectfont\color{\mycolor!70!black}\itshape\bfseries]{#3};
},
overlay first  ={
\path[draw=\mycolor,rounded corners=5pt, thick] (frame.south west) coordinate (FSW)
--(frame.north west) coordinate (FNW)
-- (frame.north east) coordinate (FNE)
--(frame.south east) coordinate (FSE)
(FSW)--(FSE)
;
\path
($ (title.north west) +(-2pt,0pt)$) coordinate (A)
($ (title.south west) +(-2pt,3pt)$) coordinate (B)
($ (title.south east)+(3pt,3pt) $) coordinate (C)
($ (title.north east)+(3pt,0) $) coordinate (D)
($ (A)+(0.99*\linewidth,0) $) coordinate (E)
;
\path[fill=\mycolor!15,rounded corners=3pt]
(A)--(B)--([xshift=0.85*\linewidth]C) coordinate (E)--([xshift=0.85*\linewidth]D) coordinate (F)--cycle;
\path[fill=\mycolor!70!black,rounded corners=2pt]
(A)--(B)--([xshift=3pt]C)--([xshift=3pt]D)--cycle;
\path[left color=\mycolor,right color=\mycolor,rounded corners=3pt]
([xshift=-2pt]A)--([xshift=-2pt]B)--(C)--(D)--cycle;
\path ($ (A)!0.5!(B) +(0pt,0)$) node[anchor=west,font=\color{white}\bfseries\sffamily\selectfont]{{\faClockO\ Bài~\thebt}};
\path ($ (C)!0.5!(D) +(9pt,0)$) node[anchor=west,font=\color{\mycolor}]{\taosao{#2}};
\path ($ (E)!0.5!(F) +(-9pt,0)$) node[anchor=east,font=\color{\mycolor!70!black}\itshape\bfseries]{#1};
},
overlay middle ={
\path[draw=\mycolor,thick] (frame.north west) rectangle (frame.south east);
},
overlay last ={
\path[draw=\mycolor,rounded corners=5pt, thick] (frame.north west) coordinate (FNW)-- (frame.south west) coordinate (FSW)
--(frame.south east) coordinate (FSE)--(frame.north east) coordinate (FNE)
(FNW)--(FNE)
;
\path ([shift={(-1pt,7pt)}]frame.south east) node[anchor=east,font=\fontsize{9.2 pt}{1pt}\selectfont\color{\mycolor!70!black}\itshape\bfseries]{#3};
},
top=\tcboxedtitleheight
}
\def\bthead#1#2#3{\begin{btbox}[#1][#2][#3]}
\def\btend{\end{btbox}}
\NewDocumentEnvironment{bt}{+!O{}O{}O{}}{
\ifblank{#1}{\def\tuychonone{}}{\def\tuychonone{#1}}
\ifblank{#2}{\def\tuychontwo{0}}{\def\tuychontwo{#2}}
\ifblank{#3}{\def\tuychonthree{}}{\def\tuychonthree{#3}}
\bthead{\tuychonone}{\tuychontwo}{\tuychonthree}
}{\btend}
\AtBeginEnvironment{bt}{%
	\renewcommand{\FalseEX}{\stepcounter{dapan}{\circlenum{\Alph{dapan}}}}
	\renewcommand{\TrueEX}{\stepcounter{dapan}{\circlenum{\Alph{dapan}}}}
	\setcounter{numTrue}{0}%
	\setcounter{numTrueSol}{0}%
	\renewcommand{\loigiai}[1]{%
		\par\noindent	
		\end{btbox}
		\def\btend{}
		\begin{center}
			\par\noindent%
			{\color{\mycolor!50!black}\reflectbox{\Large\WritingHand}\ {\fmmfamily\LARGE \textbf{Hướng dẫn giải:}}} 
		\end{center}
		#1
		\ifthenelse{\value{numTrueSol}>0}{
		\phantom{a}\hfill\textcolor{\mycolor!50!black}{\bfseries\sffamily \faKey~\circleTrue[\maunhan]{\Alph{numTrueSol}}}
		}{}
	}
}

%%====================Hộp Ví dụ có 3 tùy chọn==========================%%%
\newcounter{vd}
\NewTColorBox{vdbox}{+!O{}O{}O{}}{%
enhanced,
breakable,
left=6pt,
toprule at break=-\tcboxedtitleheight,
fonttitle=\bfseries\sffamily\color{white},
title={\faClockO\ Ví dụ~\thevd},
colback=white,
empty,
opacityback=0,
colbacktitle=white,
fonttitle=\bfseries,
coltitle=black,
code={\refstepcounter{vd}},
attach boxed title to top left=
{yshift=-2mm-\tcboxedtitleheight,yshifttext=2mm-\tcboxedtitleheight/2},
boxed title style={
frame hidden,
outer arc=0pt,
arc=0pt,
top=2pt,
bottom=2pt,
left=0pt,
right=0pt
},
overlay unbroken ={
\path[draw=\mycolor,rounded corners=5pt,ultra thick] (frame.north west) rectangle (frame.south east);
\path
($ (title.north west) +(-2pt,0pt)$) coordinate (A)
($ (title.south west) +(-2pt,3pt)$) coordinate (B)
($ (title.south east)+(3pt,3pt) $) coordinate (C)
($ (title.north east)+(3pt,0) $) coordinate (D)
($ (A)+(0.99*\linewidth,0) $) coordinate (E)
;
\path[fill=\mycolor!15,rounded corners=3pt]
(A)--(B)--([xshift=0.86*\linewidth]C) coordinate (E)--([xshift=0.86*\linewidth]D) coordinate (F)--cycle;
\path[fill=\mycolor!70!black,rounded corners=2pt]
(A)--(B)--([xshift=3pt]C)--([xshift=3pt]D)--cycle;
\path[left color=\mycolor,right color=\mycolor,rounded corners=3pt]
([xshift=-2pt]A)--([xshift=-2pt]B)--(C)--(D)--cycle;
\path ($ (A)!0.5!(B) +(0pt,0)$) node[anchor=west,font=\color{white}\bfseries\sffamily\selectfont]{{\faClockO\ Ví dụ~\thevd}};
\path ($ (C)!0.5!(D) +(9pt,0)$) node[anchor=west,font=\color{\mycolor}]{\taosao{#2}};
\path ($ (E)!0.5!(F) +(-9pt,0)$) node[anchor=east,font=\color{\mycolor!70!black}\itshape\bfseries]{#1};
\path ([shift={(-1pt,7pt)}]frame.south east) node[anchor=east,font=\fontsize{9.2 pt}{1pt}\selectfont\color{\mycolor!70!black}\itshape\bfseries]{#3};
},
overlay  first={
\path[draw=\mycolor,rounded corners=5pt, thick] (frame.south west) coordinate (FSW)
--(frame.north west) coordinate (FNW)
-- (frame.north east) coordinate (FNE)
--(frame.south east) coordinate (FSE)
(FSW)--(FSE)
;
\path
($ (title.north west) +(-2pt,0pt)$) coordinate (A)
($ (title.south west) +(-2pt,3pt)$) coordinate (B)
($ (title.south east)+(3pt,3pt) $) coordinate (C)
($ (title.north east)+(3pt,0) $) coordinate (D)
($ (A)+(0.99*\linewidth,0) $) coordinate (E)
;
\path[fill=\mycolor!15,rounded corners=3pt]
(A)--(B)--([xshift=0.86*\linewidth]C) coordinate (E)--([xshift=0.86*\linewidth]D) coordinate (F)--cycle;
\path[fill=\mycolor!70!black,rounded corners=2pt]
(A)--(B)--([xshift=3pt]C)--([xshift=3pt]D)--cycle;
\path[left color=\mycolor,right color=\mycolor,rounded corners=3pt]
([xshift=-2pt]A)--([xshift=-2pt]B)--(C)--(D)--cycle;
\path ($ (A)!0.5!(B) +(0pt,0)$) node[anchor=west,font=\color{white}\bfseries\sffamily\selectfont]{{\faClockO\ Ví dụ~\thevd}};
\path ($ (C)!0.5!(D) +(9pt,0)$) node[anchor=west,font=\color{\mycolor}]{\taosao{#2}};
\path ($ (E)!0.5!(F) +(-9pt,0)$) node[anchor=east,font=\color{\mycolor!70!black}\itshape\bfseries]{#1};
},
overlay middle ={
\path[draw=\mycolor, thick] (frame.north west) rectangle (frame.south east);
},
overlay last ={
\path[draw=\mycolor,rounded corners=5pt, thick] (frame.north west) coordinate (FNW)-- (frame.south west) coordinate (FSW)
--(frame.south east) coordinate (FSE)--(frame.north east) coordinate (FNE)
(FNW)--(FNE)
;
\path ([shift={(-1pt,7pt)}]frame.south east) node[anchor=east,font=\fontsize{9.2 pt}{1pt}\selectfont\color{\mycolor!70!black}\itshape\bfseries]{#3};
},
top=\tcboxedtitleheight
}
\def\vdhead#1#2#3{\begin{vdbox}[#1][#2][#3]}
\def\vdend{\end{vdbox}}
\NewDocumentEnvironment{vd}{+!O{}O{}O{}}{%
\ifblank{#1}{\def\tuychonone{}}{\def\tuychonone{#1}}
\ifblank{#2}{\def\tuychontwo{0}}{\def\tuychontwo{#2}}
\ifblank{#3}{\def\tuychonthree{}}{\def\tuychonthree{#3}}
\vdhead{\tuychonone}{\tuychontwo}{\tuychonthree}
}{\vdend}
\AtBeginEnvironment{vd}{%
	\renewcommand{\FalseEX}{\stepcounter{dapan}{\circlenum{\Alph{dapan}}}}
	\renewcommand{\TrueEX}{\stepcounter{dapan}{\circlenum{\Alph{dapan}}}}
	\setcounter{numTrue}{0}%
	\setcounter{numTrueSol}{0}%
	\renewcommand{\loigiai}[1]{%
		\end{vdbox}
		\def\vdend{}
		\begin{center}
			\par\noindent%
			{\color{\mycolor!50!black}\reflectbox{\Large\WritingHand}\ {\fmmfamily\LARGE \textbf{Hướng dẫn giải:}}} 
		\end{center}
		#1
		\ifthenelse{\value{numTrueSol}>0}{
		\phantom{a}\hfill\textcolor{\mycolor!50!black}{\bfseries\sffamily \faKey~\circleTrue[\maunhan]{\Alph{numTrueSol}}}
		}{}
	}
}

%%%==================Hộp vdex 3 tùy chọn==================%%%
\NewTColorBox{vdexbox}{+!O{}O{}O{}}{%
enhanced,
breakable,
left=6pt,
toprule at break=-\tcboxedtitleheight,
fonttitle=\bfseries\sffamily\color{white},
title={\faClockO\ Ví dụ~\thevd},
empty,opacityback=0,
colback=white,
colbacktitle=white,
fonttitle=\bfseries,
coltitle=black,
attach boxed title to top left=
{yshift=-2mm-\tcboxedtitleheight,yshifttext=2mm-\tcboxedtitleheight/2},
boxed title style={
frame hidden,
outer arc=0pt,
arc=0pt,
top=3pt,
bottom=3pt,
left=0pt,
right=0pt
},
overlay unbroken ={
\path[draw=\mycolor,rounded corners=5pt, thick] (frame.north west) rectangle (frame.south east);
\path
($ (title.north west) +(-2pt,0pt)$) coordinate (A)
($ (title.south west) +(-2pt,3pt)$) coordinate (B)
($ (title.south east)+(3pt,3pt) $) coordinate (C)
($ (title.north east)+(3pt,0) $) coordinate (D)
($ (A)+(0.99*\linewidth,0) $) coordinate (E)
;
\path[fill=\mycolor!15,rounded corners=3pt]
(A)--(B)--([xshift=0.85*\linewidth]C) coordinate (E)--([xshift=0.85*\linewidth]D) coordinate (F)--cycle;
\path[fill=\mycolor!70!black,rounded corners=2pt]
(A)--(B)--([xshift=3pt]C)--([xshift=3pt]D)--cycle;
\path[left color=\mycolor,right color=\mycolor,rounded corners=3pt]
([xshift=-2pt]A)--([xshift=-2pt]B)--(C)--(D)--cycle;
\path ($ (A)!0.5!(B) +(0pt,0)$) node[anchor=west,font=\color{white}\bfseries\sffamily\selectfont]{{\faClockO\ Ví dụ~\thevd}};
\path ($ (C)!0.5!(D) +(9pt,0)$) node[anchor=west,font=\color{\mycolor}]{\taosao{#2}};
\path ($ (E)!0.5!(F) +(-9pt,0)$) node[anchor=east,font=\color{\mycolor!70!black}\itshape\bfseries]{#1};
\path ([shift={(-1pt,7pt)}]frame.south east) node[anchor=east,font=\fontsize{9.2 pt}{1pt}\selectfont\color{\mycolor!70!black}\itshape\bfseries]{#3};
},
overlay first={
\path[draw=\mycolor,rounded corners=5pt, thick] (frame.south west) coordinate (FSW)
--(frame.north west) coordinate (FNW)
-- (frame.north east) coordinate (FNE)
--(frame.south east) coordinate (FSE)
(FSW)--(FSE)
;
\path
($ (title.north west) +(-2pt,0pt)$) coordinate (A)
($ (title.south west) +(-2pt,3pt)$) coordinate (B)
($ (title.south east)+(3pt,3pt) $) coordinate (C)
($ (title.north east)+(3pt,0) $) coordinate (D)
($ (A)+(0.99*\linewidth,0) $) coordinate (E)
;
\path[fill=\mycolor!15,rounded corners=3pt]
(A)--(B)--([xshift=0.85*\linewidth]C) coordinate (E)--([xshift=0.85*\linewidth]D) coordinate (F)--cycle;
\path[fill=\mycolor!70!black,rounded corners=2pt]
(A)--(B)--([xshift=3pt]C)--([xshift=3pt]D)--cycle;
\path[left color=\mycolor,right color=\mycolor,rounded corners=3pt]
([xshift=-2pt]A)--([xshift=-2pt]B)--(C)--(D)--cycle;
\path ($ (A)!0.5!(B) +(0pt,0)$) node[anchor=west,font=\color{white}\bfseries\sffamily\selectfont]{{\faClockO\ Ví dụ~\thevd}};
\path ($ (C)!0.5!(D) +(9pt,0)$) node[anchor=west,font=\color{\mycolor}]{\taosao{#2}};
\path ($ (E)!0.5!(F) +(-9pt,0)$) node[anchor=east,font=\color{\mycolor!70!black}\itshape\bfseries]{#1};
},
overlay middle ={
frame code={yshifttext=\tcboxedtitleheight,
\path[draw=\mycolor,thick] (frame.north west) rectangle (frame.south east);
}
},
overlay last ={
\path[draw=\mycolor,rounded corners=5pt, thick] (frame.north west) coordinate (FNW)-- (frame.south west) coordinate (FSW)
--(frame.south east) coordinate (FSE)--(frame.north east) coordinate (FNE)
(FNW)--(FNE)
;
\path ([shift={(-1pt,7pt)}]frame.south east) node[anchor=east,font=\fontsize{9.2 pt}{1pt}\selectfont\color{\mycolor!70!black}\itshape\bfseries]{#3};
},
top=\tcboxedtitleheight
}
\def\vdexhead#1#2#3{\begin{vdexbox}[#1][#2][#3]}
\def\vdexend{\end{vdexbox}}
\NewDocumentEnvironment{vdex}{+!O{}O{}O{}}{
\ifblank{#1}{\def\tuychonone{}}{\def\tuychonone{#1}}
\ifblank{#2}{\def\tuychontwo{0}}{\def\tuychontwo{#2}}
\ifblank{#3}{\def\tuychonthree{}}{\def\tuychonthree{#3}}
\vdexhead{\tuychonone}{\tuychontwo}{\tuychonthree}
}{\vdexend}
\AtBeginEnvironment{vdex}{
	\renewcommand{\FalseEX}{\stepcounter{dapan}{\circlenum{\Alph{dapan}}}}
	\renewcommand{\TrueEX}{\stepcounter{dapan}{\circlenum{\Alph{dapan}}}}
	\setcounter{numTrue}{0}%
	\setcounter{numTrueSol}{0}%
	\renewcommand{\loigiai}[1]{
		\end{vdexbox}
		\def\vdexend{}
		\begin{center}
		{\color{\mycolor}\reflectbox{\Large\WritingHand}\ {\fmmfamily\LARGE Lời giải:}}
		\end{center}
		#1
		\ifthenelse{\value{numTrueSol}>0}{
			\phantom{a}\hfill\textcolor{\mycolor!50!black}{\bfseries\sffamily \faKey~\circleTrue{\Alph{numTrueSol}}}
		}{}
	}
}
%%%=======================Một số lệnh mới================================%%%
%%%===============Câu trả lời ngắn=======================%%%
\tikzset{
	master/.style={
		execute at end picture={
			\coordinate (lower right) at (current bounding box.south east);
			\coordinate (upper left) at (current bounding box.north west);
		}
	},
	slave/.style={
		execute at end picture={
			\pgfresetboundingbox
			\path (upper left) rectangle (lower right);
		}
	}
}

\newcommand{\ovuong}[2][2.5]{%
	\scantokens{%
		\begin{giaibt}
			#2
	\end{giaibt}}% 
	{\linepenalty100\exhyphenpenalty0
		\hfill{\tikz{%
				\path (0,-0.15) node (char) {};
				\draw[rounded corners=2pt,thick,\maunhan] (0,-0.275) rectangle (#1,0.275);
			}\hspace*{-9pt}%
		}%
}}

\newcommand{\shortans}[2][2.5]{%
	\ovuong{#2}%
}

\RenewEnviron{loigiaibt}[1]{%
	\tikz{\node[rounded corners,draw=\maunhan,text width=4cm,align=center,minimum height=19pt] 
		{\textbf{Bài~#1.} {\boldmath\textbf{\color{\maunhan}\BODY}}}%
	}
}%

%%%#####==================Tùy chỉnh lời giải ========================########%%%

%%%================hiển thị lời giải trắc nghiệm================%%%
\newcommand{\LGexTF}{%
	\AtBeginEnvironment{ex}{%
		\global\setbool{TNdungsai}{false}%
		\xdef\maudapanT{\mauphu!50!green}%
		\xdef\maudapanF{\maunhan}%
		\renewcommand{\loigiai}[1]{%
			\par\noindent%
			{\reflectbox{\color{\maunhan}\Large\WritingHand}\ {\color{\maunhan}\fontfamily{qzc}\selectfont Lời giải.}} ##1 \hfill{{\color{\maunhan!85!black}\faKey}~ 
				\ifbool{Ads}{\TLdung{A}}{\TLsai{A}}~
				\ifbool{Bds}{\TLdung{B}}{\TLsai{B}}~
				\ifbool{Cds}{\TLdung{C}}{\TLsai{C}}~
				\ifbool{Dds}{\TLdung{D}}{\TLsai{D}} 
			}%
		}
	}
}
%\gdef\tieudehinh{}\par\noindent%
%{\color{\maunhan!85!black}\reflectbox{\Large\WritingHand}\ {\fmmfamily\LARGE Lời giải.}}
%\fontfamily{qzc}\selectfont Lời giải.
%%\fmmfamily\LARGE Lời giải.
\def\hienthiloigiai{
	\global\setbool{TNdungsai}{false}%
	\xdef\maudapanT{\mauphu!50!green}%
	\xdef\maudapanF{\maunhan}%
	\AtBeginEnvironment{ex}{%
		\renewcommand{\loigiaiTF}{%
			\renewcommand{\loigiai}[1]{%
				\end{exbox}
				\def\exend{}
				\par\noindent%
				{\xdef\tieudehinh{}\par\noindent%
					{\color{\maunhan!85!black}\reflectbox{\Large\WritingHand}\ {\fmmfamily\LARGE Lời giải.}} ####1\hfill \textcolor{\maunhan}{\faKey}~\hspace*{-7pt}
					\ifbool{Ads}{\TLdung{A}}{\TLsai{A}}~\hspace*{-7pt}
					\ifbool{Bds}{\TLdung{B}}{\TLsai{B}}~\hspace*{-7pt}
					\ifbool{Cds}{\TLdung{C}}{\TLsai{C}}~\hspace*{-7pt}
					\ifbool{Dds}{\TLdung{D}}{\TLsai{D}}}
			}%				
		}
		\renewcommand{\shortans}[2][]{\phantom{.}\hfill{\color{\maunhan!85!black}\raisebox{6pt}{\faKey}}\;\khungB{##2}}
		\ifbool{TNdungsai}{%
			\renewcommand{\loigiai}[1]{%
				\end{exbox}
				\def\exend{}
				\par\noindent%
				{\xdef\tieudehinh{}\par\noindent%
					{\color{\maunhan}\reflectbox{\Large\WritingHand}\ {\fmmfamily\LARGE Lời giải.}} ##1\hfill \textcolor{\maunhan}{\faKey}~\hspace*{-7pt}
					\ifbool{Ads}{\TLdung{A}}{\TLsai{A}}~\hspace*{-7pt}
					\ifbool{Bds}{\TLdung{B}}{\TLsai{B}}~\hspace*{-7pt}
					\ifbool{Cds}{\TLdung{C}}{\TLsai{C}}~\hspace*{-7pt}
					\ifbool{Dds}{\TLdung{D}}{\TLsai{D}}}
			}%
		}{%
			\renewcommand{\loigiai}[1]{%
				\end{exbox}
				\def\exend{}
				\gdef\tieudehinh{}\par\noindent%
				{\color{\maunhan!85!black}\reflectbox{\Large\WritingHand}\ {\fmmfamily\LARGE Lời giải.}}\par\noindent\ignorespaces%
				##1
				\ifthenelse{\value{numTrueSol}>0}{%
					\hfill {\color{\maunhan}\faKey~\circlenum[\maunhan]{\Alph{numTrueSol}}}%
				}{}}%
		}%					
	}
	\AtBeginEnvironment{bt}{%
		\renewcommand{\shortans}[2][]{%
			\scantokens{%
				\begin{giaibt}
					##2
			\end{giaibt}}% 
			\phantom{.}\hfill{\color{\maunhan}\raisebox{6pt}{\faKey}}\;\khungB{##2}}%
		\renewcommand{\loigiai}[1]{%
			\end{btbox}
			\def\btend{}
			\par\noindent%
			{\xdef\tieudehinh{}\par\noindent%
				{\color{\maunhan}\reflectbox{\Large\WritingHand}\ {\fmmfamily\LARGE Lời giải.}}%
				\par\noindent\ignorespaces
				##1
			}%
		}%
	}
}
%%%=============lưu lời giải =============%%%
\def\luuloigiai{
	\xdef\maudapanT{\maunen}%
	\xdef\maudapanF{\maunen}%
	\AtBeginEnvironment{ex}{%
		\renewcommand{\loigiaiTF}{}
		\renewcommand{\loigiai}[1]{%
			\Writetofile{ansex}{\string\def\string\writeANS{\string\faKey\, \saveans}}
			\ifbool{TNdungsai}{}{\Writetofile{ansex}{\string\def\string\writeANS{\string\faKey\, \circlenum{\Alph{numTrueSol}}}}}
			\scantokens{%
				\begin{giaiex}
					##1
				\end{giaiex}
			}% 
		}
	}
	\AtBeginEnvironment{bt}{%
		\renewcommand{\loigiai}[1]{%
			\Writetofile{ansbth}{\string\def\string\writeANS{}}
			\scantokens{%
				\begin{giaibth}
					##1
				\end{giaibth}
			}% 
		}
	}
}

%%%================tắt lời giải trắc nghiệm================%%%
\def\tatloigiai{
	\xdef\maudapanT{\maunen}%
	\xdef\maudapanF{\maunen}%
	\AtBeginEnvironment{ex}{%
		\renewcommand{\loigiai}[1]{}
		\renewcommand{\loigiaiTF}{}
	}
	\AtBeginEnvironment{bt}{%
		\renewcommand{\loigiai}[1]{}
	}
	\AtBeginEnvironment{vdex}{%
		\renewcommand{\loigiai}[1]{}
	}
}
%%%=====================Đổi thành dòng kẻ=======================%%%

\def\dongkeloigiai{
	\xdef\maudapanT{\maunen}
	\xdef\maudapanF{\maunen}
	\AtBeginEnvironment{ex}{
		\renewcommand{\loigiaiTF}{}
		\renewcommand{\loigiai}[1]{%
			\DoiThanhDongKe{##1}
		}
	}
	\AtBeginEnvironment{bt}{
		\def\maudapan{white}
		\renewcommand{\loigiaiTF}{}
		\renewcommand{\loigiai}[1]{%
			\DoiThanhDongKe{##1}
		}
	}
	\AtBeginEnvironment{vd}{
		\def\maudapan{white}
		\renewcommand{\loigiaiTF}{}
		\renewcommand{\loigiai}[1]{%
			\DoiThanhDongKe{##1}
		}
	}
}
%%%=============================================%%%
\def\dongkeHaicot{
	\xdef\maudapanT{\maunen}
	\xdef\maudapanF{\maunen}
	\AtBeginEnvironment{ex}{
		\renewcommand{\loigiaiTF}{}
		\renewcommand{\loigiai}[1]{
			\begin{multicols}{2}			
				\DoiThanhDongKeH{##1}
			\end{multicols}%
		}
	}
	\AtBeginEnvironment{bt}{
		\renewcommand{\loigiaiTF}{}
		\renewcommand{\loigiai}[1]{
			\begin{multicols}{2}			
				\DoiThanhDongKeH{##1}
			\end{multicols}%
		}
	}	
}

\newcommand{\Pointilles}[2][1.1]{%
	\par\nobreak
	\noindent\rule{0pt}{1.1\baselineskip}%
	\foreach \i in {1,...,#2}{%
		\ifnum\i=1
		\noindent\makebox[\linewidth]{\rule{0pt}{#1\baselineskip}\reflectbox{\Large\WritingHand}\ {\fmmfamily\LARGE Bài làm:}\dotfill}\endgraf
		\else
		\noindent\makebox[\linewidth]{\rule{0pt}{#1\baselineskip}\dotfill}\endgraf
		\fi
	}
}

\newcommand{\DoiThanhDongKe}[1]{
	\setbox0=\vbox{\hbox{
			\noindent\begin{minipage}{\linewidth}%
				#1 aaaaa
			\end{minipage}
	}}
	\linepar=\ht0
	\pgfmathparse{int((\linepar-\fboxsep)/(\lineheight)+1)}
	\ifnum\pgfmathresult>3
	\noindent%
	\Pointilles{\pgfmathresult}
	\else
	\noindent%
	\Pointilles{3}
	\fi
}

\newcommand{\DoiThanhDongKeH}[1]{
	\setbox0=\vbox{\hbox{
			\noindent\begin{minipage}{\linewidth}%
				#1 aaaaa
			\end{minipage}
	}}
	\linepar=\ht0
	\pgfmathparse{int((\linepar-\fboxsep)/(0.85\lineheight)+1)}
	\ifnum\pgfmathresult>3
	\noindent%
	\Pointilles{\pgfmathresult}
	\else
	\noindent%
	\Pointilles{4}
	\fi
}
%%%==================Tạo ô li===========================%%%
\newcommand{\DoiThanhOly}[2][\mycolor]{
	\setbox0=\vbox{\hbox{
			\noindent\begin{minipage}{\linewidth}%
				#1 aaaaa
			\end{minipage}
	}}
	\linepar=\ht0
	\pgfmathparse{int((\linepar-\fboxsep)/(\lineheight)+1)}
	\let\mydong\pgfmathresult	\ifnum\pgfmathresult>3
	\begin{center}
		\foreach \i in {0,1,...,\mydong}{%
			\begin{tikzpicture}
				\draw[#1!25,ultra thin,step=0.2] (0,0) grid +(17,1);
				\draw[#1!65] (0,0) grid +(17,1);
			\end{tikzpicture}\\[-1.1pt]
		}
	\end{center}
	\else
	\noindent%
	\begin{center}
		\begin{tikzpicture}
			\draw[#1!25,ultra thin,step=0.2] (0,0) grid +(17,3);
			\draw[#1!65] (0,0) grid +(17,3);
		\end{tikzpicture}
	\end{center}
	\fi
}
\def\dongkeOly{
	\AtBeginEnvironment{ex}{%
		\renewcommand{\loigiai}[1]{%
			\DoiThanhOly{##1}%
		}
	}
	\AtBeginEnvironment{exsao}{
		\renewcommand{\loigiai}[1]{%
			\DoiThanhOly{##1}
		}
	}
	
	\AtBeginEnvironment{bt}{
		\renewcommand{\loigiai}[1]{
			\DoiThanhOly{##1}
		}
	}
	\AtBeginEnvironment{btsao}{
		\renewcommand{\loigiai}[1]{
			\DoiThanhOly{##1}
		}
	}
	
	\AtBeginEnvironment{vd}{
		\renewcommand{\loigiai}[1]{
			\DoiThanhOly{##1}
		}
	}
	\AtBeginEnvironment{vdsao}{
		\renewcommand{\loigiai}[1]{
			\DoiThanhOly{##1}
		}
	}
}

\def\luuloigiaiex{
	\AtBeginEnvironment{ex}{
		\setboolean{TNdungsai}{false}%
		\gdef\loaicau{\setboolean{TNdungsai}{false}}%
		\renewcommand{\loigiai}[1]{%
			\ifthenelse{\boolean{TNdungsai}}{%Câu hỏi trắc nghiệm đúng sai
				\scantokens{%
					\begin{giaiex}
						##1
						\phantom{a}\hfill{{\faKey}~ 
							\ifbool{Ads}{\TLdung{A}}{\TLsai{A}}~
							\ifbool{Bds}{\TLdung{B}}{\TLsai{B}}~
							\ifbool{Cds}{\TLdung{C}}{\TLsai{C}}~
							\ifbool{Dds}{\TLdung{D}}{\TLsai{D}} 
						}%
				\end{giaiex}}
				\Writetofile{ansbook}{\string\def\string\writeANS{\saveans}}
				\scantokens{
					\begin{solbook}
						\writeANS
					\end{solbook}
				}%
			}{%Câu hỏi trắc nghiệm 1 phương án
				\ifnum\the\value{numTrue}=1
				\scantokens{%
					\begin{giaiex}
						##1
						\phantom{a}\hfill{ \faKey~\circlenum{A}}
				\end{giaiex}}
				\fi%
				\ifnum\the\value{numTrue}=2
				\scantokens{%
					\begin{giaiex}
						##1
						\phantom{a}\hfill{ \faKey~\circlenum{B}}
				\end{giaiex}}
				\fi%	 
				\ifnum\the\value{numTrue}=3
				\scantokens{%
					\begin{giaiex}
						##1
						\phantom{a}\hfill{ \faKey~\circlenum{C}}
				\end{giaiex}}
				\fi %	 
				\ifnum\the\value{numTrue}=4
				\scantokens{%
					\begin{giaiex}
						##1
						\phantom{a}\hfill{ \faKey~\circlenum{D}}
				\end{giaiex}}
				\fi\vspace*{-3pt}%
			} 
		}
	}
}

%%%==============Tạo số dòng kẻ như mong muốn =====================%%%

\newcommand{\sodongke}[2][5]{%
	\AtBeginEnvironment{#2}{%
		\renewcommand{\loigiai}[1]{%
			\Pointilles{#1}
		}
	}
}

\newcommand{\sodongkeH}[2][5]{%
	\AtBeginEnvironment{#2}{%
		\renewcommand{\loigiai}[1]{%
		\end{#2box}%
		\expandafter\def\csname #2end\endcsname{}%
		\Pointilles{#1}%
	}%
}%
}


\usepackage{xstring}
\newcommand{\khungB}[1]{%
\StrSubstitute{#1}{$}{}[\chuoi] % Loại bỏ dấu $
\StrLen{\chuoi}[\strlen]%
\ifnum\strlen<4
\def\strlen{4}
\fi
\foreach \i in {1,...,\strlen} {%
	\StrChar{\chuoi}{\i}[\char]%
	\tikz\node[draw=teal!65,rectangle,minimum width=19pt,minimum height=19pt,rounded corners=1pt,text=white]{\char};%
}%
}

%%%===============trắc nghiệm đúng sai======================%%%
\usepackage{multicol}
\setlength{\columnseprule}{0pt}
\setlength{\columnsep}{1cm}
\setlength{\extrarowheight}{4pt}
\def\columnseprulecolor{\color{\maunhan}}
%%%%%%%%%%%%%
\usepackage{array}% dùng căn lề các mảng
\usepackage{multirow}% gop các dòng
\usepackage{makecell}% chỉnh định dạng các ô
\usepackage{booktabs}% Chèn các đường kẻ trên và dưới
\usepackage{diagbox}
\usepackage{longtable}
\renewcommand{\arraystretch}{1.2}% giãn độ rộng của dòng
\newcolumntype{L}[1]{>{\raggedright\let
	\newline\\\arraybackslash\hspace{0pt}}p{#1}}
	\newcolumntype{C}[1]{>{\centering\let
	\newline\\\arraybackslash\hspace{0pt}}p{#1}}
	\newcolumntype{R}[1]{>{\raggedleft\let
	\newline\\\arraybackslash\hspace{0pt}}p{#1}}
	\setlength{\tabcolsep}{5pt} % giãn độ rộng của cột
	\setlength{\aboverulesep}{4pt} 
	\setlength{\belowrulesep}{4pt}
	\arrayrulecolor{\mycolor!80!black} 
	\setlength\arrayrulewidth{1pt}
	
	\newbool{Ads}
	\newbool{Bds}
	\newbool{Cds}
	\newbool{Dds}
	\makeatletter
	\newcommand{\KiemtraA}{\@ifnextchar\True{\global\setbool{Ads}{true}\xdef\saveans{\saveans\string\TLdung{A}}}{\global\setbool{Ads}{false}\xdef\saveans{\saveans\string\TLsai{A}}}}
	\newcommand{\KiemtraB}{\@ifnextchar\True{\global\setbool{Bds}{true}\xdef\saveans{\saveans\string\TLdung{B}}}{\global\setbool{Bds}{false}\xdef\saveans{\saveans\string\TLsai{B}}}}
	\newcommand{\KiemtraC}{\@ifnextchar\True{\global\setbool{Cds}{true}\xdef\saveans{\saveans\string\TLdung{C}}}{\global\setbool{Cds}{false}\xdef\saveans{\saveans\string\TLsai{C}}}}
	\newcommand{\KiemtraD}{\@ifnextchar\True{\global\setbool{Dds}{true}\xdef\saveans{\saveans\string\TLdung{D}}}{\global\setbool{Dds}{false}\xdef\saveans{\saveans\string\TLsai{D}}}}
	\makeatother
	
	
	\def\saveans{}
	\newlength{\widthNPB}
	\setlength{\widthPB}{1.07\linewidth}
	\setlength{\widthNPB}{0.53\linewidth-1.1cm}
	\setlength\LTpre{5pt}% khoảng cách trước longtable
	\setlength\LTpost{3pt}
	\def\maudapanT{\mauphu!50!green}
	\def\maudapanF{\maunhan}
	\newcommand{\choiceTF}[5][1]{%
\def\saveans{}%
\global\setbool{TNdungsai}{true}%
\loigiaiTF
\ifthenelse{\equal{#1}{1}}{%
	\begin{enumerate}[wide=\parindent,label*=\protect{\alph*)},itemsep=-3pt,topsep=-4pt]
		\item \KiemtraA#2\dotEX	
		\item \KiemtraB#3\dotEX
		\item \KiemtraC#4\dotEX
		\item \KiemtraD#5\dotEX
	\end{enumerate}
}{
	\ifthenelse{\equal{#1}{t}}{%
		\begin{longtable}{|p{\widthPB}|C{0.45cm}|C{.45cm}|}
			\hline \rowcolor{\mycolor!20}
			\thead{\normalsize\sffamily\bfseries\fontfamily{qag}\selectfont Phát biểu}& {\bfseries\fontfamily{qag}\selectfont Đ} & {\bfseries\fontfamily{qag}\selectfont S}\\
			\hline
			{a)} \KiemtraA#2\dotEX &\ifbool{Ads}{{\color{\maudapanT}\footnotesize\faCheck}}{} &\ifbool{Ads}{}{{\color{\maudapanF}\footnotesize\faClose}}  \\
			\hline
			{b)} \KiemtraB#3\dotEX & \ifbool{Bds}{{\color{\maudapanT}\footnotesize\faCheck}}{} & \ifbool{Bds}{}{{\color{\maudapanF}\footnotesize\faClose}}\\
			\hline
			{c)} \KiemtraC#4\dotEX &\ifbool{Cds}{{\color{\maudapanT}\footnotesize\faCheck}}{} &\ifbool{Cds}{}{{\color{\maudapanF}\footnotesize\faClose}} \\
			\hline
			{d)} \KiemtraD#5\dotEX & \ifbool{Dds}{{\color{\maudapanT}\footnotesize\faCheck}}{} & \ifbool{Dds}{}{{\color{\maudapanF}\footnotesize\faClose}} \\
			\hline
		\end{longtable}
	}{%
		\ifthenelse{\equal{#1}{tt}}{%
			\begin{center}
				\vspace*{-0.65\baselineskip}
				\begin{tabular}{cc}%
					\begin{tabular}[t]{|p{\widthNPB}|C{0.45cm}|C{.45cm}|}
						\hline\rowcolor{\mycolor!20}
						\thead{\normalsize\sffamily\bfseries\fontfamily{qag}\selectfont Phát biểu}& {\bfseries\fontfamily{qag}\selectfont Đ} & {\bfseries\fontfamily{qag}\selectfont S}\\
						\hline
						{a)} \KiemtraA#2\dotEX &\ifbool{Ads}{{\color{\maudapanT}\footnotesize\faCheck}}{} &\ifbool{Ads}{}{{\color{\maudapanF}\footnotesize\faClose}}  \\
						\hline
						{b)} \KiemtraB#3\dotEX &\ifbool{Bds}{{\color{\maudapanT}\footnotesize\faCheck}}{} &\ifbool{Bds}{}{{\color{\maudapanF}\footnotesize\faClose}}  \\
						\hline
					\end{tabular}~
					\begin{tabular}[t]{|p{\widthNPB}|C{0.35cm}|C{.35cm}|}
						\hline \rowcolor{\mycolor!20}
						\thead{\normalsize\sffamily\bfseries\fontfamily{qag}\selectfont Phát biểu}& {\bfseries\fontfamily{qag}\selectfont Đ} & {\bfseries\fontfamily{qag}\selectfont S}\\
						\hline
						{c)} \KiemtraC#4\dotEX &\ifbool{Cds}{{\color{\maudapanT}\footnotesize\faCheck}}{} &\ifbool{Cds}{}{{\color{\maudapanF}\footnotesize\faClose}}  \\
						\hline
						{d)} \KiemtraD#5\dotEX &\ifbool{Dds}{{\color{\maudapanT}\footnotesize\faCheck}}{} &\ifbool{Dds}{}{{\color{\maudapanF}\footnotesize\faClose}}  \\
						\hline
					\end{tabular}
				\end{tabular}
			\end{center}
		}{
			\begin{multicols}{#1}
				\begin{enumerate}[wide=0.65cm,label*=\protect{\alph*)},itemsep=-3pt,topsep=-4pt]
					\item \KiemtraA#2\dotEX	
					\item \KiemtraB#3\dotEX
					\item \KiemtraC#4\dotEX
					\item \KiemtraD#5\dotEX
				\end{enumerate}
			\end{multicols}	
		}
	}
}
\Writetofile{ansbook}{\string\def\string\writeANS{\saveans}}
\scantokens{
	\begin{solbook}
		\writeANS
	\end{solbook}
}%
\setcounter{numTrue}{0}%
}

\newcommand{\TLdung}[1]{%
\tikz[baseline=(char.base)]
{\node[shape=circle,inner sep=0.5pt,draw=\mauphu,
	font=\bfseries\footnotesize,text=\mauphu,minimum size=12pt,outer sep=0pt] (char) {#1};
	\path (char) node[text=\mauphu!65!green,anchor=center]{{\footnotesize\faCheck}};
}\ignorespaces}

\newcommand{\TLsai}[1]{%
\tikz[baseline=(char.base)]
{\node[shape=circle,inner sep=0.5pt,draw=\maunhan,font=\bfseries\footnotesize,text=\maunhan!60!black,minimum size=12pt,outer sep=0pt] (char) {#1};
	\path (char) node[text=\maunhan!30!red,anchor=center]{\footnotesize\faTimes};
}\ignorespaces}
\def\checkmarkH{{\tikz[line join=miter]{\draw (0.5ex,0.5ex) coordinate (O) circle (0.75ex);\path[preaction={draw=white,line width=0.2ex},fill=white,draw] (0,.5ex) -- (.5ex,0) -- (1.5ex,1.35ex) -- (.45ex,.25ex) -- cycle; \draw ([shift={(90:0.75ex)}]O) arc (90:270:0.75ex);}} }

\def\itemch{\item}
\newenvironment{itemchoice}{\begin{enumerate}[wide=0.65cm,leftmargin=0.65cm,label*=\protect{\alph*)},itemsep=-3pt,topsep=-4pt]}{\end{enumerate}}

\newcommand{\loigiaiTF}{
\renewcommand{\loigiai}[1]{%
	\par\noindent%
	{\xdef\tieudehinh{}\par\noindent%
		{\color{\mycolor}\reflectbox{\Large\WritingHand}\ {Lời giải.}} ##1\hfill \textcolor{\maunhan}{\faKey}~\hspace*{-7pt}
		\ifbool{Ads}{\TLdung{A}}{\TLsai{A}}~\hspace*{-7pt}
		\ifbool{Bds}{\TLdung{B}}{\TLsai{B}}~\hspace*{-7pt}
		\ifbool{Cds}{\TLdung{C}}{\TLsai{C}}~\hspace*{-7pt}
		\ifbool{Dds}{\TLdung{D}}{\TLsai{D}}}
}%
}


\RenewEnviron{Solbook}[1]{%
\tikz{
	\node[rounded corners,draw=\maunhan,font=\bfseries,text width=4cm,
	align=center,minimum height=20pt]{\resizebox{3.9cm}{!}{Câu #1. \BODY}};
}%
}
\RenewEnviron{Solution}[1]{%
\tikz{\node[draw=\mycolor,thick,inner sep=3pt,text width=0.1\linewidth-4\fboxsep,minimum height=0.65cm,
	rounded corners,fill=yellow!5,align=center](A){
		{\textbf{#1}} -	{\textbf{\BODY}}
	};
	\path ($(A)+(0,-13pt)$);
}
}
\RenewEnviron{enumEX}[2][]{
\ifthenelse{\equal{#2}{1}}%
{%
	\begin{enumerate}%[label*=\circlenum{\Alph*},itemsep=-1pt,topsep=0pt]%
		\BODY%
	\end{enumerate}%
}%
{\begin{multicols}{#2}
		\begin{enumerate}%[label*=\circlenum{\Alph*},itemsep=-1pt,topsep=0pt]%
			\BODY%
		\end{enumerate}%
\end{multicols}}%
}
\RenewEnviron{listEX}[1][1]{
\ifthenelse{\equal{#1}{1}}%
{%
	\begin{enumerate}%[label*=\circlenum{\Alph*},itemsep=-1pt,topsep=0pt]%
		\BODY%
	\end{enumerate}%
}%
{\begin{multicols}{#1}
		\begin{enumerate}%[label*=\circlenum{\Alph*},itemsep=-1pt,topsep=0pt]%
			\BODY%
		\end{enumerate}%
\end{multicols}}%
}
%%%================Lệnh bổ sung =====================%%%





