\usepackage{pgfplots}
\usepackage{filecontents}
\pgfplotsset{compat=1.18}
\usepackage{chemfig}
\usepackage{chemmacros}
%\DeclareMathAlphabet{\mathsf}{T1}{qag}{b}{n}
\DeclareMathAlphabet{\mathsf}{T1}{qhv}{b}{n}
\DeclareMathAlphabet{\mathbf}{T1}{put}{b}{n}
\DeclareMathAlphabet{\mathtt}{T1}{pcr}{m}{n}   
\DeclareMathAlphabet{\mathit}{T1}{ptm}{m}{it} 
\usetikzlibrary{decorations.markings, positioning}
%%%============Lệnh chữ không chân in đậm ===================%%%
\NewDocumentCommand{\MathsfChemfig}{O{\mycolor}O{2.3}m}{
	\renewcommand*\printatom[1]{\ensuremath{\mathsf{##1}}}
	\chemfig[atom style={#1},atom sep=#2em,bond style={#1,line width =0.8pt}]{#3}
}

\NewDocumentCommand{\MathbfChemfig}{O{\mycolor}O{2.3}m}{
	\renewcommand*\printatom[1]{\ensuremath{\mathbf{##1}}}
	\chemfig[atom style={#1},atom sep=#2em,bond style={#1,line width =0.8pt}]{#3}
}

	
%\everymath{\rm}
%%%================Tạo lệnh mới trong chemfig==================%%%
%\renewcommand*\printatom[1]{\ensuremath{\mathbf{#1}}}%%lệnh viết font không chân/in dậm\mathbf
%%% =======Lệnh tô màu nguyên tử ==========%%%
\def\colorchem{\mycolor!80!black}
\newcommand*\maucthh[1]{\color{\colorchem}{\printatom{#1}}}
%%% =======Lệnh khoanh tròn nguyên tử ==========%%%
\newcommand*\circleatom[1]{\tikz\node[circle,draw,color=\mycolor]{\printatom{#1}};}
%%%=============Tạo style nhiều tùy chọn===================%%%
\tikzset{cfhbond/.style={% 
		/utils/exec=\tikzset{cfh/.cd,#1},
		draw=none,
		postaction={decorate},
		decoration={
			markings,
			mark = at position \pgfkeysvalueof{/tikz/cfh/position} with {\fill[\pgfkeysvalueof{/tikz/cfh/color}] circle[radius=\pgfkeysvalueof{/tikz/cfh/radius}*1pt];},
			mark = at position 0.5 with {\fill[\pgfkeysvalueof{/tikz/cfh/color}] circle[radius=\pgfkeysvalueof{/tikz/cfh/radius}*1pt];},
			mark = at position 1-\pgfkeysvalueof{/tikz/cfh/position} with {\fill[\pgfkeysvalueof{/tikz/cfh/color}] circle[radius=\pgfkeysvalueof{/tikz/cfh/radius}*1pt];}
		}%
	},cfh/.cd,color/.initial=red,radius/.initial=1,position/.initial=0.22} 
%%%===========================Vòng có nhánh 5 tùy chọn bỏ trống=======================%%%
\NewDocumentCommand{\cyclohexan}{O{}O{}O{}O{}O{}m}{%
	\ifblank{#1}{\definesubmol\ortho{}}{\definesubmol\ortho{(-[,.7]#1)}}
	\ifblank{#2}{\definesubmol\meta{}}{\definesubmol\meta{(-[,.7]#2)}}
	\ifblank{#3}{\definesubmol\para{}}{\definesubmol\para{(-[,.7]#3)}}
	\ifblank{#4}{\definesubmol\metax{}}{\definesubmol\metax{(-[,.7]#4)}}
	\ifblank{#5}{\definesubmol\orthox{}}{\definesubmol\orthox{(-[,.7,]#5)}}
	{\chemfig{**6(!\metax-!\para-!\meta-!\ortho-(-[,.7]#6)-!\orthox-)}}
}

%%%============Tùy chỉnh toàn cục=======================%%%

%\setchemfig{%
%	bond join=true,
%	arrow label sep=3pt,
%	arrow offset=1.5pt,
%	arrow coeff=2.5pt,
%	atom sep=3em,
%	arrow style={\mycolor!40!black,>=stealth,thick},
%	atom style={color = \mycolor!40!black},
%	bond style={\mycolor!40!black,thick},
%	bond offset=1pt
%}

%%%==================Lệnh tạo mũi tên điền vào các orbitan===================
%%%Tùy chọn 1:màu sắc [bắt buộc, số lượng mũi tên và hướng <0,1u,1d,2uu,2dd,2ud,2du>, ngăn cách nhau bởi dấu phảy]
%%%Tùy chọn 2:Kích thước 
%%%Tùy chọn 3:Màu săc
\NewDocumentCommand{\squarerow}{O{}O{0.65}O{\mycolor}O{0pt}m}{
	\raisebox{#4}{\begin{tikzpicture}
		\tikzstyle{styline} = [color=#3,-stealth,line width=0.8pt,line cap=round,line join=round]
		\foreach \i in {1,...,#5} {
			\draw[styline] ({(\i-1)*#2},0) rectangle ({\i*#2},#2);
		}
		\if\relax\detokenize{#1}\relax
		\else
		\foreach \arrows [count=\i] in {#1} {
			\if\arrows0
			% Không có mũi tên
			\else
			\ifnum\pdfstrcmp{\arrows}{1u}=0
			\draw[styline] ({(\i-0.5)*#2},0.1*#2) -- ({(\i-0.5)*#2},0.9*#2);
			\else\ifnum\pdfstrcmp{\arrows}{1d}=0
			\draw[styline] ({(\i-0.5)*#2},0.9*#2) -- ({(\i-0.5)*#2},0.1*#2);
			\else\ifnum\pdfstrcmp{\arrows}{2uu}=0
			\draw[styline] ({(\i-0.62)*#2},0.1*#2) -- ({(\i-0.62)*#2},0.9*#2);
			\draw[styline] ({(\i-0.38)*#2},0.1*#2) -- ({(\i-0.38)*#2},0.9*#2);
			\else\ifnum\pdfstrcmp{\arrows}{2dd}=0
			\draw[styline] ({(\i-0.62)*#2},0.9*#2) -- ({(\i-0.62)*#2},0.1*#2);
			\draw[styline] ({(\i-0.38)*#2},0.9*#2) -- ({(\i-0.38)*#2},0.1*#2);
			\else\ifnum\pdfstrcmp{\arrows}{2ud}=0
			\draw[styline] ({(\i-0.62)*#2},0.1*#2) -- ({(\i-0.62)*#2},0.9*#2);
			\draw[styline] ({(\i-0.38)*#2},0.9*#2) -- ({(\i-0.38)*#2},0.1*#2);
			\else\ifnum\pdfstrcmp{\arrows}{2du}=0
			\draw[styline] ({(\i-0.62)*#2},0.9*#2) -- ({(\i-0.62)*#2},0.1*#2);
			\draw[styline] ({(\i-0.38)*#2},0.1*#2) -- ({(\i-0.38)*#2},0.9*#2);
			\fi\fi\fi\fi\fi\fi
			\fi
		}
		\fi
	\end{tikzpicture}}
}

%%% Liên kết Pi giải tỏa đều %%%
\catcode`\_=11 % manual p. 28
\tikzset{
	ddbond/.style args={#1}{
		draw=none,
		decoration={%
			markings,
			mark=at position 0 with {
				\coordinate (CF@startdeloc) at (0,\dimexpr#1\CF_doublesep/2)
				coordinate (CF@startaxis) at (0,\dimexpr-#1\CF_doublesep/2);
			},
			mark=at position 1 with {
				\coordinate (CF@enddeloc) at (0,\dimexpr#1\CF_doublesep/2)
				coordinate (CF@endaxis) at (0,\dimexpr-#1\CF_doublesep/2);
				\draw[dash pattern=on 2pt off 1.5pt] (CF@startdeloc)--(CF@enddeloc);
				\draw (CF@startaxis)--(CF@endaxis);
			}
		},
		postaction={decorate}
	}
}
\catcode`\_=8

%%%========================Lệnh tạo tùy chọn liên kết hidro================%%%
\catcode`\_=11
\tikzset{
	hidrobond/.style args={#1/#2/#3}{
		draw=none,
		decoration={%
			markings,
			mark=at position 0 with {
				\coordinate (CF@startdeloc) at (0,\dimexpr#1\CF_doublesep/50);
				\coordinate (CF@startaxis) at (0,\dimexpr#1\CF_doublesep/50);
			},
			mark=at position 1 with {
				\coordinate (CF@enddeloc) at (0,\dimexpr#1\CF_doublesep/50);
				\coordinate (CF@endaxis) at (0,\dimexpr#1\CF_doublesep/50);
				\path (CF@startdeloc)--(CF@enddeloc) node[pos =.5]{
					\begin{tikzpicture}[declare function={d=#2;k=.30*d;}]
						\foreach \x in {1*k,2*k,3*k}{\fill[#3] (\x,0) circle (0.037*d);}
					\end{tikzpicture}
				};
			}
		},
		postaction={decorate}
	}
}
\catcode`\_=8

%%%================Tạo lệnh chữ in đậm trong chemfig=================%%%
\renewcommand*\printatom[1]{\ensuremath{\mathbf{#1}}}
%%==========Phân tử nước đầu chuỗi============%%
\definesubmol\ptnuocdau{%
	\charge{135:2pt[red]=$\delta^{+} $}{H}-[:60]
	\charge{90:2pt[red,anchor=-90]=$\delta^{-} $}{O}-\charge{-90:2pt[red,anchor=90]=$\delta^{+} $}{H}}%
%%==========Phân tử nước giữa============%%
\definesubmol\ptnuoca{%
	\charge{-90:2pt[red,anchor=90]=$\delta^{-} $}{O}(-[:120]\charge{60:2pt[red,anchor=-120]=$\delta^{+} $}{H})-\charge{60:2pt[red,anchor=-90]=$\delta^{+} $}{H}}%
%%==========Phân tử nước cuối============%%
\definesubmol\ptnuocb{%
	\charge{90:2pt[red,anchor=-90]=$\delta^{-} $}{O}(-[:-120]\charge{135:2pt[red,anchor=-120]=$\delta^{+} $}{H})-\charge{60:2pt[red,anchor=-90]=$\delta^{+} $}{H}}%
%%=============== Lệnh tạo liên kết H ==================%%
\definesubmol\hbond{%
	-[,1.25,,,hidrobond={+}/{2.5em}/{red}]
}%

\newcommand{\lkhlpt}[1]{%
	\ifcase#1\relax%
	\or
	\chemfig[chemfig style={line width =1pt}]{%
		!\ptnuocdau!\hbond!\ptnuoca}
	\or
	\chemfig[chemfig style={line width =1pt}]{%
		!\ptnuocdau!\hbond!\ptnuoca!\hbond!\ptnuocb
	}%
	\or
	\chemfig[chemfig style={line width =1pt}]{%
		!\ptnuocdau!\hbond!\ptnuoca!\hbond!\ptnuocb!\hbond!\ptnuoca
	}%
	\or
	\chemfig[chemfig style={line width =1pt}]{%
		!\ptnuocdau!\hbond!\ptnuoca!\hbond!\ptnuocb!\hbond!\ptnuoca!\hbond!\ptnuocb
	}%
	\else
	\chemfig[chemfig style={line width =1pt}]{%
		!\ptnuocdau!\hbond!\ptnuoca!\hbond!\ptnuocb!\hbond!\ptnuoca!\hbond!\ptnuocb!\hbond!\ptnuoca
	}%
	\fi
}

%%%===============Lệnh công thỨc hóa học kèm theo tên phía dưới=====================%%%
\NewDocumentCommand{\ChemName}{O{3}O{0}mm}{%
	\tikz{%
		\node[inner sep =0pt,outer sep=0pt] (char){%
		\chemfig{#3}%
		};%
		\node [below=#1pt of char,xshift = #2 cm,font=\scriptsize\color{\maunhan}\bfseries\sffamily ] {#4};
		}%
}%





























%%% Tùy chỉnh mũi tên %%%
%%%% Tùy chỉnh mũi tên không phản ứng %%%
\catcode`\_=11
\definearrow4{-x>}{%
	\edef\leng{\ifx\empty#4\empty 8pt\else #4\fi}% dot radius
	\CF_arrowshiftnodes{#3}%
	\expandafter\draw\expandafter[\CF_arrowcurrentstyle](\CF_arrowstartnode)--(\CF_arrowendnode)
	coordinate[midway](mid@point);
	\path (\CF_arrowstartnode)--(\CF_arrowendnode) node[midway,sloped,pos =.5]{\tikz{
			\expandafter\draw\expandafter[\CF_arrowcurrentstyle,-](0,0)--++(35:\leng)--([turn]180:{2*\leng});
			\expandafter\draw\expandafter[\CF_arrowcurrentstyle,-](0,0)--++(-35:\leng)--([turn]180:{2*\leng});
	}};
	\CF_arrowdisplaylabel{#1}{0.5}{+}{\CF_arrowstartnode}{#2}{0.5}{-}{\CF_arrowendnode}
}
\catcode`\_=8


\makeatletter
\definearrow{9}{<X>}{%
	\CF@arrow@shift@nodes{#7}%
	%\expandafter\draw\expandafter[\CF@arrow@current@style,-CF](\CF@arrow@start@node)--(\CF@arrow@end@node)node[midway](Xarrow@arctangent){};%
	\path[allow upside down](\CF@arrow@start@node)--(\CF@arrow@end@node)%
	node[pos=0,sloped,yshift=1pt](\CF@arrow@start@node @u0){}%
	node[pos=0,sloped,yshift=-1pt](\CF@arrow@start@node @d0){}%
	node[pos=1,sloped,yshift=1pt](\CF@arrow@start@node @u1){}%
	node[pos=1,sloped,yshift=-1pt](\CF@arrow@start@node @d1){}%
	node[midway,yshift=1pt](Xarrow@arctangent@u){}%
	node[midway,yshift=-1pt](Xarrow@arctangent@d){};%
	\begingroup
	\pgfarrowharpoontrue
	\expandafter\draw\expandafter[\CF@arrow@current@style](\CF@arrow@start@node @u0)--(\CF@arrow@start@node @u1);%
	\expandafter\draw\expandafter[\CF@arrow@current@style](\CF@arrow@start@node @d1)--(\CF@arrow@start@node @d0);%
	\endgroup
	\edef\CF@tmp@str{\ifx\@empty#1\@empty[draw=none]\fi}%
	\expandafter\draw\CF@tmp@str (Xarrow@arctangent@u)%
	arc[radius=\CF@compound@sep*\CF@current@arrow@length*\ifx\@empty#8\@empty0.333\else#8\fi,start angle=\CF@arrow@current@angle-90,%
	delta angle=-\ifx\@empty#9\@empty60\else#9\fi]node(Xarrow1@start){};
	\edef\CF@tmp@str{[\ifx\@empty#2\@empty draw=none,\fi-CF]}%
	\expandafter\draw\CF@tmp@str (Xarrow@arctangent@u)%
	arc[radius=\CF@compound@sep*\CF@current@arrow@length*\ifx\@empty#8\@empty0.333\else#8\fi,start angle=\CF@arrow@current@angle-90,%
	delta angle=\ifx\@empty#9\@empty60\else#9\fi]node(Xarrow1@end){};
	\edef\CF@tmp@str{[\ifx\@empty#4\@empty draw=none,\fi-CF]}%
	\expandafter\draw\CF@tmp@str (Xarrow@arctangent@d)%
	arc[radius=\CF@compound@sep*\CF@current@arrow@length*\ifx\@empty#8\@empty0.333\else#8\fi,start angle=\CF@arrow@current@angle+90,%
	delta angle=\ifx\@empty#9\@empty60\else#9\fi]node(Xarrow2@start){};
	\edef\CF@tmp@str{[\ifx\@empty#5\@empty draw=none,\fi-CF]}%
	\expandafter\draw\CF@tmp@str (Xarrow@arctangent@d)%
	arc[radius=\CF@compound@sep*\CF@current@arrow@length*\ifx\@empty#8\@empty0.333\else#8\fi,start angle=\CF@arrow@current@angle+90,%
	delta angle=-\ifx\@empty#9\@empty60\else#9\fi]node(Xarrow2@end){};
	\edef\CF@tmp@str{\if\string-\expandafter\@car\detokenize{#7.}\@nil-\else+\fi}%
	\CF@arrow@display@label{#1}{0}\CF@tmp@str{Xarrow1@start}{#2}{1}\CF@tmp@str{Xarrow1@end}%
	\CF@arrow@display@label{#3}{0.5}\CF@tmp@str\CF@arrow@start@node{}{}{}\CF@arrow@end@node%
	\edef\CF@tmp@str{\if\string-\expandafter\@car\detokenize{#7.}\@nil+\else-\fi}%
	\CF@arrow@display@label{#4}{0}\CF@tmp@str{Xarrow2@start}{#5}{1}\CF@tmp@str{Xarrow2@end}%
	\CF@arrow@display@label{#6}{0.5}\CF@tmp@str\CF@arrow@start@node{}{}{}\CF@arrow@end@node%
}
\makeatother


%%%Tùy chỉnh mũi tên\rightxarrow[]{}%%%%%%%%%%
\NewDocumentCommand{\rightxarrow}{O{}O{}}{%
	\ifblank{#1}{\def\tuychonone{1}}{\def\tuychonone{#1}}
	\ifblank{#2}{\def\tuychontwo{}}{\def\tuychontwo{#2}}
	\mathrel{%
	\vcenter{\hbox{%
			\begin{tikzpicture}[join=round,cap=round]
				\node[minimum width=\tuychonone cm,minimum height=1ex,align=center,anchor=center,inner sep=0pt,outer sep =0pt] (a){\text{\vphantom{hg}}\\[0.0 ex]\text{\vphantom{hg}} };
				\draw[->,\tuychontwo] ([yshift=0.2ex]a.west) -- ([yshift=0.2ex]a.east);
				\path(a) node[pos=0.5,midway,yshift=0.2 ex]{\tikz[join=round,cap=round)]
					\draw[\tuychontwo] (-2pt,-2pt) --(2pt,2pt)(2pt,-2pt) --(-2pt,2pt);
				};
			\end{tikzpicture}%thêm % xóa khoảng trắng phia sua lệnh
	}}%
	}
	\mkern-2mu
}%



%%%Tùy chỉnh mũi tên\xrightarrow[]{}%%%%%%%%%%
\RenewDocumentCommand{\xrightarrow}{O{}O{}O{}O{}O{black,>=stealth}O{line width=0.65pt}}{%
	\ifblank{#1}{\def\tuychonone{}}{\def\tuychonone{#1}}
	\ifblank{#2}{\def\tuychontwo{}}{\def\tuychontwo{#2}}
	\ifblank{#3}{\def\tuychonthree{1}}{\def\tuychonthree{#3}}
	\ifblank{#4}{\def\tuychonfour{-0.5}}{\def\tuychonfour{#4}}
	\mathrel{%
		\vcenter{\hbox{%
				\begin{tikzpicture}[join=round,cap=round,#6]
					\node[minimum width=\tuychonthree cm,minimum height=1ex,align=center,anchor=center,inner sep=0pt,outer sep =0pt] (a){\text{\vphantom{hg}\small\tuychonone}\\[\tuychonfour ex] \text{\vphantom{hg}\small\tuychontwo}};
					\draw[->,#5] ([yshift=0.2ex]a.west) -- ([yshift=0.2ex]a.east);
				\end{tikzpicture}%thêm % xóa khoảng trắng phia sua lệnh
		}}%
	}%
	\mkern-2mu
}
%%%Tùy chỉnh mũi tên\xleftarrow[]%%%%%%%%%%
\RenewDocumentCommand{\xleftarrow}{O{}O{}O{}O{}O{black,>=stealth}O{line width=0.65pt}}{%
	\ifblank{#1}{\def\tuychonone{}}{\def\tuychonone{#1}}
	\ifblank{#2}{\def\tuychontwo{}}{\def\tuychontwo{#2}}
	\ifblank{#3}{\def\tuychonthree{1}}{\def\tuychonthree{#3}}
	\ifblank{#4}{\def\tuychonfour{0.2}}{\def\tuychonfour{#4}}
	\mathrel{%
		\vcenter{\hbox{%
				\begin{tikzpicture}[join=round,cap=round,#6]
					\node[minimum width=\tuychonthree cm,minimum height=1ex,align=center,anchor=center,inner sep=0pt,outer sep =0pt] (a){\text{\vphantom{hg}\small\tuychonone}\\[\tuychonfour ex] \text{\vphantom{hg}\small\tuychontwo}};
					\draw[<-,#5] ([yshift=0.2ex]a.west) -- ([yshift=0.2ex]a.east);
				\end{tikzpicture}%
		}}%
	}%
	\mkern-2mu
}

%%%Tùy chỉnh mũi tên\uparrow[]{}%%%%%%%%%%

\RenewDocumentCommand{\uparrow}{O{}O{}O{}O{}}{%
		\ifblank{#1}{\def\tuychonone{1.5}}{\def\tuychonone{#1}}
		\ifblank{#2}{\def\tuychontwo{-stealth}}{\def\tuychontwo{#2}}
		\ifblank{#3}{\def\tuychonthree{0pt}}{\def\tuychonthree{#3}}
		\ifblank{#4}{\def\tuychonfour{-2mu}}{\def\tuychonfour{#4}}
	\mathrel{%
		\vcenter{\hbox{%
				\begin{tikzpicture}[join=round,cap=round,xshift=\tuychonthree,baseline=0pt]
					\node[minimum width=0.1pt,minimum height=\tuychonone ex,inner sep=0pt,outer sep=0pt] (a) {};
					\draw[\tuychontwo] ([xshift=-0.4ex]a.south) -- ([xshift=-0.4ex]a.north);
				\end{tikzpicture}%
		}}%
	}%
	\mkern\tuychonfour%
}

%%%Tùy chỉnh mũi tên\downarrow[]{}%%%%%%%%%%
\RenewDocumentCommand{\downarrow}{O{}O{}O{}O{}}{%
	\ifblank{#1}{\def\tuychonone{1.5}}{\def\tuychonone{#1}}
	\ifblank{#2}{\def\tuychontwo{-stealth}}{\def\tuychontwo{#2}}
	\ifblank{#3}{\def\tuychonthree{0pt}}{\def\tuychonthree{#3}}
	\ifblank{#4}{\def\tuychonfour{-2mu}}{\def\tuychonfour{#4}}
	\mathrel{%
		\vcenter{\hbox{%
				\begin{tikzpicture}[join=round,cap=round,xshift=\tuychonthree,baseline=0pt]
					\node[minimum width=0.1pt,minimum height=\tuychonone ex,inner sep=0pt,outer sep=0pt] (a) {};
					\draw[\tuychontwo] ([xshift=-0.5ex]a.north) -- ([xshift=-0.5ex]a.south);
				\end{tikzpicture}%
		}}%
	}%
	\mkern\tuychonfour%
}

%%%=================Tạo mũi tên hai chiều=================%%%
\RenewDocumentCommand{\xleftrightarrow}{O{}O{}O{}O{}O{black,>=stealth}O{line width=0.65pt,>=stealth}}{%
	\ifblank{#1}{\def\tuychonone{}}{\def\tuychonone{#1}}
	\ifblank{#2}{\def\tuychontwo{}}{\def\tuychontwo{#2}}
	\ifblank{#3}{\def\tuychonthree{0.7}}{\def\tuychonthree{#3}}
	\ifblank{#4}{\def\tuychonfour{-0.9}}{\def\tuychonfour{#4}}
	\mathrel{%
		\vcenter{\hbox{%
				\begin{tikzpicture}[join=round,cap=round,#6]
					\node[minimum width=\tuychonthree cm,minimum height=1ex,anchor=south,align=center,inner sep=0pt,outer sep =0pt] (a){\text{\vphantom{hg}\scriptsize\tuychonone}\\[\tuychonfour ex] \text{\vphantom{hg}\scriptsize\tuychontwo}};
					\draw[->,#5] ([yshift=0.45ex]a.west) -- ([yshift=0.45ex]a.east);
					\draw[<-,#5] ([yshift=-0.10ex]a.west) -- ([yshift=-0.10ex]a.east);
				\end{tikzpicture}
		}}%
	}%
	\mkern-4mu
}


%%%=================Tạo mũi tên hai chiều=================%%%
\NewDocumentCommand{\xharpoonarrow}{O{}O{}O{}O{}O{black,>=stealth}O{line width=0.65pt}}{%
	\ifblank{#1}{\def\tuychonone{}}{\def\tuychonone{#1}}
	\ifblank{#2}{\def\tuychontwo{}}{\def\tuychontwo{#2}}
	\ifblank{#3}{\def\tuychonthree{1.65}}{\def\tuychonthree{#3}}
	\ifblank{#4}{\def\tuychonfour{-0.9}}{\def\tuychonfour{#4}}
	\mathrel{%
		\vcenter{\hbox{%
				\begin{tikzpicture}[join=round,cap=round,#6]
					\node[minimum width=\tuychonthree cm,minimum height=1ex,anchor=south,align=center,inner sep=0pt,outer sep =0pt] (a){\text{\vphantom{hg}\scriptsize\tuychonone}\\[\tuychonfour ex] \text{\vphantom{hg}\scriptsize\tuychontwo}};
					\draw[arrows = {-Stealth[harpoon]},#5] ([yshift=0.35ex]a.west) -- ([yshift=0.35ex]a.east);
					\draw[arrows = {-Stealth[harpoon]},#5] ([yshift=-0.10ex]a.east) -- ([yshift=-0.10ex]a.west);
				\end{tikzpicture}%thêm % xóa khoảng trắng phia sua lệnh
		}}%
	}%
	\mkern -2mu
}









