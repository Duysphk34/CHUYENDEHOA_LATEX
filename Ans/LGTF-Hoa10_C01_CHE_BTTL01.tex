\def\writeANS{\TLdung{A}\TLsai{B}\TLdung{C}\TLdung{D}}
\begin{loigiaiex}{20}
  Hạn chế của mô hình nguyên tử Bohr: \begin {itemchoice} \itemch \textbf {Đúng}. Mô hình Bohr chỉ giải thích tốt phổ của nguyên tử hydro và các ion giống hydro. \itemch \textbf {Sai}. Sự tồn tại của đồng vị không liên quan trực tiếp đến mô hình Bohr. \itemch \textbf {Đúng}. Mô hình này không đề cập đến cơ chế hình thành liên kết hóa học. \itemch \textbf {Đúng}. Mô hình Bohr xác định chính xác vị trí và động lượng của electron, trái với nguyên lý bất định Heisenberg. \end {itemchoice}  \phantom {a}\hfill { \faKey ~\writeANS }
\end{loigiaiex}
\def\writeANS{\TLdung{A}\TLsai{B}\TLdung{C}\TLsai{D}}
\begin{loigiaiex}{21}
  Về sự chuyển dời của electron trong mô hình Bohr: \begin {itemchoice} \itemch \textbf {Đúng}. Khi electron chuyển từ quỹ đạo cao xuống thấp, nó giải phóng năng lượng dưới dạng photon. \itemch \textbf {Sai}. Mọi sự chuyển dời của electron đều liên quan đến việc hấp thụ hoặc phát ra năng lượng. \itemch \textbf {Đúng}. Năng lượng của photon phát ra chính xác bằng hiệu năng lượng giữa hai mức electron chuyển đổi. \itemch \textbf {Sai}. Electron có thể chuyển đến bất kỳ mức năng lượng cao hơn nào khi được kích thích đủ. \end {itemchoice}  \phantom {a}\hfill { \faKey ~\writeANS }
\end{loigiaiex}
\def\writeANS{\TLdung{A}\TLsai{B}\TLdung{C}\TLdung{D}}
\begin{loigiaiex}{22}
  Về sự phân bố electron vào các lớp và phân lớp: \begin {itemchoice} \itemch \textbf {Đúng}. Nguyên lý vun đắp quy định thứ tự điền electron vào các orbital. \itemch \textbf {Sai}. Công thức $2n^2$ chỉ đúng cho 4 lớp đầu tiên. Từ lớp thứ 5 trở đi, số electron tối đa không theo quy luật này. \itemch \textbf {Đúng}. Phân lớp d xuất hiện từ lớp $n = 3$ trở đi. \itemch \textbf {Đúng}. Trong cùng một lớp, thứ tự mức năng lượng tăng dần là $s < p < d < f$. \end {itemchoice}  \phantom {a}\hfill { \faKey ~\writeANS }
\end{loigiaiex}
\def\writeANS{\TLdung{A}\TLsai{B}\TLdung{C}\TLsai{D}}
\begin{loigiaiex}{23}
  Về cấu hình electron: \begin {itemchoice} \itemch \textbf {Đúng}. Phân lớp p có 3 orbital, mỗi orbital chứa tối đa 2 electron, nên tổng cộng là 6 electron. \itemch \textbf {Sai}. Phân lớp f chứa tối đa 14 electron (7 orbital, mỗi orbital 2 electron). \itemch \textbf {Đúng}. Số lượng orbital trên các phân lớp $s, p, d, f$ lần lượt là $1, 3, 5, 7$. \itemch \textbf {Sai}. Số electron tối đa trong một orbital là 2, tuân theo nguyên lý Pauli. \end {itemchoice}  \phantom {a}\hfill { \faKey ~\writeANS }
\end{loigiaiex}
\def\writeANS{\TLdung{A}\TLdung{B}\TLsai{C}\TLsai{D}}
\begin{loigiaiex}{24}
  Về quy tắc Hund: \begin {itemchoice} \itemch \textbf {Đúng}. Đây là nội dung cơ bản của quy tắc Hund. \itemch \textbf {Đúng}. Các electron đơn trong các orbital cùng năng lượng sẽ có spin cùng chiều để đạt trạng thái bền vững nhất. \itemch \textbf {Sai}. Quy tắc Hund áp dụng cho tất cả các nguyên tố, không chỉ riêng nhóm p. \itemch \textbf {Sai}. Điều này trái với quy tắc Hund, electron sẽ điền đơn vào các orbital trước khi ghép đôi. \end {itemchoice}  \phantom {a}\hfill { \faKey ~\writeANS }
\end{loigiaiex}
\def\writeANS{\TLsai{A}\TLdung{B}\TLdung{C}\TLdung{D}}
\begin{loigiaiex}{25}
  Về sự phân bố electron trong nguyên tử: \begin {itemchoice} \itemch \textbf {Sai}. Nguyên tử có thể ở trạng thái kích thích khi được cung cấp năng lượng. \itemch \textbf {Đúng}. Cấu hình electron của ion được xác định bằng cách thêm hoặc bớt electron từ nguyên tử trung hòa. \itemch \textbf {Đúng}. Trạng thái cơ bản là trạng thái có năng lượng thấp nhất. \itemch \textbf {Đúng}. Các electron trong cùng một phân lớp có mức năng lượng bằng nhau, các electron thuộc cùng một lớp có mức năng lượng gần bằng nhau. \end {itemchoice}  \phantom {a}\hfill { \faKey ~\writeANS }
\end{loigiaiex}
\def\writeANS{\TLdung{A}\TLsai{B}\TLdung{C}\TLsai{D}}
\begin{loigiaiex}{26}
  Về bán kính quỹ đạo electron trong mô hình Bohr: \begin {itemchoice} \itemch \textbf {Đúng}. Bán kính quỹ đạo $r = n^2 \cdot a_0$ , với n là số lượng tử chính, $a_0$ là bán kính Borh (hằng số). \itemch \textbf {Sai}. Bán kính quỹ đạo phụ thuộc trực tiếp vào số lượng tử chính n. \itemch \textbf {Đúng}. Quỹ đạo có n lớn hơn sẽ có bán kính lớn hơn và năng lượng cao hơn. \itemch \textbf {Sai}. Bán kính quỹ đạo tỉ lệ nghịch với số nguyên tử Z, không phải tỉ lệ nghịch với số proton. \end {itemchoice}  \phantom {a}\hfill { \faKey ~\writeANS }
\end{loigiaiex}
\def\writeANS{\TLdung{A}\TLsai{B}\TLdung{C}\TLsai{D}}
\begin{loigiaiex}{27}
  Về nguyên tố X có cấu hình electron $1s^²2s^²2p^⁶3s^²3p^4$: \begin {itemchoice} \itemch \textbf {Đúng}.Dựa vào cấu hình electron ta thấy X có 16 electron do đó X có $Z=16$. \itemch \textbf {Sai}.X có 3 lớp electron, lớp M (n=3) hay lớp thứ 3 có 6 electron. \itemch \textbf {Đúng}.Trong cấu hình của X electron cuối cùng điền vào phân lớp p nên X là nguyên tố p. \itemch \textbf {Sai}.Theo quy tắc Hund 3 electron độc thân sẽ chiếm 3 AO của phân lớp 3p trước và electron cuối cùng phải tham gia ghép đôi nê cấu hình đúng phải là \squarerow [2ud,1u,1u][0.65][\mycolor ][-4pt]{3}. \end {itemchoice}  \phantom {a}\hfill { \faKey ~\writeANS }
\end{loigiaiex}
