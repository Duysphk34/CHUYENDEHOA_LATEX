\begin{loigiaibth}{1}
  Cấu hình electron của sắt $(Z=26)$: $1s^22s^22p^63s^23p^64s^23d^6$ Sắt là một kim loại chuyển tiếp vì: \begin {itemize} \item Nó có electron ở orbital d ($3d^6$) chưa được điền đầy. \item Cấu hình electron lớp ngoài cùng là $4s^23d^6$ , điều này cho phép sắt có nhiều trạng thái oxi hóa khác nhau. \item Sự hiện diện của electron ở orbital d cho phép sắt tạo ra các hợp chất có màu và có tính chất từ tính. \item Khả năng tạo phức chất đa dạng do sự tương tác giữa các orbital d với các phối tử. \end {itemize}
\end{loigiaibth}
\begin{loigiaibth}{2}
  Cấu hình electron của đồng $(Z=29)$: $1s^22s^22p^63s^23p^64s^13d^{10}$ \\ Cấu hình electron của đồng khác biệt vì: \begin {itemize} \item Theo quy luật điền electron thông thường, cấu hình dự đoán sẽ là $4s^23d^9$. \item Tuy nhiên, cấu hình thực tế là $4s^13d^{10}$, với một electron từ orbital 4s chuyển sang 3d. \item Điều này xảy ra vì orbital 3d đầy $(d^{10})$ tạo ra trạng thái năng lượng thấp hơn so với cấu hình $4s^23d^9$. \item Hiện tượng này được gọi là "hiệu ứng orbital d nửa đầy hoặc đầy", tạo ra sự ổn định đặc biệt cho nguyên tử. \end {itemize}
\end{loigiaibth}
\begin{loigiaibth}{3}
  \begin {itemize} \item Cấu hình electron: \begin {itemize} \item R ($Z=13$): $1s^2 2s^2 2p^6 3s^2 3p^1$ \item S ($Z=9$): $1s^2 2s^2 2p^5$ \end {itemize} \item Khi R nhường 3 electron và S nhận 1 electron: \begin {itemize} \item $\text {R}^{3+}$: $1s^2 2s^2 2p^6$ (như Ne) \item $\text {S}^-$: $1s^2 2s^2 2p^6$ (như Ne) \end {itemize} \item Đặc điểm: \begin {itemize} \item Cả hai ion đều có 8 electron ở lớp ngoài cùng. \item Cả hai đều đạt cấu hình electron bền vững của khí hiếm (Ne). \end {itemize} \end {itemize}
\end{loigiaibth}
\begin{loigiaibth}{4}
  \begin {itemize} \item Tổng số electron ở các phân lớp p là 11, ta có: $2p^6 3p^5$ \item Các phân lớp s trước đó phải được điền đầy đủ: $1s^2 2s^2 3s^2$ \item Cấu hình electron đầy đủ: $1s^2 2s^2 2p^6 3s^2 3p^5$ \item Tổng số electron: $2 + 2 + 6 + 2 + 5 = 17$ \item Vậy Z = 17, đây là nguyên tố Cl (Clo) \end {itemize}
\end{loigiaibth}
\begin{loigiaibth}{5}
  \begin {itemize} \item Phân lớp d có 5 electron, ta có: $3d^5$ \item Các phân lớp trước đó phải được điền đầy đủ: $1s^2 2s^2 2p^6 3s^2 3p^6 4s^2$ \item Cấu hình electron đầy đủ: $1s^2 2s^2 2p^6 3s^2 3p^63d^5 4s^2$ \item Tổng số electron: $2 + 2 + 6 + 2 + 6 + 2 + 5 = 25$ \item Vậy Z = 25, đây là nguyên tố Mn (Mangan) \item Mn là một kim loại chuyển tiếp \end {itemize}
\end{loigiaibth}
