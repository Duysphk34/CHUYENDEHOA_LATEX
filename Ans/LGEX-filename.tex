\def\writeANS{\faKey\protect \, \TLdung{A}\TLsai{B}\TLdung{C}\TLsai{D}}
\def\writeANS{\faKey\protect \, \protect \circlenum {D}}
\begin{loigiaiex}{57}
 Phương trình điện li đúng cho một hợp chất điện li yếu là: $CH_3COOH \rightleftharpoons \text {CH}_3\text {COO}^- + \text {H}^+$ \\ $CH_3COOH$ là acid yếu, chỉ phân li một phần trong dung dịch nước. Các chất còn lại là điện li mạnh, phân li hoàn toàn.
\end{loigiaiex}
\def\writeANS{\faKey\protect \, \TLdung{A}\TLsai{B}\TLdung{C}\TLsai{D}}
\def\writeANS{\faKey\protect \, \protect \circlenum {A}}
\begin{loigiaiex}{58}
  $CH_3COOH$ chất điện li mạnh nên dùng mũi tên một chiều, còn $HClO$,$H_2CO_3$, $H_2S$là chất điện li yếu nên dùng mũi tên hai chiều
\end{loigiaiex}
\def\writeANS{\faKey\protect \, \TLdung{A}\TLsai{B}\TLdung{C}\TLsai{D}}
\def\writeANS{\faKey\protect \, \protect \circlenum {D}}
\begin{loigiaiex}{59}
  $H_2SO_4$ là acid mạnh nên phân li hoàn toàn thành ion $H^+$ và $SO_4^{2-}$ \[\mathrm {H}_2\mathrm {SO}_4 \xrightarrow 2\mathrm {H}^+ + \mathrm {SO}_4^{-}\]
\end{loigiaiex}
