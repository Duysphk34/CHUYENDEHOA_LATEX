\begin{loigiaiex}{28}
 Vì tổng số electron trên phân lớp s là 8 nên cấu hình electron của Y là $1s^22s^22p^63s^23p^64s^2$ $\Rightarrow $ Y là nguyên tố Ca (Canxi) \phantom {a}\hfill { \faKey ~\circlenum {C}}
\end{loigiaiex}
\begin{loigiaiex}{29}
 Tổng số electron ở các phân lớp p là 15, tương ứng với cấu hình $1s^22s^22p^63s^23p^63d^{10}4s^24p^3$ $\Rightarrow $ Z là nguyên tố As (Asen) \phantom {a}\hfill { \faKey ~\circlenum {A}}
\end{loigiaiex}
\begin{loigiaiex}{30}
 Số electron ở phân lớp ngoài cùng là 6, tương ứng với cấu hình electron $1s^22s^22p^4$ $\Rightarrow $ M là nguyên tố O (Oxi) \phantom {a}\hfill { \faKey ~\circlenum {B}}
\end{loigiaiex}
\begin{loigiaiex}{31}
 Tổng số electron ở các phân lớp d là 5, tương ứng với cấu hình electron $1s^22s^22p^63s^23p^63d^5$ $\Rightarrow $ R là nguyên tố Mn (Mangan) \phantom {a}\hfill { \faKey ~\circlenum {C}}
\end{loigiaiex}
\begin{loigiaiex}{32}
 Tổng số electron ở các phân lớp s và p là 18, tương ứng với cấu hình electron $1s^22s^22p^63s^23p^6$ $\Rightarrow $ T là nguyên tố Ar (Argon) \phantom {a}\hfill { \faKey ~\circlenum {C}}
\end{loigiaiex}
\begin{loigiaiex}{33}
 \begin {\itemchoice } \itemch \textbf {Sai}. Lớp M (n=3) có 5 electron. \itemch \textbf {Đúng}. X có 5 electron ở lớp cuối cùng nên nguyên tử nguyên tố X có tính phi kim. \itemch \textbf {Đúng}. Theo quy tắc Hund 3 electron ở phân lớp 3p mỗi electron sẽ chiếm 1 AO trước tiên để số electron độc thân là lớn nhất (3 elctron độc thân). \itemch X có 5 electron ở lớp cuối cùng theo quy tắc bát tử sẽ nhận thêm 3 electron để đạt cấu hình bền giống khi hiếm liền sau \end {\itemchoice }  \phantom {a}\hfill { \faKey ~\circlenum {A}}
\end{loigiaiex}
\begin{loigiaiex}{34}
 Sự phân bố electron của nguyên tử $Y$ theo mức năng lượng là : $1s^22s^22p^63s^23p^63d^54s^2$. Nhận thấy electron có mức năng lượng cao nhất thuộc phân lớp 3d $\Rightarrow $ là nguyên tố d.  \phantom {a}\hfill { \faKey ~\circlenum {B}}
\end{loigiaiex}
\begin{loigiaiex}{35}
 Cấu hình electron của $Al$ là $Mg(Z=12):1s^22s^22p^63s^23p^1$ \\ $Al\xrightarrow Al^{3+} + 3e$ (nhôm nhường 3e ở lớp ngoài cùng) $\Rightarrow $ Cấu hình electron của $Al^{3+}$ là $1s^22s^22p^6$  \phantom {a}\hfill { \faKey ~\circlenum {C}}
\end{loigiaiex}
