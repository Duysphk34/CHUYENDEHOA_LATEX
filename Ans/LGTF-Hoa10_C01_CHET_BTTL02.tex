\def\writeANS{\TLdung{A}\TLdung{B}\TLsai{C}\TLsai{D}}
\begin{loigiaiex}{36}
  Về cấu hình electron của các nguyên tố: \begin {itemchoice} \itemch \textbf {Đúng}. Electron được sắp xếp theo nguyên lý vững bền (nguyên lý Aufbau), điền vào các orbital có mức năng lượng thấp nhất trước. \itemch \textbf {Đúng}. Công thức 2n^2 xác định số electron tối đa trong một lớp, với n là số lượng tử chính. \itemch \textbf {Sai}. Theo quy tắc Hund, các orbital cùng mức năng lượng sẽ được điền một electron trước khi ghép đôi. \itemch \textbf {Sai}. Mặc dù đúng cho nhiều trường hợp, nhưng có ngoại lệ, ví dụ như với các nguyên tố chuyển tiếp. \end {itemchoice}  \phantom {a}\hfill { \faKey ~\writeANS }
\end{loigiaiex}
\def\writeANS{\TLdung{A}\TLdung{B}\TLsai{C}\TLdung{D}}
\begin{loigiaiex}{37}
  \begin {itemchoice} \itemch \textbf {Đúng}. Electron được sắp xếp theo nguyên lý vững bền (nguyên lý Aufbau). \itemch \textbf {Đúng}. Công thức $2n^2$ xác định số electron tối đa trong một lớp. \itemch \textbf {Sai}. Theo quy tắc Hund, các orbital cùng mức năng lượng sẽ được điền một electron trước khi ghép đôi. \itemch \textbf {Đúng}. Quy tắc Hund áp dụng khi điền electron vào các orbital cùng mức năng lượng. \end {itemchoice}  \phantom {a}\hfill { \faKey ~\writeANS }
\end{loigiaiex}
\def\writeANS{\TLdung{A}\TLdung{B}\TLdung{C}\TLsai{D}}
\begin{loigiaiex}{38}
  \begin {itemchoice} \itemch \textbf {Đúng}. Phân lớp s có 1 orbital, mỗi orbital chứa tối đa 2 electron. \itemch \textbf {Đúng}. Phân lớp p có 3 orbital, mỗi orbital chứa tối đa 2 electron, nên tổng cộng là 6. \itemch \textbf {Đúng}. Phân lớp d có 5 orbital, mỗi orbital chứa tối đa 2 electron, nên tổng cộng là 10. \itemch \textbf {Sai}. Phân lớp f có 7 orbital, mỗi orbital chứa tối đa 2 electron, nên tổng cộng là 14, không phải 12. \end {itemchoice}  \phantom {a}\hfill { \faKey ~\writeANS }
\end{loigiaiex}
\def\writeANS{\TLdung{A}\TLdung{B}\TLsai{C}\TLdung{D}}
\begin{loigiaiex}{39}
  \begin {itemchoice} \itemch \textbf {Đúng}. Trong nguyên tử trung hòa, số electron bằng số proton. \itemch \textbf {Đúng}. Tổng số electron trong các phân lớp bằng số hiệu nguyên tử (số proton). \itemch \textbf {Sai}. Có một số ngoại lệ, ví dụ như Cr và Cu, do sự ổn định của orbital bán đầy hoặc đầy. \itemch \textbf {Đúng}. Ion dương hình thành khi nguyên tử mất electron. \end {itemchoice}  \phantom {a}\hfill { \faKey ~\writeANS }
\end{loigiaiex}
\def\writeANS{\TLdung{A}\TLdung{B}\TLdung{C}\TLsai{D}}
\begin{loigiaiex}{40}
  \begin {itemchoice} \itemch \textbf {Đúng}. Đây là nguyên tắc cơ bản của sự sắp xếp electron. \itemch \textbf {Đúng}. Nguyên lý Pauli là một trong những nguyên tắc quan trọng trong cấu hình electron. \itemch \textbf {Đúng}. Quy tắc Hund quy định cách điền electron vào các orbital cùng mức năng lượng. \itemch \textbf {Sai}. Có những trường hợp ngoại lệ, đặc biệt là ở các nguyên tố chuyển tiếp. \end {itemchoice}  \phantom {a}\hfill { \faKey ~\writeANS }
\end{loigiaiex}
\def\writeANS{\TLdung{A}\TLdung{B}\TLsai{C}\TLdung{D}}
\begin{loigiaiex}{41}
  \begin {itemchoice} \itemch \textbf {Đúng}. Đặc điểm của nguyên tố chuyển tiếp là điền electron vào phân lớp d của lớp trước. \itemch \textbf {Đúng}. Ví dụ như Cr: $[Ar]3d^5 4s^1$ thay vì $[Ar]3d^4 4s^2$. \itemch \textbf {Sai}. Một số nguyên tố chuyển tiếp có cấu hình kết thúc ở s, ví dụ Cu: [Ar]3d^10 4s^1. \itemch \textbf {Đúng}. Do có nhiều electron ở phân lớp d, chúng có thể tạo ra nhiều trạng thái oxi hóa khác nhau. \end {itemchoice}  \phantom {a}\hfill { \faKey ~\writeANS }
\end{loigiaiex}
\def\writeANS{\TLdung{A}\TLdung{B}\TLsai{C}\TLdung{D}}
\begin{loigiaiex}{42}
  \begin {itemchoice} \itemch \textbf {Đúng}. Số lớp electron tương ứng với số thứ tự chu kỳ trong bảng tuần hoàn. \itemch \textbf {Đúng}. Đối với nguyên tố nhóm A, số electron hóa trị tương ứng với số thứ tự nhóm. \itemch \textbf {Sai}. Không phải tất cả, ví dụ Zn ($[Ar]3d^10 4s^2$) không phải là kim loại kiềm thổ. \itemch \textbf {Đúng}. Cấu hình electron kết thúc ở p^6 là đặc trưng của khí hiếm (trừ He: $1s^2$). \end {itemchoice}  \phantom {a}\hfill { \faKey ~\writeANS }
\end{loigiaiex}
\def\writeANS{\TLdung{A}\TLdung{B}\TLdung{C}\TLsai{D}}
\begin{loigiaiex}{43}
  \begin {itemchoice} \itemch \textbf {Đúng}. Ion dương hình thành khi nguyên tử mất electron. \itemch \textbf {Đúng}. Ion âm hình thành khi nguyên tử nhận thêm electron. \itemch \textbf {Đúng}. Nhiều ion tạo thành có xu hướng đạt cấu hình electron ổn định giống khí hiếm. \itemch \textbf {Sai}. Không phải tất cả các ion đều có cấu hình electron giống khí hiếm, đặc biệt là các ion của kim loại chuyển tiếp. \end {itemchoice}  \phantom {a}\hfill { \faKey ~\writeANS }
\end{loigiaiex}
\def\writeANS{\TLdung{A}\TLdung{B}\TLsai{C}\TLdung{D}}
\begin{loigiaiex}{44}
  \begin {itemchoice} \itemch \textbf {Đúng}. Cấu hình này tương ứng với Zn, thuộc nhóm IIB (nhóm 12). \itemch \textbf {Đúng}. Zn thường có số oxi hóa +2 trong hợp chất. \itemch \textbf {Sai}. Zn không phải là kim loại kiềm thổ, mà là kim loại chuyển tiếp. \itemch \textbf {Đúng}. Phân lớp 3d của Zn đã bão hòa e do đó electron hóa trị bằng số electron ở phân lớp ngoài cùng 4s. \end {itemchoice}  \phantom {a}\hfill { \faKey ~\writeANS }
\end{loigiaiex}
\def\writeANS{\TLdung{A}\TLdung{B}\TLdung{C}\TLsai{D}}
\begin{loigiaiex}{45}
  \begin {itemchoice} \itemch \textbf {Đúng}. Cấu hình này tương ứng với Pb (chì), thuộc nhóm IVA (nhóm 14). \itemch \textbf {Đúng}. Pb nằm ở chu kỳ 6 của bảng tuần hoàn. \itemch \textbf {Đúng}. Pb có thể tạo hợp chất khí với hidro, ví dụ như PbH4 (plumbane). \itemch \textbf {Sai}. Pb không có tính phi kim mạnh. Nó là một kim loại yếu (hoặc á kim), có xu hướng hình thành các hợp chất ion và cộng hóa trị. \end {itemchoice}  \phantom {a}\hfill { \faKey ~\writeANS }
\end{loigiaiex}
