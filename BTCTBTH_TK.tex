\documentclass[Main_HOA10.tex]{subfiles}
\newcommand{\Textbox}[2][\mycolor]{
\begin{center}
	\tikz{
		\node[
		rectangle, 
		fill=#1!20, 
%		draw=#1!50!black,
		font=\color{#1}\Large\bfseries\fontfamily{ugq}\selectfont,
		inner sep=6pt, 
		outer sep =3pt,
		rounded corners =6pt,
		text width =0.8\linewidth,
		align=center
		] (Text){\MakeUppercase{#2}};
		
	}
\end{center}
}
\begin{document}
\Textbox[\mycolor]{Bài tập chương 2:\\Bảng tuần hoàn các nguyên tố hóa học}
\phan{Bài tập trắc nghiệm nhiều lựa chọn}
%%%=============SOẠN EX===============%%%
\Opensolutionfile{ansex}[Ans/LGEX-C02-CTBTHTH.tex]
\Opensolutionfile{ans}[Ans/Ans-C02-CTBTHTH.tex]
\hienthiloigiaiex
%\tatloigiaiex
%\luuloigiaiex
%%%=============EX_1=============%%%
\begin{ex}%[0H2N1-1]
	Trong bảng tuần hoàn các nguyên tố được sắp xếp theo nguyên tắc nào?
	\begin{enumerate}[(1)]
		\item Theo chiều tăng của điện tích hạt nhân.
		\item Các nguyên tố có cùng số lớp e trong nguyên tử được xếp thành một hàng.
		\item Các nguyên tố có cùng số e hóa trị trong nguyên tử được xếp thành một cột.
	\end{enumerate}
	\choice
	{Nguyên tắc (1)}
	{Nguyên tắc (1),(2)}
	{Nguyên tắc (2), (3)}
	{\True Nguyên tắc (1), (2), (3)}
	\loigiai{
		Có ba nguyên tắc sắp xếp các  nguyên tố trong bảng tuần hoàn:
		\begin{enumerate}[(1)]
			\item Theo chiều tăng của điện tích hạt nhân.
			\item Các nguyên tố có cùng số lớp e trong nguyên tử được xếp thành một hàng.
			\item Các nguyên tố có cùng số e hóa trị trong nguyên tử được xếp thành một cột.
		\end{enumerate}
	}
\end{ex}
%%%=============EX_2=============%%%
\begin{ex}%[0H2N1-1]
	Trong bảng tuần hoàn, số chu kì nhỏ và chu kì lớn là:
	\choice
	{\True 3 và 4}
	{4 và 3}
	{2 và 5}
	{5 và 4}
	\loigiai{
		Trong bảng tuần hoàn có 3 chu kì nhỏ là 1,2,3 và 4 chu kì lớn là 4,5,6,7.
	}
\end{ex}
%%%=============EX_3=============%%%
\begin{ex}%[0H2H1-1]
	Số nguyên tố trong chu kì 5 là:
	\choice
	{$8$}
	{$10$}
	{\True $18$}
	{$32$}
	\loigiai{%
		Số nguyên tố trong chu kì 5 là $2\times 3^2 =18 $ (nguyên tố)
	}
\end{ex}
%%%=============EX_4=============%%%
\begin{ex}%[0H2H1-1]
	Các nguyên tố hóa học trong cùng một chu kì có đặc điểm nào chung về cấu hình electron nguyên tử?
	\choice
	{Số electron hóa trị}
	{\True Số lớp electron}
	{Số electron lớp L}
	{Số phân lớp electron}
	\loigiai{%
		Các nguyên tố trong cùng chu kì có cùng số lớp electron
	}
\end{ex}
%%%=============EX_5=============%%%
\begin{ex}%[0H2H1-1]
	Bảng tuần hoàn hiện nay không áp dụng nguyên tắc sắp xếp nào sau đây?
	\choice
	{Mỗi nguyên tố hóa học được xếp vào một ô trong bảng tuần hoàn}
	{Các nguyên tố được sắp xếp theo chiều tăng dần khối lượng nguyên tử}
	{Các nguyên tố có cùng số lớp electron trong nguyên tử được xếp thành một hàng}
	{Các nguyên tố có cùng số electron hóa trị trong nguyên tử được xếp thành một cột}
	\loigiai{%
		Bảng tuần hoàn hiẹn nay không áp dụng nguyên tắc sắp xếp theo chiều tăng dần khối lượng nguyên tử.
	}
\end{ex}
%%%=============EX_6=============%%%
\begin{ex}%[0H2H1-1]
	Bảng tuần hoàn các nguyên tố có bao nhiêu cột, bao nhiêu nhóm A, bao nhiêu nhóm B?
	\choice
	{\True 18 cột được chia thành 8 nhóm A và 10 nhóm B}
	{18 cột được chia thành 10 nhóm A và 8 nhóm B}
	{18 cột được chia thành 9 nhóm A và 9 nhóm B}
	{18 cột được chia thành 8 nhóm A và 8 nhóm B}
	\loigiai{%
		Bảng tuần hoàn hiện nay chia thành 18 cột gồm 8 nhóm A (mỗi nhóm 1 cột) và 8 nhóm B riêng nhóm VIIIB có 3 cột
	}
\end{ex}
%%%=============EX_7=============%%%
\begin{ex}%[0H2H1-1]
	Ô nguyên tố không cho biết thông tin nào sau đây?
	\choice
	{Kí hiệu nguyên tố}
	{Tên nguyên tố}
	{Số hiệu nguyên tử}
	{\True Số khối của hạt nhân}
	\loigiai{
		Ô nguyên tố cho biết kí hiệu nguyên tố, tên nguyên tố, số hiệu nguyên tử.
	}
\end{ex}
%%%=============EX_8=============%%%
\begin{ex}%[0H2H1-1]
	Nguyên tố nào sau đây là nguyên tố nhóm A?
	\choice
	{\True Gồm các nguyên tố s, p}
	{Gồm các nguyên tố p, d}
	{Gồm các nguyên tố d, f}
	{Gồm các nguyên tố s, d}
	\loigiai{
		Các nguyên tố nhóm A thuộc nguyên tố s và p còn các nguyên tố nhóm B thuộc nguyên tố d và f
	}
\end{ex}
%%%=============EX_9=============%%%
\begin{ex}%[0H2H1-1]
	Cho các nguyên tố: $X: 1s^22s^22p^63s^2, Y: 1s^22s^2, Z: 1s^22s^22p^63s^23p^63d^14s^2, T: 1s^22s^22p^63s^23p^5.$ Các nguyên tố cùng chu kì là:
	\choice
	{X, Y}
	{\True X, T}
	{Y, Z}
	{X, Z}
	\loigiai{%
		Các nguyên tố cùng chu kì có cùng số lớp electron. Ta thấy X , T đều có 3 lớp electron, Y có 2 lớp electron, Z có 4 lớp electron.
	}
\end{ex}
%%%=============EX_10=============%%%
\begin{ex}%[0H2V1-1]
	Nguyên tố hóa học nào sau đây có tính chất hóa học tương tự Calcium?
	\choice
	{Carbon}
	{Potassium}
	{Sodium}
	{\True Strontium}
	\loigiai{%
		Strontium và Calcium cùng thuộc nhom IIA nên có tính chất tương tự nhau
	}
\end{ex}
%%%=============EX_11=============%%%
\begin{ex}%[0H2H1-1]
	Cặp nguyên tố hóa học nào sau đây có tính chất hóa học giống nhau?
	\choice
	{\True Ca và Mg}
	{P và S}
	{S và Cu}
	{N và O}
	\loigiai{
		Ca và Mg cùng thuộc nhomms IIA nên có tính chất hóa học giống nhau
	}
\end{ex}
%%%=============EX_12=============%%%
\begin{ex}%[0H2V1-1]
	Đại lượng nào sau đây trong nguyên tử của các nguyên tố biến đổi tuần hoàn theo chiều tăng của điện tích hạt nhân nguyên tử?
	\choice
	{Số lớp electron}
	{\True Số electron lớp ngoài cùng}
	{Nguyên tử khối}
	{Số electron trong nguyên tử}
	\loigiai{Sau mỗi chu kì số electron lớp ngoài cùng sẽ tăng dần từ 1 đến 8 trong nhóm A}
\end{ex}
%%%=============EX_13=============%%%
\begin{ex}%[0H2V1-1]
	Xét các nguyên tố nhóm IA của bảng tuần hoàn, điều khẳng định nào sau đây là đúng?
	Các nguyên tố nhóm IA:
	\choice
	{Được gọi là các kim loại kiềm thổ}
	{Dễ dàng cho 2 electron hóa trị lớp ngoài cùng}
	{\True Dễ dàng cho 1 electron để đạt cấu hình bền vững}
	{Dễ dàng nhận thêm 1 electron để đạt cấu hình bền vững}
	\loigiai{%
		Các nguyên tố nhóm IA có 1 electron lớp ngoài cùng nên có khuynh hướng nhường đi 1 elctron để đạt cấu hình bền của khí hiếm trước nó.
	}
\end{ex}
%%%=============EX_14=============%%%
\begin{ex}%[0H2H1-1]
	Cấu hình electron hóa trị của nguyên tử các nguyên tố nhóm IIA trong bảng tuần hoàn đều là:
	\choice
	{\True $ns^2$}
	{$np^2$}
	{$ns^2np^2$}
	{$ns^2np^4$}
	\loigiai{%
		Cấu hình electron hóa trị của nguyên tử các nguyên tố nhóm IIA trong bảng tuần hoàn là $ns^2$
	}
\end{ex}
%%%=============EX_15=============%%%
\begin{ex}%[0H2V1-1]
	Nguyên tố nào sau đây là nguyên tố thuộc nhóm B?
	\choice
	{Ca (Z=20)}
	{\True Fe (Z=26)}
	{K (Z=19)}
	{Na (Z=11)}
	\loigiai{
		$Fe$ là nguyên tố nhóm B vì có electron cuối cùng điền vào phân lớp $3d$
	}
\end{ex}
%%%=============EX_16=============%%%
\begin{ex}%[0H2H1-1]
	Trong một chu kì, theo chiều tăng của điện tích hạt nhân nguyên tử
	\choice
	{bán kính nguyên tử và độ âm điện đều giảm}
	{bán kính nguyên tử và độ âm điện đều tăng}
	{bán kính nguyên tử tăng, độ âm điện giảm}
	{\True bán kính nguyên tử giảm, độ âm điện tăng}
	\loigiai{%
		Trong một chu kì khi đi từ trái sang phải theo chiều tăng của điện tích hạt nhân thì bán kính nguyên tử giảm , độ âm điện tăng dần.
	}
\end{ex}
%%%=============EX_17=============%%%
\begin{ex}
	Các nguyên tố nhóm VIIA của bảng tuần hoàn, điều khẳng định nào sau đây là đúng?
	\choice
	{Các nguyên tố nhóm VIIA gọi là nhóm kim loại kiềm}
	{Dễ dàng cho 2e hóa trị lớp ngoài cùng}
	{Dễ dàng cho 1e hóa trị để đạt cấu hình bền vững}
	{\True Dễ dàng nhận thêm 1e để đạt cấu hình bền vững}
	\loigiai{
		Các nguyên tố nhóm VII A có 7 electron ở lớp ngoài cùng có xu hướng nhận thêm 1 electron để đạt cấu hình bền của khí hiếm
	}
\end{ex}
%%%=============EX_18=============%%%
\begin{ex}%[0H2H1-1]
	Trong một nhóm A (trừ nhóm VIII) theo chiều tăng của điện tích hạt nhân nguyên tử
	\choice
	{tính kim loại tăng dần, độ âm điện tăng dần}
	{\True tính phi kim giảm dần, bán kính nguyên tử tăng dần}
	{độ âm điện giảm dần, tính phi kim tăng dần}
	{tính kim loại tăng dần, bán kính nguyên tử giảm dần}
	\loigiai{%
		Trong một nhóm kh đi từ trên xuống dưới theo chiều tăng của điện tích hạt nhân  tính kim loại tăng dần, tính phi kim giảm dần , bán kính nguyên tử tăng dần.
	}
\end{ex}
%%%=============EX_19=============%%%
\begin{ex}%[0H2H1-1]
	Các nguyên tố trong cùng nhóm A có đặc điểm chung nào sau đây?
	\choice
	{\True Số electron hóa trị}
	{Số lớp electron}
	{Số electron lớp L}
	{Số phân lớp electron}
	\loigiai{%
		Các nguyên tố trong cùng nhóm A có cùng số electron hóa trị
	}
\end{ex}
%%%=============EX_20=============%%%
\begin{ex}%[0H2V1-1]
	Phát biểu nào sau đây không đúng?
	\choice
	{Nguyên tử có $Z=11$ có bán kính nhỏ hơn nguyên tử có $Z=19$}
	{Nguyên tử có $Z=12$ có bán kính lớn hơn nguyên tử có $Z=10$}
	{\True Nguyên tử có $Z=11$ có bán kính nhỏ hơn nguyên tử có $Z=13$}
	{Các nguyên tố kim loại kiềm có bán kính nguyên tử lớn nhất trong chu kì}
	\loigiai{%
		Nguyên tử $Z =11$ là kim laoị kiềm có bán kính lớn hơn nguyên tử có $Z=13$.
	}
\end{ex}
%%%=============EX_21=============%%%
\begin{ex}%[0H2H1-1]
	Những nguyên tố cuối chu kì có đặc điểm gì?
	\choice
	{\True Có 8e lớp ngoài cùng}
	{Có 1e lớp ngoài cùng}
	{Dễ dàng nhận thêm 1e}
	{Có 2e lớp ngoài cùng}
	\loigiai{%
		Những nguyên tố cuối chu kì có 8 electron ở lớp ngoài cùng.
	}
\end{ex}
%%%=============EX_22=============%%%
\begin{ex}%[0H2V1-1]
	Nguyên tố có $Z=7$. Nguyên tố đó thuộc nhóm:
	\choice
	{\True $VA$}
	{$VIA$}
	{$VIIA$}
	{$VIIIA$}
	\loigiai{%
		Nguyên tố có $Z=7$ có cấu hình electron là $1s^22s^22p^3$ $\Rightarrow$ Z có 5 electron lớp ngoài cùng nên thuộc nhóm VA.
	}
\end{ex}
%%%=============EX_23=============%%%
\begin{ex}%[0H2H1-1]
	Cho các nguyên tố sau: Li, Na, K, Cs. Nguyên tử của nguyên tố có bán kính bé nhất là
	\choice
	{Li}
	{Na}
	{K}
	{Cs}
	\loigiai{%
		Các nguyên tố Li, Na, K, Cs thuộc nhóm IA theo thứ tự trên bán kính tăng dần theo chiều tăng của điện tích hạt nhân Z. Do đó nguyên tử có bán kính nhỏ nhất là Li.
	}
\end{ex}
%%%=============EX_24=============%%%
\begin{ex}%[0H2V1-1]
	Nguyên tố có $Z=20$. Nguyên tố đó thuộc chu kì:
	\choice
	{$1$}
	{$2$}
	{$3$}
	{$4$}
	\loigiai{%
		Nguyên tố có $Z=20$ có cấu hình electron là: $1s^22s^22p^63s^23p^64s^2$
		$\Rightarrow$ có 4 lớp electron do đó thuộc chu kì 4.
	}
\end{ex}
%%%=============EX_25=============%%%
\begin{ex}%[0H2H1-2]
	Phát biểu nào sau đây không đúng?
	\choice
	{\True Nguyên tử có bán kính nhỏ nhất có $Z=1$}
	{Kim loại yếu nhất trong nhóm IA có $Z=3$}
	{Nguyên tố có độ âm điện lớn nhất có $Z=9$}
	{Phi kim mạnh nhất trong nhóm VA có $Z=7$}
	\loigiai{
		Nguyên tử có bán kính nhỏ nhất là $He$
	}
\end{ex}
%%%=============EX_26=============%%%
\begin{ex}%[0H2H1-2]
	M là nguyên tố thuộc chu kì 4 và số electron lớp ngoài cùng của M là 1. M là:
	\choice
	{\True K}
	{Mg}
	{Ca}
	{Na}
	\loigiai{M thuộc chu kì 4 và ở nhóm IA nên M có cấu hình electron là $1s^22s^22p^63s^23p^64s^1$ $\Rightarrow$ M có $Z=19$ vậy M là K}
\end{ex}
%%%=============EX_27=============%%%
\begin{ex}%[0H2V1-2]
	X là nguyên tố nhóm III .Công thức oxide cao nhất của X là:
	\choice
	{XO}
	{$XO_2$}
	{\True $X_2O_3$}
	{$X_2O$}
	\loigiai{
		X là nguyên tố nhóm III $\Rightarrow$ hoa trị cao nhất trong hợp chất với oxi là III, do đó công thức oxide cao nhất cảu X là $X_2O_3$
	}
\end{ex}
%%%=============EX_28=============%%%
\begin{ex}%[0H2H1-2]
	Những tính chất nào sau đây biến đổi tuần hoàn?
	\choice
	{Số lớp e}
	{\True Số e lớp ngoài cùng}
	{Nguyên tử khối}
	{Điện tích hạt nhân}
	\loigiai{Số electron lớp ngoài cùng tăng dần từ 1 đến 8 theo chiều từ trái sang phải sau mỗi chu kì}
\end{ex}
%%%=============EX_29=============%%%
\begin{ex}%[0H2H1-2]
	Cho các oxide sau:  $Na_2O$,  $Al_2O_3$, $MgO$,  $SiO_2$.
	Thứ tự giảm dần tính base là:
	\choice
	{$Na_2O$ >  $Al_2O_3$ > $MgO$ >  $SiO_2$}
	{$Al_2O_3$ >  $SiO_2$ > $MgO$ >  $Na_2O$}
	{$Na_2O$ > $MgO$ >  $Al_2O_3$ >  $SiO_2$}
	{$MgO$ >  $Na_2O$ >  $Al_2O_3$ >  $SiO_2$}
	\loigiai{
		Trong một chu kì đi từ trái sang phải tính base của các oxit giảm dần do đó tính base giảm dần trong dãy trên $Na_2O$ > $MgO$ >  $Al_2O_3$ >  $SiO_2$.
	}
\end{ex}
%%%=============EX_30=============%%%
\begin{ex}%[0H2V1-2]
	Dãy nào sau đây sắp xếp theo thứ tự tăng dần tính acid?
	\choice
	{ $Cl_2O_7$,  $Al_2O_3$,  $SO_3$,  $P_2O_5$}
	{ $Al_2O_3$,  $P_2O_5$,  $SO_3$,  $Cl_2O_7$}
	{ $P_2O_5$,  $SO_3$,  $Al_2O_3$,  $Cl_2O_7$}
	{ $Al_2O_3$,  $SO_3$,  $P_2O_5$,  $Cl_2O_7$}
	\loigiai{%
		Trong một chu kì khi đi từ trái sang phải tính acid tăng dần do đó tính acid tăng dần trong dãy trên là $Al_2O_3$,  $P_2O_5$,  $SO_3$,  $Cl_2O_7$.
	}
\end{ex}
%%%=============EX_31=============%%%
\begin{ex}%[0H2V1-2]
	Điện tích hạt nhân của các nguyên tử là: $X$ $(Z=6)$, $Y$ $(Z=7)$, $M$ $(Z=20)$, $Q$ $(Z=19)$. Nhận xét nào đúng?
	\choice
	{$X$ thuộc nhóm $VA$}
	{$Y$, $M$ thuộc nhóm $IIA$}
	{$M$ thuộc nhóm $IIB$}
	{\True $Q$ thuộc nhóm $IA$}
	\loigiai{%
		\begin{itemize}
			\item $X$ $(Z=6)$ có cấu hình e là : 1s22s22p2 $\Rightarrow$ $X$ thuộc nhóm $IVA$
			\item $Y$ $(Z=7)$ có cấu hình e là : 1s22s22p3 $\Rightarrow$ $Y$ thuộc nhóm $VA$
			\item $M$ $(Z=20)$ có cấu hình e là : 1s22s22p63s23p64s2 $\Rightarrow$ $M$ thuộc nhóm $IIA$
			\item $Q$ $(Z=19)$ có cấu hình e là : 1s22s22p63s23p64s1 $\Rightarrow$ $Q$ thuộc nhóm $IA$
		\end{itemize}
	}
\end{ex}
%%%=============EX_32=============%%%
\begin{ex}%[0H2H1-2]
	Nguyên tố nào có tính kim loại mạnh nhất?
	\choice
	{Na}
	{K}
	{Mg}
	{Li}
	\loigiai{%
		Các nguyên tố ở đầu chu kì và cuối nhóm có tính kim loại mạnh nhất $\Rightarrow$
		nguyên tố K là kim loại mạnh nhất trong số các kim loại đã cho.}
\end{ex}
%%%=============EX_33=============%%%
\begin{ex}%[0H2H1-2]
	Ba nguyên tố với số hiệu nguyên tử $Z=11$, $Z=12$, $Z=13$ có hydroxide tương ứng là $X$, $Y$, $T$. Chiều tăng dần tính base của các hydroxide này là:
	\choice
	{\True $X$, $Y$,$ T$}
	{$X$, $T$, $Y$}
	{$T$, $X$, $Y$}
	{$T$, $Y$, $X$}
	\loigiai{%
		Ba nguyên tố có hydroxide tương ứng là $X$, $Y$, $T$ thuộc cùng chu kì do đó theo chiều tăng dần điện tích hạt nhân tính bazơ giảm dần theo thứ tự là $X$, $Y$, $T$.
	}
\end{ex}
%%%=============EX_34=============%%%
\begin{ex}%[0H2H1-2]
	Trong một chu kì khi đi từ trái sang phải:
	\choice
	{Tính kim loại và tính phi kim tăng dần}
	{Tính kim loại và tính phi kim giảm}
	{Tính kim loại tăng tính phi kim giảm}
	{\True Tính kim loại giảm tính phi kim tăng}
	\loigiai{%
		Trong một chu kì khi đi từ trái sang phải tính kim loại giảm dần và tính phi kim tăng dần.
	}
\end{ex}
%%%=============EX_35=============%%%
\begin{ex}%[0H2H1-2]
	Tính chất nào sau đây không biến đổi tuần hoàn?
	\choice
	{\True Nguyên tử khối}
	{Số electron lớp ngoài cùng}
	{Hóa trị cao nhất với oxygen}
	{Thành phần các oxide và hydroxide cao nhất}
	\loigiai{
		Nguyên tử khối là đại lượng không biến đổi tuần hoàn
	}
\end{ex}
%%%=============EX_36=============%%%
\begin{ex}%[0H2H1-2]
	Trong các hydroxide của các nguyên tố chu kì 3, acid mạnh nhất là:
	\choice
	{$H_2SO_4$}
	{\True $HClO_4$}
	{$H_2SiO_3$}
	{$H_3PO_4$}
	\loigiai{%
		Trong một chu kì đi từ trái sang phải tính acid của các nguyên tố tăng dần do đó $HClO_4$ là acid mạnh nhất.
	}
\end{ex}
%%%=============EX_37=============%%%
\begin{ex}%[0H2H1-2]
	Trong một chu kì theo chiều tăng của điện tích hạt nhân nguyên tử:
	\choice
	{Tính kim loại tăng}
	{Tính phi kim giảm}
	{\True Hóa trị cao nhất với oxygen tăng}
	{Hóa trị cao nhất với hydrogen không đổi}
	\loigiai{
		Trongmmột chu kì Tính kim laoị giảm , tính phi kim tăng, hóa trị cao nhất với oxigen tăng và hóa trị cao nhất với hydrogen giảm từ 4 về 1 bắt đầu ntừ nhóm IVA.
	}
\end{ex}
%%%=============EX_38=============%%%
\begin{ex}%[0H2V1-2]
	Dãy nào sau đây sắp xếp theo thứ tự giảm dần tính base?
	\choice
	{$Al(OH)_3$, $NaOH$, $Mg(OH)2$, $Si(OH)4$}
	{$NaOH$, $Mg(OH)_2$, $Si(OH)_4$, $Al(OH)_3$}
	{$NaOH$, $Mg(OH)_2$, $Al(OH)_3$, $Si(OH)_4$}
	{$Si(OH)_4$, $NaOH$, $Mg(OH)_2$, $Al(OH)_3$}
	\loigiai{
		Trong một hcu kì tính base giảm dần do đo thứ tự tính base giảm dần trong dãy trên là: $NaOH$, $Mg(OH)_2$, $Al(OH)_3$, $Si(OH)_4$
	}
\end{ex}
%%%=============EX_39=============%%%
\begin{ex}%[0H2H1-2]
	Trong một chu kì theo chiều tăng của điện tích hạt nhân nguyên tử:
	\choice
	{Tính kim loại giảm}
	{Tính phi kim giảm}
	{Hóa trị cao nhất với oxygen giảm}
	{Hóa trị cao nhất với hidrogen tăng}
	\loigiai{%
		Trong một chu kì theo chiều tăng của điện tích hạt nhân nguyên tử  tính kim loại giảm , tính ohi kim tăng, hóa trị cao nhất với oxigen tăng  từ 1 đến 8, còn hóa trị cao nhất với hydrogen giảm dần từ IV về I bắt đầu từ nhóm IVA.
	}
\end{ex}

\Closesolutionfile{ans}
\Closesolutionfile{ansex}
%\bangdapan{Ans-C02-CTBTHTH.tex}
\phan{Bài tập trắc nghiệm đúng sai}
%%%=============SOẠN EXTF===============%%%
\Opensolutionfile{ansex}[Ans/LGTF-C02-CTBTHTH.tex]
\Opensolutionfile{ansbook}[Ansbook/AnsTF-C02-CTBTHTH.tex]
\Opensolutionfile{ans}[Ans/Tempt-C02B01-CTBTHTH.tex]
\luulgEXTF
%%%=========extf_1=========%%%
\begin{ex}
	\choiceTF[t]
	{}
	{}
	{}
	{}
	\loigiai{}
\end{ex}
%%%=========extf_2=========%%%
\begin{ex}
	\choiceTF[t]
	{}
	{}
	{}
	{}
	\loigiai{}
\end{ex}
\Closesolutionfile{ans}
\Closesolutionfile{ansbook}
\Closesolutionfile{ansex}
%\bangdapanTF{AnsTF-C02-CTBTHTH.tex}

\phan{Bài tập tự luận}
%\luuloigiaibt
\hienthiloigiaibt
\begin{dang}{Xác định vị trí nguyên tố dựa vào cấu hình electron và ngược lại}
\end{dang}
%%%=============SOẠN BT===============%%%
\Opensolutionfile{ansbth}[Ans/LGBT-C02-CTBTHTH01.tex]
\Opensolutionfile{ansbt}[Ans/AnsBT-C02-CTBTHTH01.tex]
%%%=========bt_1=========%%%
\begin{bt} Magnesium là nguyên tố phổ biến thứ 8 trong lớp vỏ của Trái Đất, ở điều kiện thường là chất rắn, có màu trắng bạc, rất nhẹ. Magnesium được sử dụng để làm cho hợp kim bền nhẹ, đặt biệt là cho ngành công nghiệp hàng không vũ trụ, cũng như sử dụng trong pháo hoa bởi vì nó đốt cháy với một ngọn lửa trắng rực rỡ.
Trong bảng tuần hoàn, magnessium là nguyên tố có kí hiệu $Mg$ nằm ở chu kì $3$, nhóm $IIA$. Hãy cho biết:
	\begin{enumerate}[a)]
		\item Nguyên tử $Mg$ có bao nhiêu  electron thuộc lớp ngoài cùng?
		\item Các electron lớp ngoài cùng thuộc những phân lớp nào?
		\item  Viết cấu hình electron nguyên tử của $Mg$?
		\item $Mg$ laf nguyên tố  kim loại hay phi kim?
	\end{enumerate}
	\loigiai{%
	\begin{enumerate}[a)]
		\item Vì $Mg$ thuộc nhóm $IIA$ nên có 2 electron lớp ngoài cùng.
		\item Các electron lớp ngoài cùng  thuộc phân lớp s.
		\item Vì $Mg$ thuộc chu kì 3 nên có 3 lớp electron. Cấu hình electron: $1s^22s^22p^63s^2$.
		\item $Mg$ là nguyên tố kim loại vì có 2 electron lớp ngoài cùng.
	\end{enumerate}
	}
\end{bt}
%%%=========bt_2=========%%%
\begin{bt}
	Aluminium (Al) là một nguyên tố kim loại phổ biến thứ ba trong vỏ trái đất, chỉ sau oxy và silicon. Ở điều kiện thường, Al là một kim loại nhẹ, mềm, dẫn điện tốt và dẻo.  Al được sử dụng rộng rãi trong nhiều lĩnh vực như xây dựng, giao thông vận tải, điện tử và hóa chất.
	Trong bảng tuần hoàn, Al có ký hiệu $Al$ và nằm ở chu kì $3$, nhóm $IIIA$. Hãy cho biết:
	\begin{enumerate}[a)]
		\item Nguyên tử $Al$ có bao nhiêu electron thuộc lớp ngoài cùng?
		\item Các electron lớp ngoài cùng thuộc những phân lớp nào?
		\item Viết cấu hình electron nguyên tử của $Al$?
		\item $Al$ là nguyên tố kim loại hay phi kim?
	\end{enumerate}
	\loigiai{%
		\begin{enumerate}[a)]
			\item Vì $Al$ thuộc nhóm $IIIA$ nên có 3 electron lớp ngoài cùng.
			\item Các electron lớp ngoài cùng của $Al$ thuộc phân lớp $p$.
			\item Cấu hình electron của $Al$ là: $1s^22s^22p^63s^23p^1$.
			\item $Al$ là nguyên tố kim loại vì có 3 electron lớp ngoài cùng.
		\end{enumerate}
	}
\end{bt}
\Closesolutionfile{ansbt}
\Closesolutionfile{ansbth}
%\bangdapanSA{AnsBT-C02-CTBTHTH01.tex}
%%%%%%%%%%%%%%%%%%%%%%%%%%%%
\begin{dang}{Xu hướng biến đổi một số tính chất của nguyên tử các nguyên tố trong một chu kì và nhóm}
\end{dang}
%%%%%%%%%%%%%%%%%%%%%%%%%%%
\begin{dang}{Xu hướng biến đổi thành phần và một số tính chất của hợp chất trong một chu kì}
\end{dang}

\begin{dang}{Định luật tuần hoàn và ý nghĩa của bảng tuần hoàn}
\end{dang}

\end{document}
